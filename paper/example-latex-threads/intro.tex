\section{Introduction}

The ubiquitous availability of parallel computing infrastructures
facilitated by the advent of multicore architectures requires a shift
towards multi-threaded programming to take full advantage of the
available computing resources.
Writing correct multi-threaded software is a difficult task, as the
programmer needs to keep track of a very large number of possible
interactions between the program threads.
Automated program analysis and verification tools can support
programmer in dealing with this challenge by systematically and
exhaustively exploring program behaviours and checking their
correctness.

Direct treatment of all possible thread interleavings by reasoning
about the program globally is a prohibitively expensive task, even for
small programs. 
By applying rely-guarantee techniques, see
e.g.~\cite{RelyGuarantee,OwickiAI76}, such global reasoning can be
avoided by considering each program thread in isolation, using
environment transitions to summarize the effect of executing other
threads, and applying them on the thread at hand.
The success of such an approach depends on the ability to
automatically discover environment transitions that are precise enough
to deliver a conclusive analysis/verification outcome, and yet do not
keep track of unnecessary details in order to avoid sub-optimal
efficiency.

%\todo{Modular proofs play an important role in rely-guarantee
%approaches.}
  
In this paper we present a method that automates rely-guarantee
reasoning for verifying safety of multi-threaded programs.
Our method relies on an automated discovery of environment transitions
using (transition) predicate abstraction~\cite{GrafSaidi,TransPred05}.
It performs a predicate abstraction-based reachability computation for
each thread and interleaves it with the construction of environment
transitions that over-approximate the effect of executing thread
transitions using transition predicates.
The success of our method crucially depends on an abstraction
refinement procedure that discovers (transition) predicates.
The refinement procedure attempts to minimize the amount of details
that are exposed by the environment transitions, in order to avoid
unnecessary details about thread interaction.

The crux of our refinement approach is in using a declarative
formulation of the abstraction refinement algorithm that can deal with
the thread reachability, environment transitions, and their mutual
dependencies.
We use Horn clauses to describe constraints on the desired
(transition) predicates, and solve these constraints using a general
algorithm for recursion-free Horn clauses.
Our formalization can accommodate additional requirements that express
the preference for modular predicates that do not refer to the local
variables of environment threads, together with the preference for
modular transition predicates that only deal with global variables and
their primed versions.

We implemented the proposed method in a verification tool for
multi-threaded programs and applied it on a range of benchmarks,
which includes fragments of open source software, ticket-based mutual
exclusion protocols, and multi-threaded Linux device drivers.
%We observed that some of the discovered environment assumptions are
%thread-modular, while some others involve reasoning about thread-local
%variables.
The results of the experimental evaluation indicate that our
declarative abstraction refinement approach can be effective in
finding adequate environment transitions for the verification of
multi-threaded programs.

This paper makes the following contributions:
\begin{enumerate}
\item the automatic, rely-guarantee based method for verifying
multi-threaded programs using (transition) predicate abstraction;
%together with a corresponding abstraction refinement algorithm;
\item the novel formulation of abstraction refinement schemes using
Horn clauses, and its application for the (transition) abstraction
discovery for multi-threaded programs;
\item the algorithm for solving recursion-free Horn clauses over
linear arithmetic constraints;
\item the prototype implementation and its evaluation.
\end{enumerate}


The rest of the paper is organized as follows. 
First, we illustrate our method in Section~\ref{sec-illustration}.
In Section~\ref{sec-prelim} we present necessary definitions.
Section~\ref{sec-proof-rule} presents a proof rule that provides a
basis for our method, and shows how the proof rule can be automated
using the connection to fixpoints and abstraction techniques.
We present the main algorithm in Section~\ref{sec-algo-safety}.
Section~\ref{sec-refinement} focuses on the abstraction refinement using
Horn clauses, while Section~\ref{sec-horn-solving} presents a
constraint solving algorithm for Horn clauses over linear
inequalities.
We discuss the experimental evaluation in
Section~\ref{sec-experiments}.
Related work is presented in Section~\ref{sec-related}.



%%% Local Variables: 
%%% mode: latex
%%% TeX-master: "main"
%%% End: 
