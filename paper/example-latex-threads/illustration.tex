\section{Illustration}\label{sec-illustration}

In this section we illustrate our algorithm using two multi-threaded
examples.
The first example does not have a modular proof, hence our algorithm
reasons about relationship between the local variables of different
threads.
For the second example, our algorithm succeeds in finding a modular
proof by applying an abstraction refinement procedure that guarantees
the discovery of a modular abstraction whenever it exists.

\iffalse

Firstly, the algorithm generates thread reachability assertions
in the form of trees of reachable states for each thread.
If a reachable state is found to violate the mutual exclusion
property, then our algorithm generates a set of Horn clauses that
encodes how the abstract error state was reached.
If the set of Horn clauses is satisfiable, our solving procedure
returns a solution which is used for abstraction refinement.

The first multi-threaded program does not have a thread-modular proof
and hence its verification leads to environment transitions that
reveal some facts about the local state of the threads. 
The second program can be proven correct using a thread-modular proof,
however, non-modular proofs that (unnecessarily) involve local
variables can also be used to prove correctness of this program. 
\fi

\subsection{Example \TakeLockBit}\label{subsec-takebit}

See Figure~\ref{fig-take-lock} for the program \TakeLockBit that
consists of two threads.
The threads attempt to access a critical section, and synchronize their
accesses using a global variable~$\lock\ $.
We assume that initially the lock is not taken, i.e., $\lock = 0\ $,
and that the locking statement \verb|take_lock| waits until the lock
is released and then assigns the value of its second parameter
to~$\lock\ $, thus taking the lock.
We write $\Vars = (\lock, \pcone, \pctwo)$ for the program variables,
where $\pcone$ and $\pctwo$ are local program counter variables of the
first and second thread, respectively.

% ===================== Program description =============== 


We start by representing the program using assertions $\SymbInit$ and
$\SymbError$ over program variables that describe the initial and
error states of the program, together with assertions over unprimed
and primed program variables $\rel_1$ and $\rel_2$ that describe the
transition relations for program statements.
%
\begin{equation*}
  \begin{array}[t]{r@{}l@{}}
    \SymbInit = (&\pcone=a \land \pctwo=p \land \lock=0)\ , \\[\jot]
    \SymbError = (&\pcone=b \land \pctwo=q)\ ,\\[\jot]
    \rel_1 = (&\lock=0 \land \lock'=1 \land \pcone=a \land
    \pcone'=b \land  \relLocEqOf{2})\ ,
    \\[\jot]
    \relLocEqOf{1} = (&\pcone=\pcone')\ , \\[\jot]
    \rel_2 = (&\lock=0 \land \lock'=1 \land \pctwo=p \land \pctwo'=q
    \land \relLocEqOf{1})\ , \\[\jot]
    \relLocEqOf{2} = (&\pctwo=\pctwo')\ .
 \end{array}
\end{equation*} 
%
The auxiliary assertions $\relLocEqOf{1}$ and $\relLocEqOf{2}$ state
that the local variable of the first and second thread, respectively,
is preserved during the transition.

\iffalse
The transition relations $\rel_1$ and $\rel_2$ refer to all program
variables, i.e., the global variable $\lock$ as well as the local
variables $\pcone$ and $\pctwo\ $.

For example, the transition relation $\rel_1$ of the first thread
requires that the local variable of the second thread does not change,
which is expressed by the conjunct $\pctwo = \pctwo'$ in~$\rel_1\ $.
\fi

To verify \TakeLockBit, our algorithm computes a sequence of \aret{}s
(Abstract Reachability and Environment Trees).
Each tree computation amounts to a combination of i) a standard
abstract reachability computation that is performed for each thread
and is called thread reachability, and ii) a construction and
application of environment transitions.
Abstract states represent sets of (concrete) program states, while
environment transitions are binary relations of program states.

\begin{figure}[h!]
\hrule
\centering
\begin{minipage}[t]{.48\columnwidth} 
\begin{verbatim} 

// Thread 1 
a: take_lock(lock, 1); 
b: // critical 
\end{verbatim}
\end{minipage}
%
\begin{minipage}[t]{.48\columnwidth} 
\begin{verbatim}

  // Thread 2
  p: take_lock(lock, 1);
  q: // critical

\end{verbatim}
\end{minipage}
\caption{Example program \TakeLockBit.
  Each thread waits until the lock is released, and then assigns the
integer $1$ to~$\lock\ $.
}
\label{fig-take-lock}
\end{figure}


% ===================== Iteration 1 =======================
\paragraph{First \aret computation}


The thread reachability computation for the first thread starts by
computing an abstraction of the initial program states~$\SymbInit\ $.
Here, we use an abstraction function $\abst_1\ $, where the dot
indicates that this function over-approximates sets of program states
(and not sets of pairs of states, as will take place later) and the
index $1$ indicates that this abstraction function is used for the
first thread.
In this example, we assume that the abstraction function only tracks
the value of the program counter of the first thread, i.e.,
$\artPreds_1 = \set{\pcone=a, \pcone=b}\ $, and is computed as
follows: $\abst_1(S) = \Land \set{\dot{p} \in \artPreds_1 \mid \forall
\Vars:S \limplies \dot{p}}\ $.
We obtain the initial abstract state $m_1$ as follows:
%
\begin{equation*}
  m_1  = \abst_1(\SymbInit) = (\pcone = a)\ .
\end{equation*}
%
 Next, we compute an abstract successor of $m_1$ with respect to the
transition~$\rel_1$ using the strongest postcondition operator $\post\
$ that is combined with $\abstTIdOf{1}$:
%
\begin{equation*}
  m_2 = \abst_1(\post(\rel_1, m_1)) = (\pcone = b)\ .
\end{equation*}

\begin{figure}[t]
  \hrule
  \centering
  \vspace{2ex} 
% ===================== ART computed in 1st iteration
(a)
  \begin{tikzpicture}[->, >=stealth', shorten >=1pt, auto, 
      node distance=\nodeDistance, semithick]
%    \tikzstyle{every stateBox}=[draw=black]
    % 1st thread
    \node[stateBox] (m1) {$m_1$};
    \node[stateBox, errorState, below of=m1] (m2) {$m_2$};
    \path (m1) edge node[left]{$\rel_1$} (m2);
    % 2nd thread
    \node[stateBox,right of=m1, node distance=\artDistance{1}] (n1) {$n_1$};
    \node[stateBox, errorState, below of=n1] (n2) {$n_2$};
    \path (n1) edge node[right]{$\rel_2$} (n2);
  \end{tikzpicture}
  \hspace{2cm}
% ===================== ART computed in 2nd iteration
(b)
  \begin{tikzpicture}[->, >=stealth', shorten >=1pt, auto, 
      node distance=\nodeDistance, semithick]
%    \tikzstyle{every state}=[draw=black]
    % 1st thread
    \node[stateBox] (m1) {$m_1$};
    \node[stateBox, below of=m1] (m2) {$m_2$};
    \path (m1) edge node[left]{$\rel_1$} (m2);
    \node[stateBox, errorState, below of=m2] (m3) {$m_3$};
    \path (m2) edge node[left]{$e_2$} (m3);
    % 2nd thread
    \node[stateBox,right of=m1, node distance=\artDistance{1}] (n1) {$n_1$};
    \node[stateBox, below of=n1] (n2) {$n_2$};
    \path (n1) edge node[right]{$\rel_2$} (n2);
    \node[stateBox, errorState, below of=n2] (n3) {$n_3$};
    \path (n2) edge node[right]{$e_1$} (n3);
    % environment arrows
    \path (0+\shiftEnv,-\nodeDistance/2) edge[envTransition] % edge[out=0,in=180]
    (\artDistance{1}-\shiftEnv,-\nodeDistance-\nodeDistance/2);
    \path (\artDistance{1}-\shiftEnv, -\nodeDistance/2) edge[envTransition]% edge[out=180,in=0]
    (\shiftEnv,-\nodeDistance-\nodeDistance/2);
  \end{tikzpicture}
  \vspace{2ex} \hrule \vspace{2ex}
% ===================== ART computed in 4th iteration
\renewcommand{\nodeDistance}{2cm}
(c)\quad
  \begin{tikzpicture}[->, >=stealth', shorten >=1pt, auto, 
      node distance=\nodeDistance, semithick]
%    \tikzstyle{every state}=[draw=black]
    % 1st thread    
    \node[stateBox] (m1) {$m_1$};
    \node[stateBox, below of=m1, xshift=\shiftLeft] (m2) {$m_2$};
    \path (m1) edge node[left, near start]{$\rel_1$} (m2);
    \node[stateBox, below of=m1, xshift=\shiftRight] (m3) {$m_3$};
    \path (m1) edge node[left, near end]{$e_2$} (m3);
    % 2nd thread
    \node[stateBox,right of=m1, node distance=\artDistance{2}] (n1) {$n_1$};
    \node[stateBox, below of=n1, xshift=\shiftRight] (n2) {$n_2$};
    \path (n1) edge node[right, near end]{$\rel_2$} (n2);
    \node[stateBox, below of=n1, xshift=\shiftLeft] (n3) {$n_3$};
    \path (n1) edge node[left, near start]{$e_1$} (n3);
    % environment arrows
    \path (-0.4cm+\shiftEnv,-\nodeDistance/2+0.3cm) 
      edge[envTransition] %edge[out=5,in=165] 
      (\artDistance{2}-0.4cm-\shiftEnv,-\nodeDistance/2+0.3cm);
    \path (\artDistance{2}+0.6cm-\shiftEnv,-\nodeDistance/2-0.3cm) 
      edge[envTransition] %edge[out=185,in=345] 
      (0.4cm+\shiftEnv,-\nodeDistance/2-0.3cm);
  \end{tikzpicture}
  \caption{
    Reachability trees constructed using different abstraction functions.
    Edges are labeled with a transition. 
    Nodes with gray background represent (spurious) error tuples:
    $(m_2, n_2)$ from (a) and $(m_3, n_3)$ from (b).
    No pair of states from (c) intersects $\SymbError$.
  }
  \label{fig-ex-takebit-art}
\end{figure}

% System of coordinates (x,y):
%    (0,0)      (1,0)
%
%    (0,-1)     (1,-1)
%

%%% Local Variables: 
%%% mode: latex
%%% TeX-master: "main"
%%% End: 
 

Similarly, we compute the thread reachability for the second thread.
Using predicates over the program counter of the second thread, i.e.,
$\artPreds_2 = \set{\pctwo=p, \pctwo=q}\ $, we compute the following
two abstract states:
%
\begin{equation*}
  \begin{array}[t]{l@{\;=\;}l@{\;=\;}l}
    n_1 &  \abst_2(\SymbInit) &  (\pctwo = p)\ ,\\[\jot]
    n_2 &  \abst_2(\post(\rel_2, n_1)) &  (\pctwo = q)\ .
  \end{array}
\end{equation*}
%
For each thread, we organize the computed abstract states in a tree,
see Figure~\ref{fig-ex-takebit-art}(a). 

We stop the \aret computation since we discover that the error states
overlap with the intersection of the abstract state $m_2$ from the
thread reachability of the first thread and $n_2$ from the second
thread, i.e., $m_2 \land n_2 \land \SymbError$ is satisfiable.


\paragraph{First abstraction refinement} 

We treat the pair $m_2$ and $n_2$ as a possible evidence that the
error states of the program can be reached.
Yet, we cannot assert that the program is incorrect, since abstraction
was involved when computing $m_2$ and $n_2\ $.

We check if the discovered evidence is spurious by formulating a
constraint that is satisfiable if and only if the abstraction can be
refined to exclude the spuriousness.
For each abstract state involved in the reachability of and including
$m_2$ and $n_2$ we create an unknown predicate that denotes a set of
program states. 
We obtain $\unkOf{m_1}(\Vars)$, $\unkOf{n_1}(\Vars)$,
$\unkOf{m_2}(\Vars)$, and~$\unkOf{n_2}(\Vars)\ $, which correspond to
$m_1$, $n_1$, $m_2$, and~$n_2$, respectively.
Then, we record the relation between the unknown predicates using
constraints in the form of Horn clauses.
For example, since $m_1$ was an abstraction of the initial program
states, we require that $\unkOf{m_1}(\Vars)$ over-approximates
$\SymbInit$ as well, and represent this requirement by a Horn clause
$\SymbInit \limplies \unkOf{m_1}(\Vars)\ $.
As a result, we obtain the following set of clauses.
%
\begin{equation*}
  \HornClauses_1 = \set{
  \begin{array}[t]{@{}l}
    \SymbInit \limplies \unkOf{m_1}(\Vars),\ 
    \unkOf{m_1}(\Vars) \land \rel_1 \limplies \unkOf{m_2}(\Vars'),\\[\jot]
    \SymbInit \limplies \unkOf{n_1}(\Vars),\ 
    \unkOf{n_1}(\Vars) \land \rel_2 \limplies \unkOf{n_2}(\Vars'),\\[\jot]
    \unkOf{m_2}(\Vars) \land \unkOf{n_2}(\Vars) \land \SymbError \limplies 
    \lfalse }
  \end{array}
\end{equation*}
%
The last clause in $\HornClauses_1$ requires that the intersection of
the refined versions of the abstract states $m_2$ and $n_2$ is
disjoint from the error states of the program.

We check if the conjunction of the clauses in $\HornClauses_1$ is
satisfiable using a SAT-based algorithm presented
in~\cite{GuptaATVA10}.
(Section~\ref{sec-horn-solving} presents an algorithm for solving Horn
clauses over linear inequalities.)
We obtain the following satisfying assignment~$\Solution$ that maps
each unknown predicate to an assertion of the program variables.
%
\begin{equation*}
  \begin{array}[t]{l@{\;=\;}ll@{\;=\;}l}
    \SolutionOf{\unkOf{m_1}(\Vars)} & (\pctwo=p) & 
    \SolutionOf{\unkOf{n_1}(\Vars)} & (\pcone=a)\\[\jot]
    \SolutionOf{\unkOf{m_2}(\Vars)} & (\pctwo=p) & 
    \SolutionOf{\unkOf{n_2}(\Vars)} & (\pcone=a)
  \end{array}
\end{equation*} 
%
The existence of $\Solution$ indicates that the discovered evidence is
spurious. 
We use $\Solution$ to refine the abstraction functions and hence
eliminate the source of spuriousness.
We collect the predicates that appear in the solution for abstract
states from the first thread, add them to the sets of predicates
$\artPreds_1$, and perform a similar step for the second thread.
The resulting sets of predicates are shown below.
%
\begin{equation*}
  \begin{array}[t]{l@{\;=\;}l}
  \artPreds_1 & \set{\pcone=a, \pcone=b, \pctwo=p}\\[\jot]
  \artPreds_2 & \set{\pctwo=p, \pctwo=q, \pcone=a}
  \end{array}
\end{equation*}
%
They guarantee that the same spuriousness will not appear during
subsequent \aret computations.

\iffalse
\todo{Drop?}
We observe that the assignments for $\unkOf{m_1}(\Vars)$ and
$\unkOf{m_2}(\Vars)$ from the first thread involve the local variable
$\pctwo$ of the second thread, and the assignments for
$\unkOf{n_1}(\Vars)$ and $\unkOf{n_2}(\Vars)$ involve the variable
$\pcone\ $. 
While unavoidable for this constraint system, such solutions lead to
non modular reasoning in the next \aret computation.
\fi


% ===================== Iteration 2 =======================
\paragraph{Second \aret computation}
We re-start the \aret computation using the previously discovered 
predicates. 
Figure~\ref{fig-ex-takebit-art}(b) shows the two trees computed with
the refined abstraction functions where
%
\begin{equation*}
  \begin{array}[t]{l@{\;=\;}ll@{\;=\;}l}
    m_1 & (\pcone=a \land \pctwo=p)\ , & n_1 & (\pcone=a \land
    \pctwo=p)\ ,
    \\[\jot] 
    m_2 & (\pcone=b \land \pctwo=p)\ , & n_2 & (\pcone=a \land
    \pctwo=q)\ .
  \end{array}
\end{equation*}
%
Due to the first abstraction refinement step, $m_2 \land n_2 \land
\SymbError$ is unsatisfiable.
The thread reachability computation for each thread does not discover
any further abstract states.

The \aret computation proceeds by considering interleaving of the
transitions from one thread with the transitions from the other
thread. 
We account for thread interleaving by constructing and applying
environment transitions.
First, we construct an environment transition $e_1$ that records the
effect of applying $\rel_1$ on $m_1$ in the first thread on the thread
reachability in the second thread.
This effect is over-approximated by using an abstraction
function~$\abstTransTIdOf{1}{2}$.
In this function, the double dot indicates that the function abstracts
binary relations over states (and not sets of states).
The index $\envFromTo{1}{2}$ indicates that this function is applied
to abstract effect of the first thread on the second thread.
Initially, we use the empty set of transition predicates (over pairs
of states) $\artTPreds_{\envFromTo{1}{2}}=\emptyset\ $ to
define~$\abstTransTIdOf{1}{2}$.
The environment transition $e_1$ is defined as
%
\begin{equation*}
  e_1 = \abstTrans_{\envFromTo{1}{2}}(m_1 \land \rel_1) = \ltrue\ ,
\end{equation*}
%
and it non-deterministically updates the program variables (since
$\ltrue$ does not impose any restrictions on the successor states of
the transition).

Next, we add $e_1$ to the transitions of the second thread.
Then, its thread reachability computation uses $e_1$ during the
abstract successor computation, and creates an abstract state $n_3$ by
applying $e_1$ on~$n_2$ as follows:
%
\begin{equation*}
  n_3 = \abstTIdOf{2}(\post(e_1 \land \relLocEqOf{2}, n_2)) = (\pctwo
  = q)\ .
\end{equation*}
%
The conjunct $\relLocEqOf{2}$ ensures that the local variable of the
second thread is not changed by the environment transition.

Symmetrically, we use a function $\abstTransTIdOf{2}{1}$ to abstract
the effect of applying transitions in the second thread on the thread
reachability of the first thread.
The application of $\rel_2$ on $n_1$ results in the environment
transition $e_2$ such that
%
\begin{equation*}
  e_2 = \abstTransTIdOf{2}{1}(n_1 \land \rel_2) = \ltrue\ .
\end{equation*}
%
We apply $e_2$ to contribute an abstract successor $m_3$ of the
abstract state $m_2$ to the thread reachability of the first thread:
%
\begin{equation*}
  m_3 = \abstTIdOf{1}(\post(e_2 \land \relLocEqOf{1}, m_2)) = (\pcone=b)\ .
\end{equation*}
%

\iffalse
We denote by $\EnvTransOf{j}$ the set of all environment
transitions that are to be applied in thread $j\ $. 
For example, after the previous step, $\EnvTransOf{1} = \set{e_2}$ and
$\EnvTransOf{2} = \set{e_1}\ $.
The effect of the environment transitions $e_1$ and, respectively,
$e_2$ is to havoc the values of all program variables except $\pctwo$
and respectively $\pcone$ (local variables in the thread that applies
the environment transition).
%
\begin{equation*}
  \begin{array}[t]{r@{\arrSpace}l@{\arrSpace}l}
    m_3 & = \abst_1(\post(e_2 \land \pcone=\pcone', m_2)) & =(\pcone=b)\\[\jot]
    n_3 & = \abst_2(\post(e_1 \land \pctwo=\pctwo', n_2)) & =(\pctwo=q)\\[\jot]
  \end{array}
\end{equation*}
%
\fi

We observe that the intersection of the abstract states $m_3$ and
$n_3$ contains a non-empty set of error states, i.e., $m_3 \land n_3
\land \SymbError$ is satisfiable, thus delivering a possible evidence
for incorrectness.


% Note that the branches that give the tree-shape of the counterexample
% originate either from the error check or from the application of
% environment transitions.
 
\paragraph{Second abstraction refinement} 


Similarly to the first abstraction refinement step, we construct a set
of Horn clauses $\HornClauses_2$ to check if the discovered evidence
is spurious.
We consider predicates $\unkOf{m_1}(\Vars)$, $\unkOf{n_1}(\Vars)$,
$\unkOf{m_2}(\Vars)$, $\unkOf{n_2}(\Vars)$, $\unkOf{m_3}(\Vars)$, and
$\unkOf{n_3}(\Vars)$ that represent unknown sets of program states,
together with $\unkOf{e_1}(\Vars,\Vars')$ and
$\unkOf{e_2}(\Vars,\Vars')$ that represent unknown binary relations
over program states.
%
\begin{equation*}
  \HornClauses_2 = \set{
    \begin{array}[t]{@{}l}
      \SymbInit \limplies \unkOf{m_1}(\Vars),\ 
      \unkOf{m_1}(\Vars) \land \rel_1 \limplies \unkOf{m_2}(\Vars'), \\[\jot]
      \SymbInit \limplies \unkOf{n_1}(\Vars),\ 
      \unkOf{n_1}(\Vars) \land \rel_2 \limplies \unkOf{n_2}(\Vars'), \\[\jot]
      \unkOf{m_1}(\Vars) \land \rel_1 \limplies \unkOf{e_1}(\Vars,\Vars'),
      \\[\jot]
      \unkOf{n_2}(\Vars) \land \unkOf{e_1}(\Vars,\Vars') \land \relLocEqOf{2}
      \limplies \unkOf{n_3}(\Vars'), \\[\jot]
      \unkOf{n_1}(\Vars) \land \rel_2 \limplies \unkOf{e_2}(\Vars,\Vars'),
      \\[\jot]
      \unkOf{m_2}(\Vars) \land \unkOf{e_2}(\Vars,\Vars') \land \relLocEqOf{1}
      \limplies \unkOf{m_3}(\Vars'),\\[\jot]
      \unkOf{m_3}(\Vars) \land \unkOf{n_3}(\Vars) \land \SymbError \limplies 
      \lfalse }
  \end{array}
\end{equation*}
%
The conjunction of clauses in $\HornClauses_2$ is satisfiable.
We obtain the following satisfying assignment~$\Solution$.
%
\begin{equation*}
  \begin{array}[t]{l@{\;=\;}ll@{\;=\;}l}
    \SolutionOf{\unkOf{m_1}(\Vars)}&\ltrue & 
    \SolutionOf{\unkOf{m_2}(\Vars)}&(\lock=1) \\[\jot]
    \SolutionOf{\unkOf{n_1}(\Vars)}&\ltrue &
    \SolutionOf{\unkOf{n_2}(\Vars)}&(\lock=1)\\[\jot]
    \SolutionOf{\unkOf{m_3}(\Vars)}&\lfalse & 
    \SolutionOf{\unkOf{e_1}(\Vars,\Vars')}&(\lock=0) \\[\jot]
    \SolutionOf{\unkOf{n_3}(\Vars)}&\lfalse &
    \SolutionOf{\unkOf{e_2}(\Vars,\Vars')}&(\lock=0)
  \end{array}
\end{equation*}
%
This solution constrains $\unkOf{m_2}(\Vars)$ and $\unkOf{n_2}(\Vars)$
to states where the lock is held $(\lock=1)\ $, while the environment
transitions $\unkOf{e_1}(\Vars,\Vars')$ and
$\unkOf{e_2}(\Vars,\Vars')$ are applicable only in states for which
the lock is not held by the respective thread $(\lock=0)\ $.

We add the (transition) predicates that appear in the environment
transition $e_2$ of the first thread to the set $\artTPredsOf{2}{1}\ $,
and, symmetrically, we add the predicates from $e_1$ to
$\artTPredsOf{1}{2}$.
For the next \aret computation we have the following set of
predicates:
%
\begin{equation*}
  \begin{array}[t]{l@{\;=\;}l}
    \artPreds_1 & \set{\pcone=a, \pcone=b, \pctwo=p, \lock=1}\ ,\\[\jot]
    \artPreds_2 & \set{\pctwo=p, \pctwo=q, \pcone=a, \lock=1}\ ,\\[\jot]
    \artTPredsOf{1}{2} & \set{\lock=0}\ ,\\[\jot]
    \artTPredsOf{2}{1} & \set{\lock=0}\ .
  \end{array}
\end{equation*}
%


% ===================== Iteration 4 =======================
\paragraph{Last \aret computation}

We perform another \aret computation and a subsequent abstraction
refinement step.
We add the predicate $\lock'=1$ to both $\artTPredsOf{1}{2}$
and~$\artTPredsOf{2}{1}$, and proceed with the final \aret computation.
%
Figure~\ref{fig-ex-takebit-art}(c) shows the resulting trees.
The application of thread transitions produces the following abstract
states and environment transitions:
%
\begin{equation*}
  \begin{array}[t]{l@{\arrSpace}l@{\arrSpace}l@{}}
    m_1 & = \abst_1(\SymbInit) & = (\pcone = a \land \pctwo = p)\ ,  \\[\jot]
    n_1 & = \abst_2(\SymbInit) & = (\pcone = a \land \pctwo = p)\ , \\[\jot] 
    m_2 & = \abst_1(\post(\rel_1, m_1)) & = (\pcone = b \land \pctwo = p 
    \land \lock = 1)\ , \\[\jot] 
    n_2 & = \abst_2(\post(\rel_2, n_1)) & = (\pcone = a \land \pctwo = q 
    \land \lock = 1)\ , \\[\jot] 
    e_1 & = \abstTrans_{\envFromTo{1}{2}}(m_1 \land \rel_1) & = (\lock = 0 
    \land \lock'=1)\ , \\[\jot]
    e_2 & = \abstTrans_{\envFromTo{2}{1}}(n_1 \land \rel_2) & = (\lock = 0 
    \land \lock'=1)\ .
  \end{array}
\end{equation*}
%
The environment transitions $e_1$ and $e_2$ produce the abstract
states $m_3$ and $n_3$ whose intersection does not contain any error
states.
%
\begin{equation*}
  \begin{array}[t]{l@{\arrSpace}l@{\arrSpace}l}
    m_3 &= \abst_1(\post(e_2 \land \relLocEqOf{1}, m_2)) &= (\pcone = a 
    \land \lock = 1) \\[\jot] 
    n_3 &= \abst_2(\post(e_1 \land \relLocEqOf{2}, n_2)) &= (\pctwo = p 
    \land \lock = 1)
  \end{array}
\end{equation*}

Neither thread nor environment transitions can be applied from the
abstract states $m_3$ and $n_3$, while no further abstract states are
found.
Since each pair of abstract states from different threads yields an
intersection that is disjoint from the error states, we conclude that
\TakeLockBit is safe.
The labeling of the computed trees can be directly used to construct a
safety proof for \TakeLockBit, as Sections~\ref{sec-proof-rule}
and~\ref{sec-algo-safety} will show.

\iffalse
We denote by $\artReachOf{1}$ and $\artReachOf{2}$ the set of all
abstract reachable states found in this last \aret computation.
\fi

\subsection{Example \TakeLockThreadId} \label{subsec-takethreadid}

\begin{figure}[t]
\small
\hrule
\begin{minipage}[t]{.48\columnwidth} 
\begin{verbatim} 

    // Thread 1 
    a: take_lock(lock, 1); 
    b: // critical 
\end{verbatim}
\end{minipage}
\hfill
\begin{minipage}[t]{.48\columnwidth} 
\begin{verbatim}

    // Thread 2
    p: take_lock(lock, 2);
    q: // critical

\end{verbatim}
\end{minipage}
  \caption{Example program $\TakeLockThreadId\ $.}
  \label{fig-take-lock-2}
\end{figure}


Our second example \TakeLockThreadId, shown in
Figure~\ref{fig-take-lock-2}, is a variation of \TakeLockBit.
\TakeLockThreadId uses an integer variable $\lock$ (instead of a
single bit) to record which thread holds the lock.
Due to this additional information recorded in the global variable,
the example \TakeLockThreadId has a modular proof, which does not
refer to any local variables.
We show how our algorithm discovers such a proof by only admitting
modular predicates in the abstraction refinement step.

\TakeLockThreadId differs from \TakeLockBit in its transition relation
$\rho_2$:
%
% ===================== Program description ===============
\begin{equation*}
  \rel_2 = (\lock=0 \land \lock'=2 \land \pctwo=p \land \pctwo'=q
  \land \relLocEqOf{1})\ .
\end{equation*}

% ===================== Iteration 1 =======================
\paragraph{First \aret computation} 

Similarly to \TakeLockBit, we discover that $m_2 \land n_2 \land
\SymbError$ is satisfiable, and compute the following set of Horn
clauses:
%
\begin{equation*}
  \HornClauses_3 = \set{
  \begin{array}[t]{@{}l}
     \SymbInit \limplies \unkOf{m_1}(\Vars),\
     \unkOf{m_1}(\Vars) \land \rel_1 \limplies \unkOf{m_2}(\Vars'),\\[\jot]
     \SymbInit \limplies \unkOf{n_1}(\Vars),\ 
     \unkOf{n_1}(\Vars) \land \rel_2 \limplies \unkOf{n_2}(\Vars'),\\[\jot]
     \unkOf{m_2}(\Vars) \land \unkOf{n_2}(\Vars) \land \SymbError \limplies 
     \lfalse }\ .
  \end{array}
\end{equation*}
%
One possible satisfying assignment $\Solution$ is:
\begin{equation*}
  \begin{array}[t]{l@{\;=\;}ll@{\;=\;}l@{}}
    \SolutionOf{\unkOf{m_1}(\Vars)} & (\pctwo=p)\ , & 
    \SolutionOf{\unkOf{n_1}(\Vars)} & (\pcone=a)\ , \\[\jot]
    \SolutionOf{\unkOf{m_2}(\Vars)} & (\pctwo=p)\ , & 
    \SolutionOf{\unkOf{n_2}(\Vars)} & (\pcone=a)\ .
  \end{array}
\end{equation*} 
%
This assignment uses a predicate $\pctwo=p$ over the local variable of
the second thread as a solution for the abstract state
$\unkOf{m_1}(\Vars)$ in the thread reachability of the first thread.
By collecting and using the corresponding predicates, we will discover
a non-modular proof.

To avoid the drawbacks of non-modular proofs, our algorithm does not
use $\HornClauses_3$ and attempts to find modular predicates for
abstraction refinement instead.
We express the preference for modular predicates declaratively, using
a set of Horn clauses in which the unknown predicates are restricted
to the desired variables, as described in
Sections~\ref{sec-refinement}.
For the abstract states in the first thread, we require that the
corresponding solutions are over the global variable $\lock$ and the
local variable $\pcone$ of the first thread, i.e., we have the unknown
predicates $\unkOf{m_1}(\lock, \pcone)$ and~$\unkOf{m_2} (\lock,
\pcone)$.
Similarly, for the second thread we obtain $\unkOf{n_1}(\lock,
\pctwo)$ and~$\unkOf{n_2}(\lock, \pctwo)$.
Instead of~$\HornClauses_3$,  we use a set of Horn clauses $\HornClauses_4$ shown below:
%
\begin{equation*}
  \HornClauses_4 = \set{
    \begin{array}[t]{@{}l@{}}
       \SymbInit \limplies \unkOf{m_1}(\lock, 
       \pcone),\\[\jot]
       \unkOf{m_1}(\lock, \pcone) \land \rel_1 \limplies \unkOf{m_2}(\lock', \pcone'),\\[\jot]
       \SymbInit \limplies \unkOf{n_1}(\lock, \pctwo),\\[\jot]
       \unkOf{n_1}(\lock, \pctwo) \land \rel_2 \limplies \unkOf{n_2}(\lock', \pctwo'),\\[\jot]
       \unkOf{m_2}(\lock, \pcone) \land \unkOf{n_2}(\lock, \pctwo) \land \SymbError \limplies 
       \lfalse }\ .
  \end{array}
\end{equation*}
%
The conjunction of clauses from $\HornClauses_4$ can be satisfied by
an assignment $\Solution$ such that
%
\begin{equation*}
  \begin{array}[t]{r@{\;=\;}l}
    \SolutionOf{\unkOf{m_1}(\lock, \pcone)} &\ltrue\ , \\[\jot]
    \SolutionOf{\unkOf{m_2}(\lock, \pcone)} &(\lock=1)\ , \\[\jot]
    \SolutionOf{\unkOf{n_1}(\lock, \pctwo)} &\ltrue\ ,\\[\jot]
    \SolutionOf{\unkOf{n_2}(\lock, \pctwo)} &(\lock=2)\ ,
  \end{array}
\end{equation*}
%
which contains only modular predicates.

% ===================== ART of 3rd iteration from TakeLockThreadId =============
\begin{figure}[t]
  \hrule
  \centering
  \vspace{2ex} 
  \begin{tikzpicture}[->, >=stealth', shorten >=1pt, auto, 
      node distance=\nodeDistance, semithick]
    % Reachability from thread 1
    \node[stateBox] (m1) {$m_1$};
    \node[stateBox, below of=m1] (m2) {$m_2$};
    \path (m1) edge node[left]{$\rel_1$} (m2);
    % Reachability from thread 2
    \node[stateBox, right of=m1, node distance=\artDistance{1}] (n1) {$n_1$};
    \node[stateBox, below of=n1] (n2) {$n_2$};
    \path (n1) edge node[right]{$\rel_2$} (n2);
  \end{tikzpicture}
  \caption{Tree that shows all the reachable abstract states found
    during the last \aret computation for~\TakeLockThreadId.}
  \label{fig-ex-take-threadid-art-3}
\end{figure}

%%% Local Variables: 
%%% mode: latex
%%% TeX-master: "main"
%%% End: 
 

\paragraph{Last \aret computation} 

We present the last \aret computation for \TakeLockBit, which uses on
the following (transition) predicates collected so far:
%
\begin{equation*}
  \begin{array}[t]{l@{\arrSpace}l}
    \artPreds_1 &= \set{\pcone=a, \pcone=b, \lock=1}\ ,\\[\jot]
    \artPreds_2 &= \set{\pctwo=p, \pctwo=q, \lock=2}\ ,\\[\jot]
    \artTPreds_{\envFromTo{2}{1}} &= \set{\lock=0}\ ,\\[\jot]
    \artTPreds_{\envFromTo{1}{2}} &= \set{\lock=0}\ .\\[\jot]
  \end{array}
\end{equation*}
%
Figure~\ref{fig-ex-take-threadid-art-3} shows the resulting abstract
reachability and environment trees constructed as follows:
%
\begin{equation*}
  \begin{array}[t]{l@{\;=\;}l@{\;=\;}l}
    m_1 & \abst_1(\SymbInit) &  (\pcone=a) \\[\jot] 
    n_1 & \abst_2(\SymbInit) &  (\pctwo=p) \\[\jot]
    m_2 & \abst_1(\post(\rel_1, m_1)) & (\pcone=b \land \lock = 1)\\[\jot]
    n_2 & \abst_2(\post(\rel_2, n_1)) & (\pctwo=q \land \lock = 2)\\[\jot]
    e_1 & \abstTransTIdOf{1}{2}(m_1 \land \rel_1) & (\lock = 0) \\[\jot]
    e_2 & \abstTransTIdOf{2}{1}(n_1 \land \rel_2) & (\lock = 0)  \\[\jot]
  \end{array}
\end{equation*}
%
The \aret construction is completed, since
\begin{equation*}
\begin{array}[t]{l@{\;\limplies\;}ll@{\;=\;}l@{}}
    \abst_1(\post(e_2 \land \relLocEqOf{1}, m_1)) & m_1\ ,
    & \abst_1(\post(e_2, m_2)) & \lfalse\ ,\\[\jot] 
    \abst_2(\post(e_1 \land \relLocEqOf{2}, n_1)) & n_1\ ,
    &
    \abst_2(\post(e_1, n_2)) & \lfalse\ .
  \end{array}
\end{equation*}
%
By inspecting pairs of abstract states from different trees we
conclude that \TakeLockThreadId is safe.

Furthermore no predicate in $\artPreds_1$ refers to the local variable
of the second thread, the symmetric condition holds for
$\artPreds_2$, and the predicates in $\artTPredsOf{1}{2}$ as well as
$\artTPredsOf{2}{1}$ do not refer to any local variables.
Thus, from the trees in Figure~\ref{fig-ex-take-threadid-art-3} we can
construct a modular safety proof.



%%% Local Variables: 
%%% mode: latex
%%% TeX-master: "main"
%%% End: 
