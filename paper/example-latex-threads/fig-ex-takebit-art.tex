\begin{figure}[t]
  \hrule
  \centering
  \vspace{2ex} 
% ===================== ART computed in 1st iteration
(a)
  \begin{tikzpicture}[->, >=stealth', shorten >=1pt, auto, 
      node distance=\nodeDistance, semithick]
%    \tikzstyle{every stateBox}=[draw=black]
    % 1st thread
    \node[stateBox] (m1) {$m_1$};
    \node[stateBox, errorState, below of=m1] (m2) {$m_2$};
    \path (m1) edge node[left]{$\rel_1$} (m2);
    % 2nd thread
    \node[stateBox,right of=m1, node distance=\artDistance{1}] (n1) {$n_1$};
    \node[stateBox, errorState, below of=n1] (n2) {$n_2$};
    \path (n1) edge node[right]{$\rel_2$} (n2);
  \end{tikzpicture}
  \hspace{2cm}
% ===================== ART computed in 2nd iteration
(b)
  \begin{tikzpicture}[->, >=stealth', shorten >=1pt, auto, 
      node distance=\nodeDistance, semithick]
%    \tikzstyle{every state}=[draw=black]
    % 1st thread
    \node[stateBox] (m1) {$m_1$};
    \node[stateBox, below of=m1] (m2) {$m_2$};
    \path (m1) edge node[left]{$\rel_1$} (m2);
    \node[stateBox, errorState, below of=m2] (m3) {$m_3$};
    \path (m2) edge node[left]{$e_2$} (m3);
    % 2nd thread
    \node[stateBox,right of=m1, node distance=\artDistance{1}] (n1) {$n_1$};
    \node[stateBox, below of=n1] (n2) {$n_2$};
    \path (n1) edge node[right]{$\rel_2$} (n2);
    \node[stateBox, errorState, below of=n2] (n3) {$n_3$};
    \path (n2) edge node[right]{$e_1$} (n3);
    % environment arrows
    \path (0+\shiftEnv,-\nodeDistance/2) edge[envTransition] % edge[out=0,in=180]
    (\artDistance{1}-\shiftEnv,-\nodeDistance-\nodeDistance/2);
    \path (\artDistance{1}-\shiftEnv, -\nodeDistance/2) edge[envTransition]% edge[out=180,in=0]
    (\shiftEnv,-\nodeDistance-\nodeDistance/2);
  \end{tikzpicture}
  \vspace{2ex} \hrule \vspace{2ex}
% ===================== ART computed in 4th iteration
\renewcommand{\nodeDistance}{2cm}
(c)\quad
  \begin{tikzpicture}[->, >=stealth', shorten >=1pt, auto, 
      node distance=\nodeDistance, semithick]
%    \tikzstyle{every state}=[draw=black]
    % 1st thread    
    \node[stateBox] (m1) {$m_1$};
    \node[stateBox, below of=m1, xshift=\shiftLeft] (m2) {$m_2$};
    \path (m1) edge node[left, near start]{$\rel_1$} (m2);
    \node[stateBox, below of=m1, xshift=\shiftRight] (m3) {$m_3$};
    \path (m1) edge node[left, near end]{$e_2$} (m3);
    % 2nd thread
    \node[stateBox,right of=m1, node distance=\artDistance{2}] (n1) {$n_1$};
    \node[stateBox, below of=n1, xshift=\shiftRight] (n2) {$n_2$};
    \path (n1) edge node[right, near end]{$\rel_2$} (n2);
    \node[stateBox, below of=n1, xshift=\shiftLeft] (n3) {$n_3$};
    \path (n1) edge node[left, near start]{$e_1$} (n3);
    % environment arrows
    \path (-0.4cm+\shiftEnv,-\nodeDistance/2+0.3cm) 
      edge[envTransition] %edge[out=5,in=165] 
      (\artDistance{2}-0.4cm-\shiftEnv,-\nodeDistance/2+0.3cm);
    \path (\artDistance{2}+0.6cm-\shiftEnv,-\nodeDistance/2-0.3cm) 
      edge[envTransition] %edge[out=185,in=345] 
      (0.4cm+\shiftEnv,-\nodeDistance/2-0.3cm);
  \end{tikzpicture}
  \caption{
    Reachability trees constructed using different abstraction functions.
    Edges are labeled with a transition. 
    Nodes with gray background represent (spurious) error tuples:
    $(m_2, n_2)$ from (a) and $(m_3, n_3)$ from (b).
    No pair of states from (c) intersects $\SymbError$.
  }
  \label{fig-ex-takebit-art}
\end{figure}

% System of coordinates (x,y):
%    (0,0)      (1,0)
%
%    (0,-1)     (1,-1)
%

%%% Local Variables: 
%%% mode: latex
%%% TeX-master: "main"
%%% End: 
