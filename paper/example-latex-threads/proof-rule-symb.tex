\section{Proof rule, fixpoints, and abstraction}\label{sec-proof-rule}
 \begin{figure}[t]
  \hrule
  \mbox{}\\
  For assertions $\Reach_1, \dots, \Reach_\Num$ over $\Vars$\ ,\\
  and $\Env_1, \dots, \Env_\Num$ over $\Vars$ and $\Vars'$%\\[.5\jot]

 \begin{equation*}
   \begin{array}[t]{c@{:\quad}l@{\;\rightarrow\;}l@{\qquad}c}
     \ruleInit & \SymbInit & \Reach_i & \text{for } i\in\ThreadIds\\[\jot]
     \ruleTransStep & \Reach_i \land \relTIdOf{i}
     &  \Reach_i' &  \text{for } i\in\ThreadIds\\[\jot]
     \ruleEnvStep & \Reach_i \land \Env_i \land \relLocEqOf{i} & \Reach_i' & \text{for } i\in\ThreadIds\\[\jot]
     \ruleTransEnv & (\Lor_{i\in\ThreadIds\setminus\set{j}} \Reach_i  \land \relTIdOf{i}) & \Env_j & \text{for } j\in\ThreadIds\\[\jot]
     \ruleCheck & \Reach_1  \land \dots \land \Reach_\Num \land
     \SymbError & \lfalse & 
   \end{array}
 \end{equation*}
 \centering
 \rule{.85\linewidth}{.5pt}\\
 program $\program$ is safe
 \caption{Proof rule \ProofRule for compositional safety proofs
   of multi-threaded programs.
   $\Reach_i'$ stands for~$\Reach_i\substXtoY{\Vars'}{\Vars}\ $.
   \ProofRule yields a pre-fixpoint characterization through
   Equations~\eqref{eq-pre-func}.
 }
  \label{fig-eq-abst-mod-reach}
\end{figure}
%%% Local Variables: 
%%% mode: latex
%%% TeX-master: "main"
%%% End: 


In this section we develop the foundations for our verification
algorithm.
We present a compositional proof rule and then derive a corresponding
characterization in terms of least fixpoints and their approximations.
We present the ability of our proof rule to facilitate modular
reasoning, when admitted by the program, without losing the ability
for global reasoning otherwise.

\paragraph{Proof rule}

Figure~\ref{fig-eq-abst-mod-reach} presents a proof rule \ProofRule
for compositional verification of program safety.
The proof rule is inspired by the existing proof rules for
compositional safety reasoning, see
e.g.~\cite{OwickiAI76,JonesTOPLAS83,CohenFMSD09,HenzingerPLDI04}.
Our formulation of \ProofRule directly leads to a pre-fixpoint
characterization, thus, providing a basis for the proof rule
automation using abstraction and refinement techniques.

\ProofRule relies on \emph{thread reachability} assertions $\Reach_1,
\dots, \Reach_\Num$ that keep track of program states reached
by threads $1, \dots, \Num\ $ together with their respective
\emph{environment transitions} $\Env_1, \dots, \Env_\Num\ $.
The environment transition of each thread keeps track of modifications
of program states by other threads.
The auxiliary assertions used in our proof rule can refer to all
program variables, that is, they are not restricted to a combination
of global variables and local variables of a particular thread. 


If the provided auxiliary assertions satisfy all premises of the proof
rule, i.e., \ruleInit, \dots, \ruleCheck, then the program is safe.
The premise \ruleInit requires that each thread reachability
over-approximates the initial program states.
\ruleTransStep ensures that the thread reachability of each thread is
invariant under the application of the thread transitions.
In addition, \ruleEnvStep requires invariance under the environment
transitions of the thread. 
The conjunct $\relLocEqOf{i}$ in \ruleEnvStep selects the subset of
the environment transition that does not modify the local variables of
the thread.
Given a thread $j\ $, the premise \ruleTransEnv collects transitions
that start from states in the thread reachability of other threads and
combines them into the environment transition for~$j\ $.
Finally, \ruleCheck checks that there is no error state that appears
in each thread reachability set.

The proof rule \ProofRule can be directly used to prove program safety
following a two-step procedure. 
First, we need to identify candidate assertions for the thread
reachability and environment transitions.
Second, we need to check that these candidate assertions satisfy the
premises of proof rule.
The correctness of the conclusion is formalized by the following
theorems.

%
\begin{theorem}[Soundness]
  \label{thm-rule-sound}
  % 
The proof rule \ProofRule is sound.\,\eofClaim
%
\end{theorem}
\includeProof{
\begin{proof}
%  See Appendix~\ref{app-proofs}.
% 
Let $\Reach_1, \dots, \Reach_\Num$ and $\Env_1, \dots, \Env_\Num$
satisfy the premises \ruleInit, \dots, \ruleCheck.
We show that the program is safe.
To prove safety, for each reachable state $s\models\SymbReach$ we
prove that $s\models\Reach_1\land\dots\land\Reach_\Num$ by induction
over the length $k$ of a shortest computation segment $s_1, \dots,
s_k$ such that $s_1\models \SymbInit$ and~$s_k=s\ $.


For the base case $k=1$, the inclusion holds due to the premise
\ruleInit.
For the induction step, we assume that the above statement holds for
states reachable in $k \geq 1$ steps and prove the statement for their
immediate successors.
That is, let $s_k\models\SymbReach$ and hence $s_k \models
\Reach_1\land\dots\land\Reach_\Num\ $.
If $s_k$ does not have any successor, i.e., $\neg (\exists s_{k+1} :
(s_k, s_{k+1}) \models \relProg)\ $, then there are no more states to
consider.
Otherwise, we choose a successor state $s_{k+1}$ of $s_k$ that is
reached by taking a transition in a thread~$i\ $, i.e., $(s_k,
s_{k+1})\models\relTIdOf{i}\ $.
From \ruleTransStep follows that~$s_{k+1}\models\Reach_i\ $.

To show that $s_{k+1}\models\Reach_j$ for each
$j\in\ThreadIds\setminus\set{i}$ we rely on the premises \ruleTransEnv
and \ruleEnvStep.
By induction hypothesis $s_k\models\Reach_i$ and due to \ruleTransEnv
we have $(s_k, s_{k+1})\models\Env_j\ $.
Now $s_{k+1}\in\Reach_j$ follows from \ruleEnvStep.
\end{proof}
}
%
\begin{theorem}[Relative completeness]\label{thm-rule-complete}
%
The proof rule \ProofRule is complete relative to first-order
reasoning.\eofClaim
%
\end{theorem}
%
\includeProof{
\begin{proof} 
Let $\program$ be safe.
We define $\Reach_1 = \dots = \Reach_\Num = \SymbReach$ and 
$\Env_1 = \dots = \Env_\Num = \SymbReach \land \relProg\ $.
Then, the premises \ruleInit, \dots, \ruleCheck are immediately
satisfied.
\end{proof}
}

\paragraph{Modular and  global proofs}

Reasoning about multi-threaded program is more complex than reasoning
about sequential programs since thread interaction needs to be taken
into account.
Some programs admit modular reasoning that deals with each thread in
isolation, i.e., assertions used in the proof only refer to the global
variables and the local variables of one thread at a time.

The proof rule \ProofRule facilitates \emph{modular} reasoning about
multi-threaded programs. 
If a program has a modular safety proof, then the following
\emph{modular} assertions satisfy the proof rule premises:
%
\begin{equation*}
  \begin{array}[t]{l@{\defeq}l@{\qquad}c@{}}
    \Reach_i & \exists \Vars\setminus (\GlobVars \cup \LocVarsOf{i}) :
    \SymbReach\ , & \text{for }
    i \in \ThreadIds\\[\jot]
    \Env_i & \exists (\Vars\cup \Vars')\setminus (\GlobVars \cup
    \GlobVars') : \SymbReach\land \relProg\ . & \text{for }
    i  \in \ThreadIds
  \end{array}
\end{equation*}
% 

\ProofRule is \emph{not} restricted to modular proofs. 
Since the assertions used in \ProofRule can refer to each of the program
variables, non-modular proofs can be directly used.
In fact, the proof of Theorem~\ref{thm-rule-complete} relies on
non-modular assertions, since $\SymbReach$ may refer to local
variables of different threads.

In Section~\ref{sec-algo-safety} we will present our algorithm that
can discover modular assertions for \ProofRule if the program admits
modular proofs, and delivers non-modular assertions otherwise.




\paragraph{Fixpoints}

The proof rule \ProofRule in Figure~\ref{fig-eq-abst-mod-reach}
directly leads to a fixpoint-based characterization, which defines our
algorithm in Section~\ref{sec-algo-safety}.

From the premises \ruleTransStep, \ruleEnvStep, and \ruleTransEnv we
obtain a function $\eqPreFunc$ on $\Num$-tuples of assertions over the
program variables and $\Num$-tuples of assertions over the unprimed
and primed program variables such that
%
\begin{equation}\label{eq-pre-func}
  \begin{split}
    & \eqPreFunc(S_1, \dots, S_\Num, T_1, \dots, T_\Num) \defeq \\[\jot]
    &\;\,\,  (
    \begin{array}[t]{@{}l@{\; \dots,\; }l}
      \post(\relTIdOf{1} \lor T_1\land \rho^=_1, S_1), 
      &
      \post(\relTIdOf{\Num} \lor T_\Num\land \rho^=_\Num, S_\Num),\\[2\jot]
      \Lor_{i\in\ThreadIds\setminus\set{1}}S_i \land \relTIdOf{i},
      &
      \Lor_{i\in\ThreadIds\setminus\set{\Num}}S_i \land \relTIdOf{i} )\ .
    \end{array}
\end{split}
\end{equation} 
%
We formalize the relation between $\eqPreFunc$ and \ProofRule as
follows.
%
\begin{lemma}\label{lem-rule-fix}
  %
      Each pre-fixpoint of $\eqPreFunc$ satisfies the premises
\emph{\ruleTransStep,} \emph{\ruleEnvStep,} and \emph{\ruleTransEnv}
of \ProofRule.
That is, if
%
\begin{equation*}
  \eqPreFunc(\Reach_1, \dots, \Reach_\Num, \Env_1, \dots, \Env_\Num) \limplies
  (\Reach_1, \dots, \Reach_\Num, \Env_1, \dots, \Env_\Num)
\end{equation*}
%
then $\Reach_1, \dots, \Reach_\Num, \Env_1, \dots, \Env_\Num$
satisfies \emph{\ruleTransStep,} \emph{\ruleEnvStep,} and
\emph{\ruleTransEnv}.\eofClaim
%
\end{lemma}
%
We define a distinguished tuple $\eqPreBot$:
%
\begin{equation}\label{eq-pre-bot}
  \eqPreBot \defeq (\SymbInit,\dots,\SymbInit, \underbrace{\lfalse, \dots,
    \lfalse}_{\Num \text{ times}})\ .
\end{equation}
%
Then, each pre-fixpoint of $\eqPreFunc$ that is greater than
$\eqPreBot$ satisfies the premise \ruleInit.
By choosing a pre-fixpoint $(\Reach_1, \dots, \Reach_\Num, \Env_1,
\dots, \Env_\Num)$ above $\eqPreBot$ such that $\Reach_1 \land \dots,
\Reach_\Num \land \SymbError \limplies \lfalse$ we will satisfy all
premises of the proof rule \ProofRule, and hence prove the program
safety.


\paragraph{Fixpoint abstraction}

Computing pre-fixpoints of $\eqPreFunc$ that satisfy \ruleInit and
\ruleCheck is a difficult task.
We automate this computation using the framework of abstract
interpretation~\cite{lattice77}, which uses over-approximation to
strike a balance between reasoning precision and efficiency.
To implement required over-approximation functions, we will use a
collection of abstraction functions $\abstTIdOf{i}$ and
$\abstTransTIdOf{i}{j}$, where $i \neq j \in\ThreadIds$, that
over-approximate sets and binary relations over programs states,
respectively.

We define a function $\eqFunc$ that over-approximates $\eqPreFunc$
using given abstraction functions:
%
\begin{equation}\label{eq-func}
  \begin{split}
    & \eqFunc(S_1, \dots, S_\Num, T_1, \dots, T_\Num) \defeq \\[\jot]
    &\qquad  (
    \begin{array}[t]{@{}l}
      \abst_1(\post(\relTIdOf{1}, S_1)) \lor 
      \abst_1(\post(T_1 \land \relLocEqOf{1}, S_1))\ , \\[\jot]
      \qquad\qquad\cdots\\
      \abst_\Num(\post(\relTIdOf{\Num}, S_\Num) \lor 
      \abst_\Num(\post(T_\Num \land \relLocEqOf{\Num}, S_\Num))\ ,
    \\[3\jot]
    \Lor_{i\in\ThreadIds\setminus\set{1}}\abstTransTIdOf{i}{1}(S_i \land \relTIdOf{i})\ ,\\[2\jot]
    \qquad\qquad\cdots \\[\jot]
    \Lor_{i\in\ThreadIds\setminus\set{\Num}}\abstTransTIdOf{i}{\Num}(S_i \land \relTIdOf{i} ))\ .
  \end{array}
\end{split}
\end{equation} 
%
Let $\eqBot$ be an over-approximation of $\eqPreFunc$ such that
%
\begin{equation}\label{eq-bot}
  \eqBot \defeq (\abst_1(\SymbInit),\dots,\abst_\Num(\SymbInit), \underbrace{\lfalse, \dots,
    \lfalse}_{\Num \text{ times}})\ .
\end{equation}
%
The least pre-fixpoint of $\eqFunc$ above $\eqBot$ can be used to
prove program safety by applying the following theorem, and is the key
outcome of our algorithm in Section~\ref{sec-algo-safety}.
%
\begin{theorem}[Abstract fixpoint checking]\label{thm-abst-fix}
  % 
    If the least pre-fixpoint of $\eqFunc$ above $\eqBot$, say
$(\Reach_1, \dots, \Reach_\Num, \Env_1, \dots, \Env_\Num)$, satisfies
the premise \emph{\ruleCheck} then the program is safe.\eofClaim
  %
\end{theorem}
%
\includeProof{
\begin{proof} The theorem follows directly from the soundness of the
proof rule \ProofRule, Lemma~\ref{lem-rule-fix}, and
over-approximations introduced by the applied abstraction functions.
\end{proof}
}

The choice of the abstract domains, i.e., the range sets of the
abstraction functions, determines if the least fixpoint of $\eqFunc$
yields a modular proof.
Our abstraction discovery algorithm in Section~\ref{sec-refinement}
automatically chooses the abstraction such that modular proofs are
preferred.


%%% Local Variables: 
%%% mode: latex
%%% TeX-master: "main"
%%% End: 
