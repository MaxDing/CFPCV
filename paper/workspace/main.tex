\documentclass{llncs}

\newcommand{\isDraft}{false}

\sloppy

\pagestyle{plain}

\newcommand{\isTechReport}{true}

\newcommand{\seeTechReport}{See~\cite{FunVTechReport}.}

\usepackage{float} % provides the [H] option for figures
\usepackage{amsmath,color}
\usepackage{amssymb}
\usepackage{xspace}
\usepackage{ar-alg}
\usepackage[inference]{semantic}
% \usepackage[usenames,dvipsnames]{xcolor}
\usepackage{mathpartir}
\usepackage{graphics}
% \usepackage[amsmath,thmmarks]{ntheorem}
\usepackage{definitions}
\usepackage{rotating}
\usepackage[nocompress]{cite}

\newcommand{\myexp}[1]{}

\newcommand{\setOf}[1]{\{ #1 \}}
\newcommand{\ltrue}{\mathit{true}}
\newcommand{\lfalse}{\mathit{false}}

\newcommand{\theFunction}{\ensuremath{f}}
\newcommand{\theTI}{\ensuremath{\mathit{TI}}}
\newcommand{\theArity}{\ensuremath{N}}

\newcommand{\todo}[1]{\textcolor{red}{\bf #1}}
% tuning

\newcommand{\ttt}[1]{\texttt{#1}}
\newcommand{\tsf}[1]{\textsf{#1}}
\newcommand{\trm}[1]{\textrm{#1}}
\newcommand{\tbf}[1]{\textbf{#1}}

% \newcommand{\gray}[1]{\textcolor{Gray}{#1}} % not used

% misc

\newcommand{\versimple}{}
% \newcommand{\versimple}{simple version of\ }

\newcommand{\dsolve}{\tsf{Dsolve}}
\newcommand{\ocaml}{OCaml}
\newcommand{\miniocaml}{\mbox{Mini-OCaml}\xspace}
\newcommand{\camlp}{\tsf{Camlp4}}
\newcommand{\haskell}{Haskell}

\newcommand{\nats}{\ensuremath{\mathbb{N}}}
\newcommand{\ints}{\ensuremath{\mathbb{Z}}}

% illustration section

\newcommand{\simple}{\ttt{iter\_down}}
\newcommand{\bodysimple}{\expr[\ttt{id}]}
\newcommand{\sttranssimple}{\ttt{mtrans\_id}}
\newcommand{\simplem}{\ttt{simple\_m}}

\newcommand{\foldleft}{\ttt{fold\_left}}
\newcommand{\bodyfl}{\expr[\ttt{fold\_left}]}
\newcommand{\sttransfl}{\ttt{mtrans\_fl}}
\newcommand{\foldleftm}{\ttt{fold\_left\_m}}

\newcommand{\ack}{\ttt{ack}}
\newcommand{\bodyack}{\expr[\ttt{ack}]}
\newcommand{\sttransack}{\ttt{mtrans\_ack}}
\newcommand{\ackm}{\ttt{ack\_m}}

\newcommand{\mccarthy}{\ttt{mccarthy}}
\newcommand{\bodymc}{\expr[\ttt{mccarthy91}]}
\newcommand{\sttransmc}{\ttt{mtrans\_mc}}
\newcommand{\mccarthym}{\ttt{mccarthy\_m}}

% syntax

\newcommand{\vars}{\ensuremath{X}}
\newcommand{\csts}{\ensuremath{C}}
\newcommand{\cstrs}{\ensuremath{\vec{C}}}

\newcommand{\freevars}{\tsf{FreeVars}}

\newcommand{\patts}{\ensuremath{P}}
\newcommand{\exprs}{\ensuremath{E}}

\newcommand{\patt}[1][]{\ensuremath{p_{#1}}}
\newcommand{\expr}[1][]{\ensuremath{e_{#1}}}
\newcommand{\cst}{\ensuremath{c}}
\newcommand{\vst}[1][]{\ensuremath{s_{#1}}}
\newcommand{\var}[1][]{\ensuremath{x_{#1}}}
\newcommand{\cstr}[1][]{\ensuremath{\vec{c}_{#1}}}

% semantics

\newcommand{\vals}{\ensuremath{V}}
\newcommand{\evjs}{\ensuremath{J}}

\newcommand{\val}[1][]{\ensuremath{v_{#1}}}
\newcommand{\evjvar}[1][]{\ensuremath{j_{#1}}}
\newcommand{\evt}[1][]{\ensuremath{\mathfrak{t}_{#1}}}
\newcommand{\evtP}[1][]{\ensuremath{\mathfrak{t}'_{#1}}}

\newcommand{\patternmatch}{\textsf{Bindings}}
\newcommand{\dom}{\ensuremath{\mathit{Dom}}}

\newcommand{\ectx}[1][]{\ensuremath{\mathcal{E}_{#1}}}
\newcommand{\emptyectx}{\ensuremath{\emptyset}}
\newcommand{\eext}{\ensuremath{+}}
\newcommand{\emapsto}{\ensuremath{\mapsto}}

\newcommand{\evj}[3]{\ensuremath{#1~\vdash~#2~\Rightarrow~#3}}

\newcommand{\scst}{(Cst)}
\newcommand{\svar}{(Var)}
\newcommand{\svarerr}{(Var Error)}
\newcommand{\stup}{(Tuple)}
\newcommand{\stuperr}{(Tuple Error)}
\newcommand{\stupexn}{(Tuple Exn)}
\newcommand{\scstr}{(Constr)}
\newcommand{\scstrerr}{(Constr Error)}
\newcommand{\sapp}{(App)}
\newcommand{\sapprec}{(App Rec)}
\newcommand{\sappcst}{(App Cst)}
\newcommand{\sapperrfun}{(App Error)}
\newcommand{\sapperrparam}{(App Param Error)}
\newcommand{\sfun}{(Fun)}
\newcommand{\sletrec}{(Let Rec)}
\newcommand{\slet}{(Let)}
\newcommand{\sleterrmatch}{(Let Error Match)}
\newcommand{\sleterrbound}{(Let Error Bound)}
\newcommand{\sasserttrue}{(Assert True)}
\newcommand{\sassertfalse}{(Assert False)}
\newcommand{\sasserterr}{(Assert Error)}
\newcommand{\siftrue}{(If True)}
\newcommand{\siffalse}{(If False)}
\newcommand{\siferr}{(If Cond Error)}
\newcommand{\smatch}{(Match)}
\newcommand{\smatcherr}{(Match Error)}

%%%\newcommand{\ercst}{
  \inference[{\tiny \scst}]{
  }{
    \evj{\ectx}{\cst}{\cst}
  }
}

\newcommand{\ervar}{
  \inference[{\tiny \svar}]{
    \var \in \dom\ \ectx
  }{
    \evj{\ectx}{\var}{\ectx\ \var}
  }
}

\newcommand{\ertup}{
  \inference[{\tiny \stup}]{
    \evj{\ectx}{\expr[n]}{\val[n]}
    &
    \ldots
    &
    \evj{\ectx}{\expr[1]}{\val[1]}
  }{
    \evj{\ectx}{\ntupleexpr}{\ntupleval}
  }
}

\newcommand{\ercstr}{
  \inference[{\tiny \scstr}]{
    \evj{\ectx}{\expr[n]}{\val[n]}
    &
    \ldots
    &
    \evj{\ectx}{\expr[1]}{\val[1]}
  }{
    \evj{\ectx}{\ttt{\cstr(\ntupleexpr)}}{\cstr(\ntupleval)}
  }
}

\newcommand{\erappcst}{
  \inference[{\tiny \sappcst}]{
    \evj{\ectx}{\expr[p]}{\val[p]}
    &
    \evj{\ectx}{\expr[f]}{\cst}
  }{
    \evj{\ectx}{\appexpr}{\cst\ \val[p]}
  }
}

\newcommand{\erapp}{
  \inference[{\tiny \sapp}]{
    \evj{\ectx}{\expr[p]}{\val[p]}
    &
    \evj{\ectx}{\expr[f]}{\closure[b]}
    \\
    \evj{\ectx[b] \eext \var \emapsto \val[p]}{\expr[b]}{\val}
  }{
    \evj{\ectx}{\appexpr}{\val}
  }
}

\newcommand{\erapprec}{
  \inference[{\tiny \sapprec}]{
    \evj{\ectx}{\expr[p]}{\val[p]}
    &
    \evj{\ectx}{\expr[f]}{\recclosure[b]}
    \\
    \evj{\ectx[b] \eext \var[f] \emapsto \recclosure[b] \eext \var
      \emapsto \val[p]}{\expr[b]}{\val}
  }{
    \evj{\ectx}{\appexpr}{\val}
  }
}

\newcommand{\erlet}{
  \inference[{\tiny \slet}]{
    \evj{\ectx}{\expr[1]}{\val[1]}
    &
    \evj
    { \ectx \eext \patternmatch~\patt~\val[1] }
    {\expr[2]}
    {\val}
  }{
    \evj
    {\ectx}
    {\ttt{let \patt\ = \expr[1]\ in \expr[2]}}
    {\val}
  }
}

\newcommand{\erfun}{
  \inference[{\tiny \sfun}]{
  }{
    \evj
    {\ectx}
    {\funexpr}
    {\closure}
  }
}

\newcommand{\erletrec}{
  \inference[{\tiny \sletrec}]{
    \evj
    { \ectx \eext \var[f] \emapsto \recclosure}
    {\expr[2]}
    {\val}
  }{
    \evj
    {\ectx}
    {\ensuremath{
        \ttt{let rec \var[f] = \funexpr\ in \expr[2]}
      }}
    {\val}
  }
}

\newcommand{\ermatch}{
  \inference[{\tiny \smatch}]{
    \evj{\ectx}{\expr[m]}{\val[m]}
    \\
    \trm{\ncstrpatt{k} is the first pattern to match \val[m]}
    \\
    \evj
    { \ectx \eext \patternmatch~(\ncstrpatt{k})~\val[m] }
    {\expr[k]}
    {\val}
  }{
    \evj
    {\ectx}
    {\ensuremath{
        \left(
        \begin{array}{l}
          \ttt{match \expr[m]\ with} \\
          \ttt{\tabT | \ncstrpatt{1}\ -> \expr[1]} \\
          \ttt{\tabT \ldots} \\
          \ttt{\tabT | \ncstrpatt{i}\ -> \expr[i]}
        \end{array}
        \right)
      }}
    {\val}
  }
}

%%% Local Variables:
%%% mode: latex
%%% TeX-master: "main"
%%% End: 

\newcommand{\ercst}{
  \inference[]{
  }{
    \evj{\ectx}{\cst}{\cst}
  }
}

\newcommand{\ervar}{
  \inference[]{
    \var \in \dom\ \ectx
  }{
    \evj{\ectx}{\var}{\ectx\ \var}
  }
}

\newcommand{\ertup}{
  \inference[]{
    \evj{\ectx}{\expr[n]}{\val[n]}
    &
    \ldots
    &
    \evj{\ectx}{\expr[1]}{\val[1]}
  }{
    \evj{\ectx}{\ntupleexpr}{\ntupleval}
  }
}

\newcommand{\ercstr}{
  \inference[]{
    \evj{\ectx}{\expr[n]}{\val[n]}
    &
    \ldots
    &
    \evj{\ectx}{\expr[1]}{\val[1]}
  }{
    \evj{\ectx}{\ttt{\cstr(\ntupleexpr)}}{\cstr(\ntupleval)}
  }
}

\newcommand{\erappcst}{
  \inference[]{
    \evj{\ectx}{\expr[p]}{\val[p]}
    &
    \evj{\ectx}{\expr[f]}{\cst}
  }{
    \evj{\ectx}{\appexpr}{\cst\ \val[p]}
  }
}

\newcommand{\erapp}{
  \inference[]{
    \evj{\ectx}{\expr[p]}{\val[p]}
    &
    \evj{\ectx}{\expr[f]}{\closure[b]}
    \\
    \evj{\ectx[b] \eext \var \emapsto \val[p]}{\expr[b]}{\val}
  }{
    \evj{\ectx}{\appexpr}{\val}
  }
}

\newcommand{\erapprec}{
  \inference[]{
    \evj{\ectx}{\expr[p]}{\val[p]}
    &
    \evj{\ectx}{\expr[f]}{\recclosure[b]}
    \\
    \evj{\ectx[b] \eext \var[f] \emapsto \recclosure[b] \eext \var
      \emapsto \val[p]}{\expr[b]}{\val}
  }{
    \evj{\ectx}{\appexpr}{\val}
  }
}

\newcommand{\erlet}{
  \inference[]{
    \evj{\ectx}{\expr[1]}{\val[1]}
    &
    \evj
    { \ectx \eext \patternmatch~\patt~\val[1] }
    {\expr[2]}
    {\val}
  }{
    \evj
    {\ectx}
    {\ttt{let \patt\ = \expr[1]\ in \expr[2]}}
    {\val}
  }
}

\newcommand{\erfun}{
  \inference[]{
  }{
    \evj
    {\ectx}
    {\funexpr}
    {\closure}
  }
}

\newcommand{\erletrec}{
  \inference[]{
    \evj
    { \ectx \eext \var[f] \emapsto \recclosure}
    {\expr[2]}
    {\val}
  }{
    \evj
    {\ectx}
    {\ensuremath{
        \ttt{let rec \var[f] = \funexpr\ in \expr[2]}
      }}
    {\val}
  }
}

\newcommand{\ermatch}{
  \inference[]{
    \evj{\ectx}{\expr[m]}{\val[m]}
    \\
    \trm{\ncstrpattDots{k} is the first pattern to match \val[m]}
    \\
    \evj
    { \ectx \eext \patternmatch~(\ncstrpattDots{k})~\val[m] }
    {\expr[k]}
    {\val}
  }{
    \evj
    {\ectx}
    {\ensuremath{
        \left(
        \begin{array}{l}
          \ttt{match \expr[m]\ with} %\\
%          \ttt{\tabT | \ncstrpattDots{1}\ -> \expr[1]} \\
          \ttt{\ | \ldots} %\\
%          \ttt{\tabT | \ncstrpattDots{i}\ -> \expr[i]}
        \end{array}
        \right)
      }}
    {\val}
  }
}

%%% Local Variables:
%%% mode: latex
%%% TeX-master: "main"
%%% End: 


% types

\newcommand{\basetypes}{\ensuremath{B}}
\newcommand{\typevars}{\ensuremath{A}}
\newcommand{\typecstrs}{\ensuremath{\vec{B}}}

\newcommand{\type}[1][]{\ensuremath{\tau_{#1}}}
\newcommand{\tybase}[1][]{\ensuremath{\iota_{#1}}}
\newcommand{\tyvar}{\ensuremath{\alpha}}
\newcommand{\tycstr}{\ensuremath{\vec{\iota}}}

\newcommand{\arity}{\tsf{Arity}}

% frequently used expressions and values

\newcommand{\ntupleexpr}{\ttt{\expr[1],\ \ldots,\ \expr[n]}}
\newcommand{\appexpr}{\expr[f]~\expr[p]}
\newcommand{\funexpr}{\ttt{fun~\var~->~\expr[b]}}

\newcommand{\ntupleval}{\ensuremath{\val[1],\ \ldots,\ \val[n]}}
\newcommand{\closure}[1][]{
  \ensuremath{(\funexpr,\ \ectx[#1])}
}
\newcommand{\recclosure}[1][]{
  \ensuremath{(\var[f], \funexpr,\ \ectx[#1])}
}

\newcommand{\vstup}[1][]{\ensuremath{s_{#1}^{\uparrow}}}
\newcommand{\vstdown}[1][]{\ensuremath{s_{#1}^{\downarrow}}}
\newcommand{\varapp}{\var[\mathit{app}]}

\newcommand{\ncstrpatt}[1]{\ttt{\cstr[#1](\var[\ensuremath{#1,1}],\ \ldots,\ \var[\ensuremath{#1,j_{#1}}])}}


\newcommand{\ncstrpattDots}[1]{\ttt{\cstr[#1](\ensuremath{\ldots_{#1}})}}

% monitors and monitored evaluation trees

\newcommand{\mon}{\ensuremath{\mathcal{M}}}
\newcommand{\states}{\ensuremath{\Sigma}}
\newcommand{\st}[1][]{\ensuremath{\sigma_{#1}}}
\newcommand{\trans}{\ensuremath{\rho}}

\newcommand{\monplus}{\ensuremath{\mon_{\ttt{+}}}}
\newcommand{\transplus}{\ensuremath{\trans_{\ttt{+}}}}

\newcommand{\monps}{\ensuremath{\mon_{\ttt{+}/\ttt{-}}}}
\newcommand{\transps}{\ensuremath{\trans_{\ttt{+}/\ttt{-}}}}

\newcommand{\rt}{\evjvar}

\newcommand{\mevjs}{\ensuremath{J^{m}}}
\newcommand{\mevjvar}[1][]{\ensuremath{j^{m}_{#1}}}
%%%\newcommand{\mevj}[3]{\ensuremath{#2,\ #1,\ #3}}
\newcommand{\mevj}[3]{\ensuremath{#1,\ #3}}
\newcommand{\mevt}[1][]{\ensuremath{\mathfrak{t}^{m}_{#1}}}

\newcommand{\stupa}[1][]{\ensuremath{\st[#1]^{\uparrow}}}
\newcommand{\stdown}[1][]{\ensuremath{\st[#1]^{\downarrow}}}

\newcommand{\monitor}{\tsf{Augment}\xspace}

\newcommand{\smstepcst}{
    \inference[\scst]{
    }{
      \mevj{\evj{\ectx}{\cst}{\cst}}{\st}{\st'}
    }
}

\newcommand{\smstepvar}{
    \inference[\svar]{
    }{
      \mevj{\evj{\ectx}{\var}{\ectx\ \var}}{\st}{\st'}
    }
}

\newcommand{\smstepappcst}{
  \inference[\sappcst]{
    \mevj{\evj{\ectx}{\expr[p]}{\val[p]}}{\st}{\st'}
    &
    \mevj{\evj{\ectx}{\expr[f]}{\cst}}{\st'}{\st''}
  }{
    \mevj{\evj{\ectx}{\appexpr}{\cst\ \val[p]}}{\st}{\st''}
  }
}

% \newcommand{\smstepapp}{
%   \inference[\sapp]{
%     \mevj{\evj{\ectx}{\expr[p]}{\val[p]}}{\st}{\st'}
%     &
%     \mevj{\evj{\ectx}{\expr[f]}{\closure[b]}}{\st'}{\st''}
%     &
%     \mevj{\evj{\ectx[b] \eext \var \emapsto \val[p]}{\expr[b]}{\val}}{\st''}{\st'''}
%   }{
%     \mevj{\evj{\ectx}{\appexpr}{\val}}{\st}{\st'''}
%   }
% }

% \newcommand{\smsteplet}{
%   \inference[\slet]{
%     \mevj{\evj{\ectx}{\expr[1]}{\val[1]}}{\st}{\st'}
%     &
%     \mevj
%     {
%       \evj
%       { \ectx \eext \patternmatch~\patt~\val[1] }
%       {\expr[2]}
%       {\val}
%     }{
%       \st'
%     }{
%       \st''
%     }
%   }{
%     \mevj
%     {
%       \evj
%       {\ectx}
%       {\ensuremath{
%           \ttt{let \patt\ = \expr[1]\ in \expr[2]}
%        }}
%       {\val}
%     }{
%       \st
%     }{
%       \st''
%     }
%   }
% }

%%% Local Variables:
%%% mode: latex
%%% TeX-master: "main"
%%% End: 

% \newcommand{\functx}{\ectx[\mathit{fun}]}
% \newcommand{\rcl}{\trm{(\var[f],
% \exprFun \var[p] \exprArrow \ensuremath{\expr[b],}
% \functx)}}

\newcommand{\mstepcst}{
  \inference[\scst]{
  }{
    \mevj{\evj{\ectx}{\cst}{\cst}}{\stupa}{\stdown}
  }
}

\newcommand{\mstepvar}{
  \inference[\svar]{
  }{
    \mevj{\evj{\ectx}{\var}{\ectx\ \var}}{\stupa}{\stdown}
  }
}

\newcommand{\msteptup}{
  \inference[\stup]{
    \mevj{\evj{\ectx}{\expr[n]}{\val[n]}}{\stupa[n]}{\stdown[n]}
    &
    \ldots
    &
    \mevj{\evj{\ectx}{\expr[1]}{\val[1]}}{\stupa[1]}{\stdown[1]}
  }{
    \mevj{\evj{\ectx}{\ntupleexpr}{\ntupleval}}{\stupa}{\stdown}
  }
}

\newcommand{\mstepcstr}{
  \inference[\scstr]{
    \mevj{\evj{\ectx}{\expr[n]}{\val[n]}}{\stupa[n]}{\stdown[n]}
    &
    \ldots
    &
    \mevj{\evj{\ectx}{\expr[1]}{\val[1]}}{\stupa[1]}{\stdown[1]}
  }{
    \mevj{\evj{\ectx}{\cstr(\ntupleexpr)}{\cstr(\ntupleval)}}{\stupa}{\stdown}
  }
}

\newcommand{\mstepappcst}{
  \inference[\sappcst]{
    \mevj{\evj{\ectx}{\expr[p]}{\val[p]}}{\stupa[p]}{\stdown[p]}
    &
    \mevj{\evj{\ectx}{\expr[f]}{\cst}}{\stupa[f]}{\stdown[f]}
  }{
    \mevj{\evj{\ectx}{\appexpr}{\cst\ \val[p]}}{\stupa}{\stdown}
  }
}

\newcommand{\mstepapp}{
  \inference[\sapp]{
    \mevj{\evj{\ectx}{\expr[p]}{\val[p]}}{\stupa[p]}{\stdown[p]}
    &
    \mevj{\evj{\ectx}{\expr[f]}{\closure[b]}}{\stupa[f]}{\stdown[f]}
    &
    \mevj{\evj{\ectx[b] \eext \var \emapsto \val[p]}{\expr[b]}{\val}}{\stupa[b]}{\stdown[b]}
  }{
    \mevj{\evj{\ectx}{\appexpr}{\val}}{\stupa}{\stdown}
  }
}

\newcommand{\msteplet}{
  \inference[\slet]{
    \mevj{\evj{\ectx}{\expr[1]}{\val[1]}}{\stupa[1]}{\stdown[1]}
    &
    \mevj
    {
      \evj
      { \ectx \eext \patternmatch~\patt~\val[1] }
      {\expr[2]}
      {\val}
    }{
      \stupa[2]
    }{
      \stdown[2]
    }
  }{
    \mevj
    {
      \evj
      {\ectx}
      {\ensuremath{
          \underbrace{
            \ttt{let \patt\ = \expr[1]\ in \expr[2]}
          }_{\expr}
        }}
      {\val}
    }{
      \stupa
    }{
      \stdown
    }
  }
}

\newcommand{\mstepfun}{
  \inference[\sfun]{
  }{
    \mevj
    {
      \evj
      {\ectx}
      {\funexpr}
      {\closure}
    }{
      \stupa
    }{
      \stdown
    }
  }
}

\newcommand{\mstepletrec}{
  \inference[\sletrec]{
    \mevj
    {
      \evj
      { \ectx \eext \var[f] \emapsto \recclosure}
      {\expr[2]}
      {\val}
    }{
      \stupa[2]
    }{
      \stdown[2]
    }
  }{
    \mevj
    {
      \evj
      {\ectx}
      {\ensuremath{
          \underbrace{
            \ttt{let rec \var[f] = \funexpr\ in \expr[2]}
          }_{\expr}
        }}
      {\val}
    }{
      \stupa
    }{
      \stdown
    }
  }
}

\newcommand{\mstepmatch}{
  \inference[\smatch]{
    \mevj{\evj{\ectx}{\expr[m]}{\val[m]}}{\stupa[m]}{\stdown[m]}
    &
    \mevj
    {
      \evj
      { \ectx \eext \patternmatch~(\ncstrpatt{k})~\val[m] }
      {\expr[k]}
      {\val}
    }{
      \stupa[k]
    }{
      \stdown[k]
    }
  }{
    \mevj
    {
      \evj
      {\ectx}
      {\ensuremath{
          \underbrace{
            \left(
              \begin{array}{l}
                \ttt{match \expr[m]\ with} \\
                \ttt{\tabT | \ncstrpatt{1}\ -> \expr[1]} \\
                \ttt{\tabT \ldots} \\
                \ttt{\tabT | \ncstrpatt{i}\ -> \expr[i]}
              \end{array}
            \right)
          }_{\expr}
        }}
      {\val}
    }{
      \stupa
    }{
      \stdown
    }
  }
}

%%% Local Variables:
%%% mode: latex
%%% TeX-master: "main"
%%% End: 


% monitor specifications

\newcommand{\stransformers}{\ensuremath{T}}

\newcommand{\ms}[1][]{\ensuremath{\tsf{Spec}_{#1}}}
\newcommand{\tsel}{\tsf{transsel}}
\newcommand{\tiCheck}{\tsf{trans}}
\newcommand{\tselEnter}{\tsf{enter}}
\newcommand{\tselExit}{\tsf{exit}}


\newcommand{\monfun}{\ttt{f}} % the monitored function
\newcommand{\sttrans}{\ensuremath{mtrans}} % state transformer

\newcommand{\sem}[1]{\ensuremath{[|#1|]}}

\newcommand{\msplus}{\ensuremath{\ms[\ttt{+}]}}
\newcommand{\tselplus}{\ensuremath{\tsel_{\ttt{+}}}}

\newcommand{\msid}{\ensuremath{\ms[\mathit{id}]}}
\newcommand{\tselid}{\ensuremath{\tsel_{\mathit{id}}}}

% product construction

\newcommand{\moncstr}{\ttt{m}}

\newcommand{\monadic}{\tsf{monadic}\xspace}
\newcommand{\premonadic}{\tsf{pre\_m}}

\newcommand{\freshvars}{\tsf{FreshVars}}
\newcommand{\freshvar}{\tsf{FreshVar ()}}

\newcommand{\quo}[1]{\tsf{``}#1\tsf{''}}
\newcommand{\aq}[1]{`#1`}
\newcommand{\depremon}{\tsf{de\_premonadize}}
\newcommand{\product}{\tsf{Transform}\xspace}

\newcommand{\scstprod}{
  \begin{minipage}{1.0\linewidth}
    \ttt{update \aq{\tsel\ \cst} >>= fun () -> \\
      unit (fun \var[1] -> \\
      \tabT \ldots \\
      \tabTT unit (fun \var[n] -> \\
      \tabTTT unit (\cst\ \var[1] \ldots\ \var[n])) \ldots\ )
    }
  \end{minipage}
}

\newcommand{\svarprod}{
  \begin{minipage}{1.0\linewidth}
    \ttt{update \aq{\tsel\ \var} >>= fun () -> \\
      unit \var
    }
  \end{minipage}
}

\newcommand{\sappprod}{
  \begin{minipage}{1.0\linewidth}
    \ttt{\aq{\product\ \expr[p]} >>= fun \var[p] -> \\
      \aq{\product\ \expr[f]} >>= fun \var[f] -> \\
      \var[f] \var[p]
    }
  \end{minipage}
}

% \newcommand{\sletprod}{
%   \begin{minipage}{1.0\linewidth}
%     \ttt{(fun \vst\ -> \\
%       \tabT let \patt, $\vst[1]'$\ = \aq{\product\ \expr[1]} \vst\ in \\
%       \tabT \aq{\product\ \expr[2]} $\vst[1]'$)
%     }
%   \end{minipage}
% }

%%% Local Variables:
%%% mode: latex
%%% TeX-master: "main"
%%% End: 

\newcommand{\cstprod}{
  \begin{minipage}{1.0\linewidth}
    \ttt{update \aq{\tsel\ \cst\ $\uparrow$} >>= fun () -> \\
      update \aq{\tsel\ \cst\ $\downarrow$} >>= fun () -> \\
      unit (fun \var[1] -> \\
      \tabT \ldots \\
      \tabTT unit (fun \var[n] -> \\
      \tabTTT unit (\cst\ \var[1] \ldots\ \var[n])) \ldots\ )
    }
  \end{minipage}
}

\newcommand{\varprod}{
  \begin{minipage}{1.0\linewidth}
    \ttt{update \aq{\tsel\ \var\ $\uparrow$} >>= fun () -> \\
      update \aq{\tsel\ \var\ $\downarrow$} >>= fun () -> \\
      unit \var
    }
  \end{minipage}
}

\newcommand{\tupleprod}{
  \begin{minipage}{1.0\linewidth}
    \ttt{update \aq{\tsel\ $($\ntupleexpr$)$ $\uparrow$} >>= fun () ->\\
      \aq{\product\ \expr[n]} >>= fun \var[n] -> \\
      \ldots \\
      \aq{\product\ \expr[1]} >>= fun \var[1] -> \\
      update \aq{\tsel\ $($\ntupleexpr$)$ $\downarrow$} >>= fun () -> \\
      unit (\var[1], \ldots, \var[n])
    }
  \end{minipage}
}

\newcommand{\cstrprod}{
  \begin{minipage}{1.0\linewidth}
    \ttt{update \aq{\tsel\ $($\cstr(\ntupleexpr)$)$ $\uparrow$} >>= fun () ->\\
      \aq{\product\ \expr[n]} >>= fun \var[n] -> \\
      \ldots \\
      \aq{\product\ \expr[1]} >>= fun \var[1] -> \\
      update \aq{\tsel\ $($\cstr(\ntupleexpr)$)$ $\downarrow$} >>= fun () -> \\
      unit (\cstr (\var[1], \ldots, \var[n]))
    }
  \end{minipage}
}

\newcommand{\appcstprod}{
  \begin{minipage}{1.0\linewidth}
    \ttt{update \aq{\tsel\ $($\appexpr$)$ $\uparrow$} >>= fun () ->\\
      \aq{\product\ \expr[p]} >>= fun \var[p] -> \\
      \aq{\product\ \expr[f]} >>= fun \var[f] -> \\
      \var[f] \var[p] >>= fun \var[{\cst\,\val[p]}] -> \\
      update \aq{\tsel\ $($\appexpr$)$ $\downarrow$} >>= fun () -> \\
      unit \var[{\cst\,\val[p]}]
    }
  \end{minipage}
}

\newcommand{\appprod}{
  \begin{minipage}{1.0\linewidth}
    \ttt{update \aq{\tsel\ $($\appexpr$)$ $\uparrow$} >>= fun () ->\\
      \aq{\product\ \expr[p]} >>= fun \var[p] -> \\
      \aq{\product\ \expr[f]} >>= fun \var[f] -> \\
      \var[f] \var[p] >>= fun \varapp -> \\
      update \aq{\tsel\ $($\appexpr$)$ $\downarrow$} >>= fun () -> \\
      unit \varapp
    }
  \end{minipage}
}

\newcommand{\funprod}{
  \begin{minipage}{1.0\linewidth}
    \ttt{update \aq{\tsel\ $($\funexpr$)$ $\uparrow$} >>= fun () ->\\
      update \aq{\tsel\ $($\funexpr$)$ $\downarrow$} >>= fun () -> \\
      unit (fun \var\ -> \aq{\product\ \expr[b]})
    }
  \end{minipage}
}

\newcommand{\letprod}{
  \begin{minipage}{1.0\linewidth}
    \ttt{update \aq{\tsel\ \expr\ $\uparrow$} >>= fun () -> \\
      (\aq{\product\ \expr[1]}\hspace{1.3ex} >>= fun \patt\ -> \\
      \tabT\hspace{-2ex}  \aq{\product\ \expr[2]}) >>= fun \var[v] -> \\
      update \aq{\tsel\ \expr\ $\downarrow$} >>= fun () -> \\
      unit \var[v]
      % update \aq{\tsel\ \expr\ $\uparrow$} >>= fun () ->\\
      % (fun \vstup -> \\
      % \tabT let \patt, \vstdown[1]\ = \aq{\product\ \expr[1]} \vstup\ in \\
      % \tabT \aq{\product\ \expr[2]} \vstdown[1]) >>= fun \var[v] -> \\
      % update \aq{\tsel\ \expr\ $\downarrow$} >>= fun () -> \\
      % unit \var[v]
    }
  \end{minipage}
}

\newcommand{\letrecprod}{
  \begin{minipage}{1.0\linewidth}
    \ttt{update \aq{\tsel\ \expr\ $\uparrow$} >>= fun () ->\\
      (let rec \var[f] = fun \var\ -> \aq{\product\ \expr[b]} \\
      \tabT\hspace{-2ex} in \aq{\product\ \expr[2]}) >>= fun \var[v]\ -> \\
      update \aq{\tsel\ \expr\ $\downarrow$} >>= fun () -> \\
      unit \var[v]
    }
  \end{minipage}
}

\newcommand{\matchprod}{
  \begin{minipage}{1.0\linewidth}
    \ttt{update \aq{\tsel\ \expr\ $\uparrow$} >>= fun () -> \\
      \aq{\product\ \expr[m]} >>= fun \var[m]\ -> \\
      (match \var[m] with \\
      \tabT | \ncstrpatt{1}\ -> \aq{\product\ \expr[1]} \\
      \tabT \ldots \\
      \tabT | \ncstrpatt{i}\ -> \aq{\product\ \expr[i]}) >>= fun \var[v]\ -> \\
      update \aq{\tsel\ \expr\ $\downarrow$} >>= fun () -> \\
      unit \var[v]
    }
  \end{minipage}
}

%%% Local Variables:
%%% mode: latex
%%% TeX-master: "main"
%%% End: 


% correctness

\newcommand{\seq}[1][]{\ensuremath{\tsf{Seq}_{#1}}}
\newcommand{\seqm}[1][]{\ensuremath{\tsf{Seqm}_{#1}}}

\newcommand{\peelval}{\tsf{peel\_val}}
\newcommand{\peelenv}{\tsf{peel\_env}}
\newcommand{\simpprod}{\mbox{\tsf{remove\_update}}}

\newcommand{\pectx}{\peelenv~\ectx}
\newcommand{\valprod}{\val[\mathit{prod}]}
\newcommand{\exprprod}{\expr[\mathit{prod}]}
\newcommand{\exprprodid}{\expr[\mathit{id}]}
\newcommand{\exprprodsimp}{\expr[\mathit{simp}]}

\newcommand{\correctness}{Theorem~\ref{thm-correctness}}
\newcommand{\lemmsimpprod}{Lemma~\ref{lem-simpprod}}
\newcommand{\lemmvals}{Lemma~\ref{lem-vals}}
\newcommand{\lemmvalsconcr}{Lemma~\ref{lem-valsconcr}}
\newcommand{\lemmstates}{Lemma~\ref{lem-states}}

% \newtheorem{corrproof}[thm]{Proof~of~Theorem~\ref{thm-correctness}} % incompatible w/ llncs

% comparing monitored evaluation steps and instrumentation cases

\newcommand{\mstepprodrel}{\ensuremath{\leftrightarrow}}

% experiments

\newcommand{\rpt}{\tsf{Prod}}
\newcommand{\funv}{\tsf{FunV}}

\newcommand{\sflinear}{\tsf{LinIneq}\xspace}
\newcommand{\sfchan}{\tsf{Set}\xspace} 
\newcommand{\liter}{\tsf{TI-Term}\xspace} 
\newcommand{\literrank}{\tsf{R-Term}\xspace} 

\newcommand{\hobenchmark}{\texttt{(H)}}
%\newcommand{\hobenchmark}{{\tiny\ensuremath{\heartsuit}}}
\newcommand{\benchmarkontrees}{{\tiny\ensuremath{\clubsuit}}}
\newcommand{\benchmarkonfiledescriptors}{{\tiny\ensuremath{\spadesuit}}}


%%% Local Variables: 
%%% mode: latex
%%% TeX-master: "main"
%%% End: 


%%% Local Variables: 
%%% mode: latex
%%% TeX-master: "main"
%%% End: 




\ifthenelse{\equal{\isDraft}{true}}{
  \usepackage[a4paper, top=2cm, bottom=3cm, left=2cm, right=2cm]{geometry}
  \pagestyle{plain}
}{}

\begin{document}

\nocite{DOsualdoSAS13}

\ifthenelse{\equal{\isDraft}{false}}{
  \title{An SMT-based Approach to Coverability Analysis}
  \author{
    Javier~Esparza\inst{1}
    \and
    Rupak~Majumdar\inst{2}
    \and
    Philipp~Meyer\inst{1}
    \and
    Filip~Niksic\inst{2}
    \and
    Rusl\'{a}n~Ledesma-Garza\inst{1}
  }
  \institute{
    Technische Universit\"at M\"unchen
    \and
    MPI-SWS,~Kaiserslautern~and~Saarbr\"ucken
  }
  
  \maketitle

  \begin{abstract}
    Model checkers based on Petri net coverability have been used
    successfully in recent years to verify safety properties of concurrent
    shared-memory or asynchronous message-passing software. The bottleneck
    in the verification process is the iterative computation of the
    (backward) reachable or (forward) coverable set, and most recent
    research has focused on non-trivial heuristics to speed it up.

    In this paper, we revisit a constraint approach to the problem based on
    classical Petri net analysis techniques: marking equation, place
    invariants, and traps. Given a marking M to be covered, the approach
    constructs constraints corresponding to increasingly precise
    overapproximations of the reachable markings covering M. If one of
    these constraints is unsatisfiable, then M is not coverable.

    Previous work on this approach suffered from the absence of efficient
    decision procedures for linear arithmetic, which nowadays exist as a
    part of SMT solvers. We show how to utilize an SMT solver to implement
    the constraint approach, and additionally, to generate a minimized
    inductive invariant from a safety proof.

    We empirically evaluate our procedure on a large set of existing Petri
    net benchmarks arising out of software verification. Surprisingly, even
    though our technique is incomplete, we show that it can quickly
    discharge all safe instances, usually competitive with, and occasionally
    much faster than, state-of-the-art techniques. Additionally, the
    inductive invariants computed are usually orders of magnitude smaller
    than those produced by existing solvers.
  \end{abstract}
}{
}

\section{Introduction}

Tools and techniques for proving program termination are important for
increasing software quality~\cite{CACM}.
System routines written in imperative programming languages received a
significant amount of attention recently,
e.g.,~\cite{copytrick,Julia,KroeningCAV10,GieslBytecode,YangESOP08,SepLogTerm}. 
A number of the proposed approaches rely on transition invariants -- a
termination argument that can be constructed incrementally using
abstract interpretation~\cite{transitioninvariants}.
Transition invariants are binary relations over program states.
Checking if an incrementally constructed candidate is in fact a
transition invariant of the program is called binary reachability
analysis.
For imperative programs, its efficient implementation can be obtained
by a reduction to the reachability analysis, for which practical tools
are available, e.g.,~\cite{Slam,Blast,fsoft,Impact}.
The reduction is based on a program transformation that stores one
component of the pair of states under consideration in auxiliary program
variables, and then checks if the pair is in the transition
invariant~\cite{copytrick}. 
The transformed program is verified using an existing safety checker. 
If the safety checker succeeds then the original program terminates on all inputs.  
% The set of reachable program states of the transformed program can be
% directly used in the binary reachability analysis.

For functional programs, recent approaches for proving termination
apply the size change termination (SCT) argument~\cite{SCTPOPL2001}.
This argument requires checking the presence of an infinite descent within
data values passed to application sites of the program on any infinite
traversal of the call graph.
This check can be realized in two steps.
First, every program function is translated into a set of so-called
size-change graphs that keep track of decrease in values between the
actual arguments and values at the application sites in the
function.
Second, the presence of a descent is checked by computing a transitive
closure of the size-change graphs. 
Originally, the SCT analysis was formulated for first order functional
programs manipulating well-founded data, yet using an appropriate
control-flow analysis, it can be extended to higher order programs,
see
e.g.,~\cite{SereniICFP07,Sereni05terminationanalysis,SereniAPLAS05}.
Alternatively, an encoding into term rewriting can be used to make
sophisticated decidable well-founded orderings on terms applicable to proving
termination of higher order programs~\cite{GieslHaskell}.

The SCT analysis is a decision problem (that checks if there is an infinite descent in the abstract program defined by size change graphs), however it is an incomplete method for proving termination. 
SCT can return ``don't know'' for terminating programs that
manipulate non-well-founded data, e.g., integers, or when an interplay
of several variables witnesses program termination. 
In such cases, a termination prover needs to apply a more general termination argument.
Usually, such termination arguments require proving that certain expressions over program variables decrease as the computation progresses and yet the decrease cannot happen beyond a certain bound.

In this paper, we present a general approach for proving termination of higher order functional programs that goes beyond the SCT analysis.
Our approach explores the applicability of transition invariants to
this task by proposing an extension (wrt.\ imperative case) that deals
with partial applications, a programming construct that is particular for functional programs.
Partial applications of curried functions, i.e., functions that return other functions, represent a major obstacle for the binary reachability analysis. 
For a curried function, the set of variables whose values need to be stored in auxiliary variables keeps increasing as the function is subsequently applied to its arguments. 
However these arguments are not necessarily supplied simultaneously,
which requires intermediate storage of the argument values given so
far.
In this paper, we address such complications.

We develop the binary reachability analysis for higher order programs in two steps.
First, we show how intermediate nodes of program evaluation trees, \mbox{so-called} judgements, can be augmented with auxiliary values needed for tracking binary reachability. 
The auxiliary values store arguments provided at application sites.
Then, we show how this augmentation can be performed on the program source code such that the evaluation trees of the augmented program correspond to the result of augmenting the evaluation trees of the original program.
The source code transformation introduces additional parameters to functions occurring in the program. For curried functions, these additional parameters are interleaved with the original parameters, which allows us to deal with partial applications.

Our binary reachability analysis for higher order programs opens up an approach for termination proving in the presence of higher order functions that exploits a highly optimized safety checker, e.g.,~\cite{TerauchiPOPL10,RupakRanjit,KobayashiPLDI11,Dsolve,HMC},
for checking the validity of a candidate termination argument.
Hence, we can directly benefit from sophisticated abstraction
techniques and algorithmic improvements offered by these tools, as inspired by~\cite{copytrick}.

In summary, this paper makes the following contributions.
%
\begin{itemize}
\item A notion of binary reachability analysis for 
  higher order functional programs.
\item A program transformation that reduces the binary
  reachability analysis to the reachability analysis.
  % code   transformation. 
\item An implementation of our approach and its evaluation on micro
  benchmarks from the literature.
\end{itemize}


%%% Local Variables: 
%%% mode: latex
%%% TeX-master: "main"
%%% End: 

\section{Illustration}

In this section we illustrate what our transformation adds to the
program in order to keep track of pairs of argument valuations for
checking a transition invariant candidate.

We consider the following curried function \texttt{f} that has a type
\texttt{x:int -> y:int -> ret:int}.
Here, we annotated the parameter and return value types with
identifiers to improve readability.
%
\begin{center}
\begin{minipage}[h]{.8\linewidth}
\begin{small}
\begin{verbatim}
let rec f x = if x > 0 then f (x-1) else fun y -> f x y
\end{verbatim}
\end{small}
\end{minipage}
\end{center}
%
This function shows that -- in contrast to proving termination of
recursive procedures in imperative programs -- it is important to
differentiate between partial and complete applications when dealing
with curried functions.
First, we observe that any partial application of \texttt{f}
terminates.
For example, \texttt{f 10} stops after ten recursive calls and returns
a function \texttt{fun y -> f 10 y} where \texttt{f} is bound to a
closure.
That is, there is no infinite sequence of \texttt{f} applications that
are passed only one argument.
In contrast, any complete application of \texttt{f} does not terminate.
For example, \texttt{f 1 1} will lead to an infinite sequence of
\texttt{f} applications such that each of them is given two arguments.

Our binary reachability analysis takes as input a specification that
determines which kind of applications we want to keep track of. 
The specification consists of a function identifier, e.g., \texttt{f},
a number of parameters, e.g., one, and a transition invariant candidate. 
Then, such a specification requires that applications of \texttt{f} to
one argument satisfy the transition invariant candidate.
Alternatively, we may focus on applications of \texttt{f} to two
arguments.

Once the specification is given, we transform \texttt{f} into a
function \texttt{f\_m} that keeps track of arguments on which \texttt{f} was
applied using additional parameters \texttt{old\_x} and~\texttt{old\_y}.
As a result, \texttt{f\_m} fulfills two requirements. 
First, it computes a result value \texttt{res} such that $\mathtt{res}
= \mathtt{f\ x\ y}$. 
Second, it computes new values of additional parameters.
If \texttt{f\_m} was an imperative program, we would obtain the type
%
\begin{small}
\begin{verbatim}
        x:int * y:int * copied:bool * old_x:int * old_y:int ->
          ret:int * new_copied:bool * new_x:int * new_y:int
\end{verbatim}
\end{small}
% 
where \texttt{new\_x} and \texttt{new\_y} are computed as follows. 
If \texttt{old\_x} already stores a value that was given to \texttt{x} in the past, i.e.,
if $\mathtt{copied} = \mathtt{true}$, then $\mathtt{new\_x} = \mathtt{old\_x}$. 
Otherwise, \texttt{f\_m} can nondeterministically either store
\texttt{x} in \texttt{new\_x} and set $\mathtt{new\_copied} =
\mathtt{true}$, or leave $\mathtt{new\_x} = \mathtt{old\_x}$
and~$\mathtt{new\_copied} = \mathtt{false}$.
The computation of \texttt{new\_y} is similar.
Given a transformed program, checking binary reachability amounts to
checking that at each application site the pair of tuples
$(\mathtt{old\_x}, \mathtt{old\_y})$ and $(\mathtt{x}, \mathtt{y})$
satisfies the transition invariant whenever \texttt{copied}
is~\texttt{true}.

Due to partial applications we cannot expect that values of \texttt{x}
and \texttt{y} are provided simultaneously, which complicates both
computation of \texttt{new\_x} and \texttt{new\_y} and checking if
$(\mathtt{old\_x}, \mathtt{old\_y})$ together with $(\mathtt{x},
\mathtt{y})$ satisfy the transition invariant. 
Hence, we need to keep track of arguments as they are provided, which
requires ``waiting'' for missing arguments.
We implement this waiting process by introducing additional parameters
\texttt{old\_state\_x} and \texttt{old\_state\_y} for each partial
application, together with their updated versions
\texttt{new\_state\_x} and~\texttt{new\_state\_y}.
Each additional parameter accumulates arguments in its first
component, and it keeps a tuple of previously provided arguments in
its second component. 
We obtain the following type for \texttt{f\_m}.
%
\begin{small}
\begin{verbatim}
x:int 
-> old_state_x:((int * int) *                (* accumulate x and y     *)
                (bool * int * int))          (* store copied, x, and y *)
-> (y:int 
    -> old_state_y:((int * int) *            (* accumulate x and y     *)    
                    (bool * int * int))      (* store copied, x, and y *)
    -> ret:int * new_state_y:((int * int) * 
                              (bool * int * int))) *
   new_state_x:((int * int) * 
                (bool * int * int))
\end{verbatim}
\end{small}
% 
We refer to \texttt{(int * int) * (bool * int * int)} as~\texttt{state}. 
Then \texttt{f\_m} has the type:
%
\begin{small}
\begin{verbatim}
      x:int -> old_state_x:state -> 
        (y:int -> old_state_y:state -> ret:int * new_state_y:state) *
        new_state_x:state
\end{verbatim}
\end{small}
% 
We formalize the above transformation in
Section~\ref{sec-transformation}. 
Figure~\ref{fig-ex-prod} presents a detailed execution protocol of
applying our transformation on the above program.

\iffalse
When dealing with unary applications, the binary reachability
analysis can show that every pair of unary applications is included
in a transition invariant using the following check.
\fi

Note that if complete applications of a function terminate, then every
partial application of the function terminates.
For example, consider the following function~\texttt{g}.
%
\begin{center}
\begin{verbatim}
let rec g x = if x > 0 then g (x-1) else fun y -> x+y
\end{verbatim}
\end{center}
%
This function does not have any infinite application sequences neither
for complete nor for partial applications.


%%% Local Variables: 
%%% mode: latex
%%% TeX-master: "main"
%%% End: 

\section{Preliminaries}

A \emph{Petri net} is a tuple $(P, T, F, M_0)$, where $P$ is the set of
\emph{places}, $T$ is the set of \emph{transitions},
$F \subseteq P\times T \cup T\times P$ is the \emph{flow relation}
and $M_0$ is the initial marking.
We identify $F$ with its characteristic function
$P\times T \cup T\times P \to \{0, 1\}$.

For $x\in P\cup T$, the \emph{pre-set} is
$\pre{x}=\{y\in P\cup T\mid (y,x) \in F\}$
and the \emph{post-set} is $\post{x}=\{y\in P\cup T\mid (x,y) \in F\}$.
The pre- and post-set of a subset of $P \cup T$ are the union of
the pre and post-sets of its elements.

A \emph{marking} of a Petri net is a function $M\colon P \to \mathbb{N}$,
which describes the number of tokens $M(p)$ that the marking puts in
each place $p$.

Petri nets are represented graphically as follows: places and transitions
are represented as circles and boxes, respectively. An element $(x,y)$
of the flow relation is represented by an arc leading from $x$ to $y$.
A token on a place $p$ is represented by a black dot in the circle
corresponding to $p$.

A transition $t \in T$ is \emph{enabled at $M$} iff
$\forall p \in \pre{t} \colon M(p) \ge F(p, t)$.
If $t$ is enabled at $M$, then $t$ may \emph{fire} or \emph{occur},
yielding a new marking $M'$ (denoted as $M \xrightarrow{t} M'$),
where $M'(p) = M(p) + F(t,p) - F(p,t)$.

A sequence of transitions, $\sigma = t_1 t_2 \ldots t_r$ is an
$\emph{occurence sequence}$ of $N$ iff there exist markings
$M_1, \ldots, M_r$ such that $M_0 \xrightarrow{t_1} M_1
\xrightarrow{t_2} M_2 \ldots \xrightarrow{t_r} M_r$. The marking
$M_r$ is said to be \emph{reachable} from $M_0$ by the occurence
of $\sigma$ (denoted $M \xrightarrow{\sigma} M_r$).

A property $P$ is a safety property expressed over linear arithmetic
formulas. The property $P$ holds on a marking $M$ iff $M \models P$.
Examples of properties are $s_1 \le 2$, $s_1 + s_2 \ge 1$ and
$((s_1 \le 1) \land (s_2 \ge 1)) \lor (s_3 \le 1)$.

A Petri net $N$ satisfies a property $P$ (denoted by $N \models P$)
iff for all reachable markings $M_0 \xrightarrow{\sigma} M$
$M \models P$ holds.

An invariant $I$ of a Petri net $N$ is a property such that $N$ satisfies $I$.
The invariant $I$ is inductive iff for all markings
$M$, if $M \models I$ and $M \xrightarrow{t} M'$ for some $t$, then
$M' \models I$.

A trap is a set of places $S \subseteq P$ such that $\post{S} \subseteq \pre{S}$.
If a trap $S$ is marked in $M_0$, i.e. $\sum_{p \in S} M_0(p) > 0$, then it is also marked in all reachable markings.


\newpage
\section{Method Safety}

The method Safety checks that a given Petri net \verb=N= never violates a property \verb=P=.
We present the method Safety by example on Lamport's 1-bit algorithm [Esparza1997].

\begin{verbatim}
* Method Safety:

  Subprocedure \mathcal{C} constructs state constraints C corresponding to N.
\end{verbatim}

\ifthenelse{\equal{\isDraft}{true}}{\begin{figure}
  \begin{center}
    \begin{tikzpicture}[
      every path/.style={draw, ->, >=stealth, shorten >=2pt, shorten <=2pt}
      ]
      \node[state] (begin) {BEGIN};
      \node[action, below=of begin] (c) {$C:=\mathcal C(N)$};
      \node[decision, below=of c] (satc) {$\text{SAT}(C \cup \{\neg P\})$};
      \node[print, right=of satc] (yes) {N satisfies P};
      \node[print, below=of satc] (dontknow) {Don't know};
      \node[state, right=of yes] (end1) {END};
      \node[state, below=of dontknow] (end2) {END};
      
      \draw (begin) edge (c);
      \draw (c) edge (satc);
      \draw (satc) edge node[above]{NO} (yes);
      \draw (yes) edge (end1);
      \draw (satc) edge node[right]{YES} (dontknow);
      \draw (dontknow) edge (end2);
    \end{tikzpicture}
  \end{center}
  \caption{Diagram for Method Safety}
  \label{fig:method-safety-pseudocode}
\end{figure}}{}

\begin{figure}
  \begin{algorithmic}[1]
    \State $C := \mathcal C(N)$
    \If {SAT$(C \cup \{\neg P\})$}
    \State \Return 'The petri net may not satisfy the property'
    \Else
    \State \Return 'The petri net satisfies the property!'
    \EndIf
  \end{algorithmic}
  \caption{Pseudocode for Method Safety}
  \label{fig:method-safety-pseudocode}
\end{figure}


\begin{verbatim}
* Property of state constraints C: If C U {\neg P} is unsat then N |= P.

* Place equation:
  
  For a given place s the place equation is

  # of tokens in s = initial number of tokens of place s
                     + # times each input transition of s fires
                     - # times each output transition of s fires

* Non-negativity conditions:

  # of tokens in place s           >= 0
  # of times transition t is fired >= 0

* Subprocedure \mathcal{C}:

  Input:
    (S, T, E, M0) : Petri net
  Output:
    C : State constraints

  Pseudocode:

\end{verbatim}

\begin{align*}
  C(S, T, E, M_0) :=& \left( \bigwedge_{s \in S} \left(
    s = M_0(s) + \sum_{(t, s) \in E} t - \sum_{(s, t) \in E} t
  \right) \right) \land
    \left( \bigwedge_{s \in S} s \ge 0 \right) \land
    \left( \bigwedge_{t \in T} t \ge 0 \right)
\end{align*}

\newpage

\begin{verbatim}
* Example

  - Code:

  Process 1:                     |      Process 2:
                                 |
      bit1 := false              |      bit2 := false
      while true do              |      while true do
  p1:   bit1 := true             |  q1:   bit2 := true
  p2:   while bit2 do skip od    |  q2:   if bit1 then
  p3:   <critical section>       |  q3:     bit2 := true
        bit1 := false            |  q4:     while bit1 do skip od
      od                         |          goto q1
                                 |        fi
                                 |  q5:   <critical section>
                                 |        bit2 := false
                                 |      od

  - Property 1: Process 1 and Process 2 are never in their respective critical section at the same time.

  - Property 1: p3 + q5 <= 1

  - Property 2: Process 1 is in at most one state at a time

  - Property 2: p1 + p2 + p3 <= 1
  
  - Property 3: Variable bit1 is either true or false

  - Property 3: bit1 + nbit1 = 1, alternatively written as
                bit1 + nbit1 <= 1 /\ bit1 + nbit1 >= 1

  - Petri net:

\end{verbatim}

\begin{figure}
\begin{center}
  \begin{tikzpicture}
  \draw[box] (-2,8.5) rectangle (1,0);
  \draw[box] (3,8.5) rectangle (7.5,0);
  
  \node at (-0.5, 0.5) {First Process};
  \node at (5, 0.5) {Second Process};

  \node[place, label=left:p3] (p3) at (0,7) {};
  \node[transition, label=left:s3] (s3) at (0,6) {};
  \node[place, tokens=1, label=left:p1] (p1) at (0,5) {};
  \node[transition, label=left:s1] (s1) at (0,4) {};
  \node[place, label=left:p2] (p2) at (0,3) {};
  \node[transition, label=left:s2] (s2) at (0,2) {};

  \draw (p3) edge[flow] (s3);
  \draw (s3) edge[flow] (p1);
  \draw (p1) edge[flow] (s1);
  \draw (s1) edge[flow] (p2);
  \draw (p2) edge[flow] (s2);
  \draw[rounded corners=4mm] (s2) |- (-1.5,1) -- (-1.5,8) [flow] -| (p3);
 
  \node[transition, label=above:t2] (t2) at (4,7) {};
  \node[place, label=right:q3] (q3) at (4,6) {};
  \node[transition, label=right:t3] (t3) at (4,5) {};
  \node[place, label=right:q4] (q4) at (4,4) {};
  \node[transition, label=left:t4] (t4) at (4,3) {};

  \node[place, label=right:q2] (q2) at (5.5,7) {};
  \node[transition, label=right:t5] (t5) at (5.5,6) {};
  \node[place, label=right:q5] (q5) at (5.5,5) {};
  \node[transition, label=right:t6] (t6) at (5.5,4) {};
  \node[place, tokens=1, label=right:q1] (q1) at (5.5,3) {};
  \node[transition, label=right:t1] (t1) at (5.5,2) {};
  
  \draw (q2) edge[flow] (t5);
  \draw (t5) edge[flow] (q5);
  \draw (q5) edge[flow] (t6);
  \draw (t6) edge[flow] (q1);
  \draw (q1) edge[flow] (t1);
  \draw[rounded corners=4mm] (t1) |- (7,1) -- (7,8) [flow] -| (q2);
  
  \draw (q2) edge[flow] (t2);
  \draw (t2) edge[flow] (q3);
  \draw (q3) edge[flow] (t3);
  \draw (t3) edge[flow] (q4);
  \draw (q4) edge[flow] (t4);
  \draw (t4) edge[flow] (q1);
  
  \node[place, label=above:bit1] (bit1) at (2,6) {};
  \node[place, tokens=1, label=below:$\neg$bit1] (nbit1) at (2,4) {};
  \node[place, tokens=1, label=below:$\neg$bit2] (nbit2) at (2,2) {};
  
  \draw (bit1) edge[flow] (s3);
  \draw (s3) edge[flow] (nbit1);
  \draw (nbit1) edge[flow] (s1);
  \draw (s1) edge[flow] (bit1);
  \draw (t2) edge[flow, <->] (bit1);
  \draw (nbit1) edge[flow, <->, bend left=18] (t5);
  \draw (nbit1) edge[flow, <->] (t4);
  \draw (nbit2) edge[flow, <->] (s2);
  \draw (t3) edge[flow] (nbit2);
  \draw (t6) edge[flow, bend left=30] ([xshift=-0.2mm, yshift=1mm] nbit2.east);
  \draw (nbit2) edge[flow] (t1.west);
\end{tikzpicture}
\end{center}
\caption{Petri net for Lamport's 1-bit algorithm}
\label{fig:lamport-petri-net}
\end{figure}


\newpage

\begin{verbatim}
  - State constraints C:

    Place equations:
  


    p1    = 1 - s1      + s3
    ^       ^   ^         ^
    |       |   |         |
    |       |   |         |
    |       |   |         |
    |       |   |        # of tokens given to p1
    |       |   |    
    |       |  # of tokens taken from p1
    |       |
    |      initial number of tokens in p1
    |
    number of tokens in p1

    p2    = 0 + s1 - s2
    p3    = 0      + s2 - s3
    bit1  = 0 + s1      - s3
    nbit1 = 1 - s1      + s3
    q1    = 1 +              - t1           + t4      + t6
    q2    = 0 +              + t1 - t2           - t5
    q3    = 0 +                   + t2 - t3
    q4    = 0 +                        + t3 - t4
    q5    = 0 +                                  + t5 - t6
    nbit2 = 1 +              - t1      + t3           + t6
    
    Non-negativity conditions:
  
    p1    >= 0
    p2    >= 0
    p3    >= 0
    bit1  >= 0
    nbit1 >= 0
    q1    >= 0
    q2    >= 0
    q3    >= 0
    q4    >= 0
    q5    >= 0
    nbit2 >= 0

  - Using property 1, with negated property \neg P:

    p3 + q5 >= 2

    The constraints are satisfiable, therefore we can not
    prove safety with this method.

  - Using property 2, with negated property \neg P:

    p1 + p2 + p3 >= 2
    
    The constraints are unsatisfiable, therefore we
    prove safety with this method.
  
  - Using property 3, with negated property \neg P:

    nbit1 + nbit1 <= 0 \/ nbit1 + nbit1 >= 2
    
    The constraints are unsatisfiable, therefore we
    prove safety with this method.
\end{verbatim}

\newpage

\section{Method Safety by Refinement}

The method Safety by Refinement applies trap conditions to check that a given Petri net \verb=N= never violates a property \verb=P=.

\begin{verbatim}
* Method Safety by Refinement:

  Subprocedure TrapConditions constructs trap conditions C_\theta corresponding to N and A.
  Subprocedure \Delta constructs refinement constraint \delta corresponding to A_\theta.
\end{verbatim}

\ifthenelse{\equal{\isDraft}{true}}{\begin{figure}
  \begin{center}
    \begin{tikzpicture}[
      every path/.style={draw, ->, >=stealth, shorten >=2pt, shorten <=2pt}
      ]
      \node[state] (begin) {BEGIN};
      \node[action, below=of begin] (c) {$C:=\mathcal C(N)$};
      \node[decision, below=of c] (satc) {$\text{SAT}(C \cup \{\neg P\})$};
      \node[action, below=of satc] (modelc) {$A:=\text{Model}(C \cup \{\neg P\})$\\
        $C_{\theta}:=\text{TrapConditions}(N, A)$};
      \node[decision, below=of modelc] (satctheta) {$\text{SAT}(C_\theta)$};
      \node[action, below=of satctheta] (modelctheta) {$A_\theta:=\text{Model}(C_\theta)$\\
        $\delta:=\Delta(A_\theta)$\\
        $C:=C \cup \{\delta\}$};
      \node[print, right=of satc] (yes) {N satisfies P};
      \node[print, right=of satctheta] (dontknow) {Don't know};
      \node[state, right=of yes] (end1) {END};
      \node[state, right=of dontknow] (end2) {END};

      \draw (begin) edge (c);
      \draw (c) edge coordinate[pos=.5] (edgein) (satc);
      \draw (satc) edge node[above]{NO} (yes);
      \draw (yes) edge (end1);
      \draw (satc) edge node[right]{YES} (modelc);
      \draw (modelc) edge (satctheta);
      \draw (satctheta) edge node[above]{NO} (dontknow);
      \draw (dontknow) edge (end2);
      \draw (satctheta) edge node[right]{YES} (modelctheta);
      \draw (modelctheta.south) -- ([yshift=-0.5cm] modelctheta.south)
      -| ([xshift=-2cm] modelctheta.west) |- (edgein);
    \end{tikzpicture}
  \end{center}
  \caption{Diagram for Method Safety by Refinement}
  \label{fig:method-safety-by-refinement-diagram}
\end{figure}}{}

\begin{figure}
  \begin{algorithmic}[1]
    \State $C := \mathcal C(N)$
    \While {SAT$(C \cup \{\neg P\})$}
    \State $A := $Model$(C \cup \{\neg P\})$
    \State $C_\theta := $TrapConditions$(N, A)$
    \If {SAT$(C_\theta)$}
    \State $A_\theta := $Model$(C_\theta)$
    \State $\delta := \Delta(A_\theta)$
    \State $C := C \cup \{\delta\}$
    \Else
    \State \Return 'The petri net may not satisfy the property'
    \EndIf
    \EndWhile
    \State \Return 'The petri net satisfies the property!'
  \end{algorithmic}
  \caption{Pseudocode for Method Safety by Refinement}
  \label{fig:method-safety-by-refinement-pseudocode}
\end{figure}

\begin{verbatim}
* Property of trap conditions C_\theta: If C_\theta is sat then there is a set S such that
  1. S is a trap in the net N
  2. S is marked in the initial marking M0
  3. S is unmarked in the assignment A

* Property of A_\theta: for each place s, A_\theta(s) iff s \in S

* Property of refinement constraint \delta: Constraint \delta refines the abstraction, i.e.
  1. A ^ \delta is unsat (\delta excludes A)
  2. N |= \delta is sat  (\delta is a property of N)
\end{verbatim}

\newpage

\begin{verbatim}
* Subprocedure TrapConditions:

  Input:
    (S, T, E, M0) : Petri net
    A             : Satisfying assignment for C \cup { ~P }
  Output:
    C_\theta      : Trap conditions

  Pseudocode:
  
\end{verbatim}
\begin{align*}
  C_\theta(S, T, E, M_0) :=& \left( \bigwedge_{s \in S} \left( s \Rightarrow
      \bigwedge_{(s, t) \in E} \bigvee_{(t, p) \in E} p
    \right) \right) \land
    \left( \bigvee_{s \in S: M_0(s) > 0} s \right) \land
    \left( \bigwedge_{s \in S: A(s) > 0} \neg s \right)
\end{align*}
\begin{verbatim}

* Subprocedure \Delta:

  Input:
    A_\theta      : Satisfying assignment for C_\theta
  Output:
    \delta        : Refinement constraint \delta

  Pseudocode:

\end{verbatim}
\begin{align*}
  & \delta(A_\theta) := \left( \sum_{A_\theta(s)} s \ge 0 \right)
\end{align*}

\newpage

\begin{verbatim}
* Example

  - State constraints C same as in the example for method Safety.
  
  - Using property 1, with negated property 1 \neg P:

    p3 + q5 >= 2
 
  - Assignment A:
    p1    = 0
    p2    = 0
    p3    = 1
    bit1  = 1
    nbit1 = 0
    q1    = 0
    q2    = 0
    q3    = 0
    q4    = 0
    q5    = 1
    nbit2 = 0

  - Trap conditions C_\theta:

    - Trap implications:
                s1
             ----------
    p1    => p2 \/ bit1
    ^           ^      ^
    |           |      |
    |           |     bit1 \in S
    |           |
    |          p2 \in S
    |
    p1 \in S
    
    p2    => p3 \/ nbit2
    p3    => p1 \/ nbit1
    bit1  => (p1 \/ nbit1) /\ (q3 \/ bit1)
    nbit1 => (p2 \/ bit1) /\ (q1 \/ nbit1) /\ (q5 \/ nbit1)
    q1    => q2
    q2    => (q3 \/ bit1) /\ (q5 \/ nbit1)
    q3    => q4 \/ nbit2
    q4    => q1 \/ nbit1
    q5    => q1 \/ nbit2
    nbit2 => q2 /\ (p3 \/ nbit2)

    - At least one of the initially marked places belongs to S:
    p1 \/ q1 \/ nbit1 \/ nbit2

    - None of the marked places in A belongs to S:
    ~p3 /\ ~q5 /\ ~bit1
  
  - Assignment A_\theta:
    p1    = false
    p2    = true
    p3    = false
    bit1  = false
    nbit1 = true
    q1    = false
    q2    = true
    q3    = true
    q4    = false
    q5    = false
    nbit2 = true

  - Refinement constraint \delta:
    p2 + q2 + q3 + nbit1 + nbit2 >= 1
    ^    ^    ^    ^       ^
    |    |    |    |       |
    ------------------------
      |
      S = {p2, q2, q3, nbit1, nbit2}, therefore \delta excludes
      assignment A in the next iteration
\end{verbatim}

\newpage

\begin{verbatim}

* Trap implication:

  place s \in S =>    /\       \    /
                     /  \       \  /     place p \in S
                    /    \       \/
                   t \in s*   p \in t*   


  "if s is in trap S then for each output transition t at least one successor p is in trap S"


* Refinement constraint \delta:

  \Sigma s >= 1  
  A(s)

  "At least one place in S is always marked"
\end{verbatim}

\newpage
\section{Method Invariant}

The method Invariant constructs an invariant \verb=I= for given Petri
net \verb=N= and \verb=P= when \verb=N= never violates \verb=P=.

\begin{verbatim}
* Code and Petri net: same as in section Method Safety.
\end{verbatim}

\begin{verbatim}
* Method Invariant

  Subprocedure \mathcal{C'} constructs dual state constraints C' corresponding to N and P.
  Subprocedure Model returns assignment A such that A satisfies C'.
  Subprocedure Inv constructs invariant I corresponding to N and A'.
\end{verbatim}

\ifthenelse{\equal{\isDraft}{true}}{\begin{figure}
  \begin{center}
    \begin{tikzpicture}[
      every path/.style={draw, ->, >=stealth, shorten >=2pt, shorten <=2pt}
      ]
      \node[state] (begin) {BEGIN};
      \node[action, below=of begin] (c) {$C':=\mathcal C'(N, \neg P)$};
      \node[decision, below=of c] (satc) {$\text{SAT}(C')$};
      \node[print, right=of satc] (noinv) {No Invariant};
      \node[action, below=of satc] (inv) {$A':=\text{Model}(C')$\\
        $I := \text{Inv}(N, A')$};
      \node[print, below=of inv] (printinv) {Invariant: $I$};
      \node[state, right=of noinv] (end1) {END};
      \node[state, below=of printinv] (end2) {END};
      
      \draw (begin) edge (c);
      \draw (c) edge (satc);
      \draw (satc) edge node[above]{NO} (noinv);
      \draw (satc) edge node[right]{YES} (inv);
      \draw (noinv) edge (end1);
      \draw (inv) edge (printinv);
      \draw (printinv) edge (end2);
    \end{tikzpicture}
  \end{center}
  \caption{Diagram for Method Invariant}
  \label{fig:method-invariant-diagram}
\end{figure}}{}

\begin{figure}
  \begin{algorithmic}[1]
    \State $C' := \mathcal C'(N, \neg P)$
    \If {SAT$(C')$}
    \State $A' := $Model$(C')$
    \State $I := $Inv$(N, A')$
    \State \Return 'Invariant I for the petri net: $I$'
    \Else
    \State \Return 'Failed at finding an invariant'
    \EndIf
  \end{algorithmic}
  \caption{Pseudocode for Method Invariant}
  \label{fig:method-invariant-pseudocode}
\end{figure}



\begin{verbatim}
* Property of dual state constraints C':  If C' is sat then N |= P.

* Property of invariant I:
  - I is reachable: For each reachable marking M, I(M) is valid
  - I is safe:      For markings that violate the property, I(M) is unsat
  - I is inductive: For each marking M, if I(M) is valid and M -> M1 then I(M1) is valid
\end{verbatim}

\newpage

\begin{verbatim}
* Subprocedure \mathcal{C'}:

  Input:
    N = (S, T, E, M0)  : Petri net
    \neg P = (p_1,1 + ... + p_1,m_1 >= b_1 /\ p_2,1 + ... + p_2,m_2 >= b_2 /\
              ... /\ p_n,1 + ... + p_n,m_n >= b_n ) : Negated property
  Output:
    C'            : Dual state constraints

Pseudocode:
\end{verbatim}

\begin{align*}
  C'(N, \neg P) =& \left( \bigwedge_{t \in T} \left( 0 \ge
                      \sum_{(t, s) \in E} s
                    - \sum_{(s, t) \in E} s \right) \right) \land
     \left( \sum_{s \in S} M_0(s) \cdot s <
       \sum_{i=1}^n b_i \cdot target_i \right) \land \\
     & \left( \bigwedge_{s \in S} \left ( s \ge 
       \sum_{i : s \in \{ p_{i,1}, \ldots, p_{i,m_i} \} } target_i \right) \right) \land
     \left( \bigwedge_{i=1}^n \left( target_i \ge 0 \right) \right)
\end{align*}

\begin{verbatim}
* Subprocedure Inv:

  Input:
    N = (S, T, E, M0) : Petri net
    A'                : Satisfying assignment for C'

  Output:
    I                 : Invariant

  Pseudocode:
\end{verbatim}

\begin{align*}
  I(N, A') =& \left( \sum_{s \in S} A'(s) \cdot s \le
                     \sum_{s \in S} A'(s) \cdot M_0(s) \right)
\end{align*}

\newpage

\begin{verbatim}
* Derivation of invariant I for Petri net N that satisfies property P:

  M                                    : marking          ~ places
  C                                    : incidence matrix ~ rows correspond to places, columns correspond to transitions, relates places to transitions
  X                                    : firing vector    ~ transitions

  The following constraints C1 are unsat.
  M = M0 + C*X                         : place equations
  M >= 0                               : non-negativity conditions for places
  X >= 0                               : non-negativity conditions for transitions
  AM >= b                              : property P negated

  Substitute M to obtain constraints C2.
  A(M0 + CX) >= b                      : property P negated
  M0 + CX >= 0                         : non-negativity conditions for places
  X >= 0                               : non-negativity conditions for transitions

  Rewrite each system to obtain constraints C3.
  (-A*C)      (A*M0-b)                 : property P negated
  (  -C)*X <= (  M0  )                 : non-negativity conditions for places
  (  -I)      (   0  )                 : non-negativity conditions for transitions

  Apply Farkas' Lemma to obtain constraints C4.
  yT*(-A*C)                            : 
     (  -C) = 0                        : 
     (  -I)                            : 

  yT*(A*M0-b)                          : 
     (  M0  ) < 0                      : 
     (   0  )                          : 

  y >= 0                               : 

  The constraints C4 are sat iff the following constraints C5 are sat.
  y1 * A * C + y2 * C + y3 = 0         : 
  y1 * (A*M - b) + y2 * M0 < 0         : 
  y1 >= 0                              : 
  y2 >= 0                              : 
  y3 >= 0                              : 

  The constraints C5 are sat iff the following constraints C6 are sat.
  (y1 * A + y2) * C  <= 0              : 
  (y1 * A + y2) * M0 < y1 * b          : 
  y1 >= 0                              : 
  y2 >= 0                              : 

  The constraints C6 are sat iff the following constraints C' are sat.
  \lambda * C  <= 0                    : inductivity constraint
  \lambda * M0 < y1 * b                : safety constraint
  \lambda >= y1 * A                    : property constraint
  y1 >= 0                              : non-negativity constraint

  For \lambda satisfying C' the invariant is the following.
  I(M) = (\lambda * M <= \lambda * M0) : invariant
\end{verbatim}

\newpage

\begin{verbatim}
* Example:
  
  - Using property 2

  - Dual state constraints C':

    0 >= - p1 + p2      + bit1 - nbit1
    0 >=      - p2 + p3
    0 >= + p1      - p3 - bit1 + nbit1

    0 >= - q1 + q2                - nbit2
    0 >=      - q2 + q3
    0 >=           - q3 + q4      + nbit2
    0 >= + q1           - q4
    0 >=      - q2           + q5
    0 >= + q1                - q5 + nbit2

    p1 + q1 + nbit1 + nbit2 < 2 * target_1

    p1    >= target_1
    p2    >= target_1
    p3    >= target_1
    bit1  >= 0
    nbit1 >= 0
    q1    >= 0
    q2    >= 0
    q3    >= 0
    q4    >= 0
    q5    >= 0
    nbit2 >= 0

    target_1 >= 0

  - Model A':

    p1       = 3
    p2       = 2
    p3       = 2
    bit1     = 1
    nbit1    = 0
    q1       = 0
    q2       = 0
    q3       = 0
    q4       = 0
    q5       = 0
    nbit2    = 0
    target_1 = 2
    
  - Invariant:

    3 * p1 + 2 * p2 + 2 * p3 + bit1 <= 3

\end{verbatim}

\newpage

\section{Method Invariant with Minimization}

The method Invariant with Minimization constructs an invariant \verb=I=
that uses a minimal number of places for given Petri net \verb=N= and
\verb=P= when \verb=N= never violates \verb=P=. 

\begin{verbatim}
* Code, property, and Petri net: same as in section Method Safety.

* Method Invariant with Minimization

  Subprocedure \text{MinConstraints} constructs minimization constraints C_M corresponding to N and A'
\end{verbatim}

\ifthenelse{\equal{\isDraft}{true}}{\begin{figure}
  \begin{center}
    \begin{tikzpicture}[
      every path/.style={draw, ->, >=stealth, shorten >=2pt, shorten <=2pt}
      ]
      \node[state] (begin) {BEGIN};
      \node[action, below=of begin] (c) {$C':=\mathcal C'(N, \neg P)$};
      \node[decision, below=of c] (satc) {$\text{SAT}(C')$};
      \node[print, right=of satc] (noinv) {No Invariant};
      \node[action, below=of satc] (model1) {$A':=\text{Model}(C')$};
      \node[action, below=of model1] (cm) {$C_M:=\text{MinConstraints}(N, A')$};
      \node[decision, below=of cm] (satcm) {$\text{SAT}(C' \cup C_M)$};
      \node[action, right=of satcm] (model2) {$A':=\text{Model}(C' \cup C_M)$};
      \node[action, below=of satcm] (inv) {$I := \text{Inv}(N, A')$};
      \node[print, below=of inv] (printinv) {Invariant: $I$};
      \node[state, right=of noinv] (end1) {END};
      \node[state, below=of printinv] (end2) {END};
      
      \draw (begin) edge (c);
      \draw (c) edge (satc);
      \draw (satc) edge node[above]{NO} (noinv);
      \draw (noinv) edge (end1);
      \draw (satc) edge node[right]{YES} (model1);
      \draw (model1) edge coordinate[pos=.5] (edgein) (cm);
      \draw (cm) edge (satcm);
      \draw (satcm) edge node[above]{YES} (model2);
      \draw (model2) |- (edgein);
      \draw (satcm) edge node[right]{NO} (inv);
      \draw (inv) edge (printinv);
      \draw (printinv) edge (end2);
    \end{tikzpicture}
  \end{center}
  \caption{Diagram for Method Invariant with Minimization}
  \label{fig:method-invariant-with-minimization-diagram}
\end{figure}
}{}

\begin{figure}
  \begin{algorithmic}[1]
    \State $C' := \mathcal C'(N, \neg P)$
    \If {SAT$(C')$}
    \State $A' := $Model$(C')$
    \State $C_M := $MinConstraints$(N, A')$
    \While {SAT$(C' \cup C_M)$}
    \State $A' := $Model$(C' \cup C_M)$
    \State $C_M := $MinConstraints$(N, A')$
    \EndWhile
    \State $I := $Inv$(N, A')$
    \State \Return 'Invariant I for the petri net: $I$'
    \Else
    \State \Return 'Failed at finding an invariant'
    \EndIf
  \end{algorithmic}
  \caption{Pseudocode for Method Invariant with Minimization}
  \label{fig:method-invariant-with-minimization-pseudocode}
\end{figure}



\begin{verbatim}
* Property of minimization constraints C_M generated from A': 
    If A'' satisfies C' u C_M, then Inv(N, A'') uses less places than Inv(N, A')
\end{verbatim}

\newpage

\begin{verbatim}
* Subprocedures \mathcal{C'} and Inv are the same as in Method Invariant

* Subprocedure \text{MinConstraints}:

  Input:
    N = (S, T, E, M0)  : Petri net
    A'                 : Satisfying assignment for C'
  Output:
    C_M                : Minimization Constraints

  Pseudocode:

\end{verbatim}

\begin{align*}
  C_M(N, A') =& \left( \bigwedge_{s \in S} \left(
      (s > 0 \Rightarrow b_s = 1) \land (s = 0 \Rightarrow b_s = 0)
    \right) \right) \land
    \left( \sum_{s \in S} b_s < \sum_{s \in S : A'(s) > 0} 1 \right)
\end{align*}

\newpage

\begin{verbatim}
* Example:

  - Dual state constraints C' and satisfying assignment A' for C' as in example for method Invariant with property 2

  - Minimization constraints C_M:

    p1     > 0 => b_p1   = 1
    |            -----------
    |               |
    |            place p1 appears in invariant
    |
    place p1 coefficient in the invariant

    p1     = 0 => b_p1   = 0
    |           ------------
    |               |
    |            place p1 does not appear in invariant
    |
    place p1 coefficient in the invariant

    p2    > 0 => b_p2    = 1
    p2    = 0 => b_p2    = 0
    p3    > 0 => b_p3    = 1
    p3    = 0 => b_p3    = 0
    bit1  > 0 => b_bit1  = 1
    bit1  = 0 => b_bit1  = 0
    nbit1 > 0 => b_nbit1 = 1
    nbit1 = 0 => b_nbit1 = 0

    q1    > 0 => b_q1    = 1
    q1    = 0 => b_q1    = 0
    q2    > 0 => b_q2    = 1
    q2    = 0 => b_q2    = 0
    q3    > 0 => b_q3    = 1
    q3    = 0 => b_q3    = 0
    q4    > 0 => b_q4    = 1
    q4    = 0 => b_q4    = 0
    q5    > 0 => b_q5    = 1
    q5    = 0 => b_q5    = 0
    nbit2 > 0 => b_nbit2 = 1
    nbit2 = 0 => b_nbit2 = 0

    b_p1 + b_p2 + b_p3 + b_bit1 + b_nbit1 + b_q1 + b_q2 + b_q3 + b_q4 + b_q5
         + b_nbit2 < 4
    --------------   |
                |    |
                |    number of places appearing in current invariant for A'
                |    = #{s | A'(s) > 0} = #{p1, p2, p3, bit1}
                |
  number of places appearing in new invariant
  
  - Model A' for C' \cup C_M:

    p1       = 1
    p2       = 1
    p3       = 1
    bit1     = 0
    nbit1    = 0
    q1       = 0
    q2       = 0
    q3       = 0
    q4       = 0
    q5       = 0
    nbit2    = 0
    target_1 = 1
    
    b_p1     = 1
    b_p2     = 1
    b_p3     = 1
    b_bit1   = 0
    b_nbit1  = 0
    b_q1     = 0
    b_q2     = 0
    b_q3     = 0
    b_q4     = 0
    b_q5     = 0
    b_nbit2  = 0
    
    
  - Minimized Invariant:

    p1 + p2 + p3 <= 1
\end{verbatim}
    
\newpage

\section{Method Invariant by Refinement}

\begin{verbatim}
* Code, property, and Petri net: same as in section Method Safety by Refinement.

* Method Invariant by Refinement
\end{verbatim}

\ifthenelse{\equal{\isDraft}{true}}{\begin{figure}
  \begin{tikzpicture}[
    every path/.style={draw, ->, >=stealth, shorten >=2pt, shorten <=2pt}
    ]
    \node[state] (begin) {BEGIN};
    \node[action, below=of begin] (c) {$C:=\mathcal C(N)$\\
      $D:=\{\}$};
    \node[decision, below=of c] (satc) {$\text{SAT}(C \cup \{\neg P\})$};
    \node[action, below=of satc] (modelc) {$A:=\text{Model}(C \cup \{\neg P\})$\\
      $C_{\theta}:=\text{TrapConditions}(N, A)$};
    \node[decision, below=of modelc] (satctheta) {$\text{SAT}(C_\theta)$};
    \node[action, below=of satctheta] (modelctheta) {$A_\theta:=\text{Model}(C_\theta)$\\
      $\delta:=\Delta(A_\theta)$\\
      $C:=C \cup \{\delta\}$\\
      $D:=D \cup \{\delta\}$};
    \node[print, right=of satctheta] (noinv2) {No Invariant};
    \node[state, below=of noinv2] (end3) {END};
    
    \node[action, right=of satc] (cprime) {$C':=\mathcal C'(N, \neg P \land D)$};
    \node[decision, right=of cprime] (satcprime) {$\text{SAT}(C')$};
    \node[action, below=of satcprime] (inv) {$A':=\text{Model}(C')$\\
      $I:=D \land \text{Inv}(N, A')$};
    \node[print, below=of inv] (printinv) {Invariant: $I$};
    \node[print, right=of satcprime] (noinv) {No Invariant};
    \node[state, below=of noinv] (end1) {END};
    \node[state, below=of printinv] (end2) {END};
    
    \draw (cprime) edge (satcprime);
    \draw (satcprime) edge node[above]{NO} (noinv);
    \draw (satcprime) edge node[right]{YES} (inv);
    \draw (inv) edge (printinv);
    \draw (noinv) edge (end1);
    \draw (printinv) edge (end2);
    
    \draw (begin) edge (c);
    \draw (c) edge coordinate[pos=.5] (edgein) (satc);
    \draw (satc) edge node[above]{NO} (cprime);
    \draw (satc) edge node[right]{YES} (modelc);
    \draw (modelc) edge (satctheta);
    \draw (satctheta) edge node[above]{NO} (noinv2);
    \draw (noinv2) edge (end3);
    \draw (satctheta) edge node[right]{YES} (modelctheta);
    \draw (modelctheta.south) -- ([yshift=-0.5cm] modelctheta.south)
    -| ([xshift=-2cm] modelctheta.west) |- (edgein);
  \end{tikzpicture}
  \caption{Diagram for Method Invariant by Refinement}
  \label{fig:method-invariant-by-refinement-diagram}
\end{figure}
}{}

\begin{figure}
  \begin{algorithmic}[1]
    \State $C := \mathcal C(N)$
    \State $D := $true
    \While {SAT$(C \cup \{\neg P\})$}
    \State $A := $Model$(C \cup \{\neg P\})$
    \State $C_\theta := $TrapConditions$(N, A)$
    \If {SAT$(C_\theta)$}
    \State $A_\theta := $Model$(C_\theta)$
    \State $\delta := \Delta(A_\theta)$
    \State $C := C \cup \{\delta\}$
    \State $D := D \land \delta$
    \Else
    \State \Return 'Failed at finding an invariant'
    \EndIf
    \EndWhile
    \State $C' := \mathcal C'(N, \neg P \land D)$
    \If {SAT$(C')$}
    \State $I := D \land $Inv$(N, A')$
    \State \Return 'Invariant I for the petri net: $I$'
    \Else
    \State \Return 'Failed at finding an invariant'
    \EndIf
  \end{algorithmic}
  \caption{Pseudocode for Method Invariant by Refinement}
  \label{fig:method-invariant-by-refinement-pseudocode}
\end{figure}



\begin{verbatim}
* Subprocedures TrapConditions and \Delta are the same as
  in Method Safety by Refinement

* Subprocedures \mathcal{C'} and Inv are the same as
  in Method Invariant
\end{verbatim}

\newpage

\begin{verbatim}
* Example:

  - Using property 1

  - Trap constraints as in example from section Method Safety by Refinement.
  
  - Dual state constraints C':

    0 >= - p1 + p2      + bit1 - nbit2
    0 >=      - p2 + p3
    0 >= + p1      - p3 - bit1 + nbit2

    0 >= - q1 + q2                - nbit2
    0 >=      - q2 + q3
    0 >=           - q3 + q4      + nbit2
    0 >= + q1           - q4
    0 >=      - q2           + q5
    0 >= + q1                - q5 + nbit2

    p1 + q1 + nbit1 + nbit2 < 2 * target_1 + trap_1

    p1    >= 0
    p2    >= trap_1
    p3    >= target_1
    bit1  >= 0
    nbit1 >= trap_1
    q1    >= 0
    q2    >= trap_1
    q3    >= trap_1
    q4    >= 0
    q5    >= target_1
    nbit2 >= trap_1

    target_1 >= 0
    trap_1   >= 0

  - Model A':

    p1       = 0
    p2       = 1
    p3       = 1
    bit1     = 0
    nbit1    = 1
    q1       = 0
    q2       = 1
    q3       = 1
    q4       = 0
    q5       = 1
    nbit2    = 1
    target_1 = 1
    trap_1   = 1
    
  - Invariant:

    p2        q2 + q3      + nbit1 + nbit2 >= 1
    p2 + p3 + q2 + q3 + q5 + nbit1 + nbit2 <= 2

\end{verbatim}

\newpage
\section{Experiments}

To evaluate the described methods, we implemented them in a tool called
\emph{Petrinizer}. Petrinizer is essentially a set of Bash and Prolog scripts
that mediate between the input, specified in the input format of
MIST2\footnote{\url{https://github.com/pierreganty/mist}}, and the SMT
solver Z3\footnote{\url{http://z3.codeplex.com/}} \cite{DeMouraTACAS08}. With
the choice of Bash and Prolog over languages with bindings for Z3 API, we
restricted Z3's full power. Indeed, by using Z3 as an external tool, we
were not able to fully exploit the incremental nature of its linear arithmetic
solver (TODO: cite appropriate papers). However, even the non-optimal
implementation turned out to be robust and scalable, with the main advantage of
being able to tackle problem instances that are out of reach for
state-of-the-art coverability checkers.

The evaluation of Petrinizer had three main goals. First, we wanted to compare
its performance against state-of-the-art tools like MIST2, BFC
and IIC (cite!). Second, as the method it
implements is incomplete, we wanted to measure its rate of success on safe problem instances. As a
subgoal, we wanted to investigate the usefulness and necessity of traps, as
well as to what extent real arithmetic suffices over integer arithmetic in
proving safety. And last, since the language of linear arithmetic, used by
Petrinizer, allows for a more succint representation of formulas than the
language used by IIC, we wanted to compare sizes of invariants the two tools produce.

Inputs that were used in tests come from various sources (TODO: write a
sentence to justify the variety).
One source is the collection of Petri nets from the literature that is part of the MIST2 toolkit.
The second source are Petri nets originating from the analysis of concurrent C
programs that were used in the evaluation of BFC \cite{KaiserCONCUR12}. Then
there are inputs originating from the provenance analysis of a medical and a
bug-tracking system \cite{MajumdarSAS13}. Finally, we generated a set of Petri
nets from Erlang programs, using an Erlang verification tool Soter
\cite{DOsualdoSAS13}. (TODO: Acknowledge Emanuele D'Osualdo for help.)

All experiments were performed on the identical machines, equipped with
Intel Xeon 2.66 GHz CPUs and 48 GB of memory, running Linux 3.2.48.1 in 64-bit
mode. Execution time was limited to 100,000 seconds (27 hours, 46 minutes and
40 seconds), and memory to 2 GB.

\subsection{Performance}

Explain what we observe:
\begin{itemize}
  \item On small examples IIC mostly outperforms Petrinizer. Probably due
    to a slower parser, setting up and invoking Z3.
  \item On examples of a similar size, coming from the same source,
    unlike with IIC, resource consumption is fairly uniform. (Medical examples.)
  \item Petrinizer is able to handle some huge examples that are out of
    scope for IIC.
  \item There were no cases when IIC was able to finish, but Petrinizer ran
    out of resources. As a matter of fact, Petrinizer finishes in all cases,
    whereas other tools somethimes fail.
\end{itemize}

\emph{Invariant sizes.} Again, explain how we measure it. For IIC, we measure
the number of non-null literals, the number of clauses, total length of
the invariant in characters. For Petrinizer, we measure the number of
non-null coefficients in the linear expressions and the total length of the
invariant in characters. Argue why it makes sense to compare any of these
quantities. Point out the difference in size with and without the invariant
minimization.
  
\emph{Rate of success.} Point out that Petrinizer mostly successfully proves
safety. Specifically, it proves safety in all of the safe Soter examples.
Point out the cases where traps were actually useful, and potentially progress
into some pseudo-philosophical discussion on what constitutes a good set of examples.


\section{Related work}

TODO
\section{Conclusions}

TODO

\bibliographystyle{abbrv}
\bibliography{references}

\end{document}

