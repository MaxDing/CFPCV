\section{A limitation and future work}
\label{sec:conclusion}

Compared to imperative programs, higher order functions impose
additional complication on the binary reachability analysis.
The major current limitation of our approach lies in the treatment of
partial applications when arguments are provided at different program
points, as illustrated by the following example. 
%
\begin{small}
\begin{verbatim}
let rec f x = if x > 0 then f (x - 1) else fun y -> y + 1 in
let g = f 1 in
g 2
\end{verbatim}
\end{small}
%
Proving termination of $\mathtt{f}/2$ applications is not possible,
since the second argument to \texttt{f} is given indirectly through an
application of~\texttt{g}. 
Without a control-flow analysis, we cannot store \texttt{2} in the
auxiliary state since it is not known syntactically that \texttt{2} is
the second argument for~\texttt{f}.
Removing the above limitation is an important step for future work,
which can be accomplished either by relying on results of a
control-flow analysis, e.g.~\cite{MightPOPL11}, that is performed
before the transformation takes place.
In simple cases, by using the following transformation that is based
on $\beta$ reduction.
%
\begin{small}
\begin{verbatim}
let rec f x = if x > 0 then f (x - 1) else fun y -> y + 1 in
(* let g = f 1 in *)
(f 1) 2
\end{verbatim}
\end{small}
%
In the transformed program, both arguments are given to \texttt{f}
directly.

% In the future we would like to integrate our binary reachability
% analysis in a counter example guided abstraction refinement loop,
% following the steps of~\cite{copytrick}.


% Our monitorability condition on expressions interferes with modularity
% in the following two ways. Consider a user program \expr\ with a free
% variable representing a library value, say \ttt{List.map~:~('a->'b)->'a~list->'b~list}. 
% \product\ can construct a product \exprprod\ for \expr. 
% First, \exprprod\ can only be used for runtime verification in the presence
% of a monadic value \ttt{List.map\_m} of type
%  \ttt{\monadic~('a->'b)->'a~list->'b~list}. Since \product\ can deal
%  with arbitrary higher-order function declarations, a solution is to
%  prepend \expr\ with the definition of \ttt{List.map}. Second, the
%  static reasoning about \exprprod\ can only be done under given
%  assumptions about \ttt{List.map}. A solution is to use tool-specific
%  codifications of assumptions on free variables. As shown in the
%  illustration section and our experiments, \dsolve\ provides such a mechanism.


% % \begin{remark}[TODO: Higher-order functions]
% %   \product\ can construct products for expressions featuring
% %   higher-order functions. Given a higher-order abstraction, \product\
% %   constructs a higher-order abstraction.  Higher-order abstractions
% %   given by \product\ can only be applied to product parameters. Thus,
% %   runtime verification of higher-order products can only be done in
% %   the presence of product parameters. Static verification of
% %   higher-order products can be done by making assumptions on given
% %   product parameters. \dsolve\ provides a ``value specifications'' for
% %   making these assumptions~\cite{Dsolve}.
% % \end{remark}

% We leave as future work the development of a user-friendly language
% specification based on temporal logics. 

% \paragraph{Future work}
% \begin{enumerate}
% \item User friendliness of monitor specifications.
% \item Cannot assign blame to original locations of the program neither
%   on runtime nor statically. \dsolve\ only provides 
% \end{enumerate}

%%% Local Variables: 
%%% mode: latex
%%% TeX-master: "main"
%%% End: 
