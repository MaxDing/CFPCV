\section{Examples}

In this section we show several functional programs that illustrate
complications that arise when proving termination.  We also highlight
how our implementation applies binary reachability analysis to address
such complications.

We write our examples in OCaml, and use \texttt{nondet ()} to model
non-deterministic input, which is useful for abstracting complex
expressions that are irrelevant for our reasoning process.

\paragraph{\bf Non-well-founded data}

The program \texttt{choice} is shown below.
%
\begin{center}
\begin{minipage}[h]{.5\linewidth}
\begin{small}
\begin{verbatim}
let rec choice (x, y) = 
  if x > 0 && y > 0 then
    if nondet () then 
      choice (x-1, x)
    else 
      choice (y-2, x+1)
  else 
    ()
in
choice (read_int (), read_int ())
\end{verbatim}
\end{small}
\end{minipage}
\end{center}
%
\texttt{choice} requires complex termination reasoning, since the else
branch may lead to the increase of the value that is passed as the
second component of the tuple. 
For this reason, for example, a size-change based method is no longer
applicable.

Using transition invariants, we can prove termination by showing that
between every pair of applications of \texttt{choice} either the value
of the first component of the parameter, the second component, or
their sum is decreasing.

\paragraph{\bf Partial applications}

We consider the following curried function \texttt{f} that has a type 
\texttt{int -> int -> int}.
%
\begin{center}
\begin{minipage}[h]{.8\linewidth}
\begin{small}
\begin{verbatim}
let rec f x = if x > 0 then f (x-1) else fun y -> f x y
\end{verbatim}
\end{small}
\end{minipage}
\end{center}
%
This function shows that -- in contrast to proving termination of
recursive procedures in imperative programs -- it is important to
differentiate between partial and complete applications when dealing
with curried functions.
First, we observe that any partial application of \texttt{f}
terminates.
For example, \texttt{f 10} stops after ten recursive calls and returns
a function \texttt{fun y -> f 10 y} where \texttt{f} is bound to a
closure.
That is, there is no infinite sequence of \texttt{f} applications that
are passed only one argument.
In contrast, any complete application of \texttt{f} does not terminate.
For example, \texttt{f 1 1} will lead to an infinite sequence of
\texttt{f} applications such that each of them is given two arguments.

Our binary reachability analysis takes as input a specification that
determines which kind of applications we want to keep track of. 
The specification consists of a function identifier, e.g., \texttt{f},
a number of parameters, e.g., one, and a transition invariant candidate. 
Then, such a specification requires that applications of \texttt{f} to
one argument satisfy the transition invariant candidate.
Alternatively, we may focus on applications of \texttt{f} to two
arguments.

When dealing with one-ary applications, the binary reachability
analysis can show that every pair of one-ary applications is included
in a transition invariant. 


Note that if complete applications of a function terminate, then every
partial application of the function terminates.
For example, the following function \texttt{g} does not have any
infinite application sequences neither for complete nor for partial
applications.
%
\begin{center}
\begin{minipage}[h]{.8\linewidth}
\begin{small}
\begin{verbatim}
let rec g x = if x > 0 then g (x-1) else fun y -> x+y
\end{verbatim}
\end{small}
\end{minipage}
\end{center}




\paragraph{\bf Functions as arguments}

The following example contains functions that take functions as
arguments.
%
\begin{center}
\begin{minipage}[h]{.8\linewidth}
\begin{small}
\begin{verbatim}
let rec map_skip (f, skip) = function
  | x :: xr -> f x :: map_skip (f, skip) (skip xr)
  | [] -> []
in
let add x y = x+y in
let tail = function
  | _ :: tl -> tl
  | [] -> []
in
map_skip (add 1, tail) [0;1]
\end{verbatim}
\end{small}
\end{minipage}
\end{center}
%
While such arguments may not participate in the expressions that
exhibit decent, it is important to keep track of their effect on the
arguments of the recursive calls.
For example, it is important that no application of \texttt{skip}
increases the length of its input list.

Our binary reachability analysis keep track of the above condition by
relying on a program transformation that augments functions with
additional inputs.
By storing arguments of a function application in such inputs, they
become available, i.e., they persist through applications of functions
that are passed as parameter, for the comparison with the transition
invariant when necessary. 

%%% Local Variables: 
%%% mode: latex
%%% TeX-master: "main"
%%% End: 
