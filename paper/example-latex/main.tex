% \documentclass[nocopyrightspace]{sigplanconf}
%\documentclass{llncs}
\documentclass{llncs}

\sloppy

\sloppy

\pagestyle{plain}

\newcommand{\isTechReport}{true}

\newcommand{\seeTechReport}{See~\cite{FunVTechReport}.}

\usepackage{float} % provides the [H] option for figures
\usepackage{amsmath,color}
\usepackage{amssymb}
\usepackage{xspace}
\usepackage{ar-alg}
\usepackage[inference]{semantic}
% \usepackage[usenames,dvipsnames]{xcolor}
\usepackage{mathpartir}
\usepackage{graphics}
% \usepackage[amsmath,thmmarks]{ntheorem}
\usepackage{definitions}
\usepackage{rotating}
\usepackage[nocompress]{cite}

\newcommand{\myexp}[1]{}

\newcommand{\setOf}[1]{\{ #1 \}}
\newcommand{\ltrue}{\mathit{true}}
\newcommand{\lfalse}{\mathit{false}}

\newcommand{\theFunction}{\ensuremath{f}}
\newcommand{\theTI}{\ensuremath{\mathit{TI}}}
\newcommand{\theArity}{\ensuremath{N}}

\newcommand{\todo}[1]{\textcolor{red}{\bf #1}}
% tuning

\newcommand{\ttt}[1]{\texttt{#1}}
\newcommand{\tsf}[1]{\textsf{#1}}
\newcommand{\trm}[1]{\textrm{#1}}
\newcommand{\tbf}[1]{\textbf{#1}}

% \newcommand{\gray}[1]{\textcolor{Gray}{#1}} % not used

% misc

\newcommand{\versimple}{}
% \newcommand{\versimple}{simple version of\ }

\newcommand{\dsolve}{\tsf{Dsolve}}
\newcommand{\ocaml}{OCaml}
\newcommand{\miniocaml}{\mbox{Mini-OCaml}\xspace}
\newcommand{\camlp}{\tsf{Camlp4}}
\newcommand{\haskell}{Haskell}

\newcommand{\nats}{\ensuremath{\mathbb{N}}}
\newcommand{\ints}{\ensuremath{\mathbb{Z}}}

% illustration section

\newcommand{\simple}{\ttt{iter\_down}}
\newcommand{\bodysimple}{\expr[\ttt{id}]}
\newcommand{\sttranssimple}{\ttt{mtrans\_id}}
\newcommand{\simplem}{\ttt{simple\_m}}

\newcommand{\foldleft}{\ttt{fold\_left}}
\newcommand{\bodyfl}{\expr[\ttt{fold\_left}]}
\newcommand{\sttransfl}{\ttt{mtrans\_fl}}
\newcommand{\foldleftm}{\ttt{fold\_left\_m}}

\newcommand{\ack}{\ttt{ack}}
\newcommand{\bodyack}{\expr[\ttt{ack}]}
\newcommand{\sttransack}{\ttt{mtrans\_ack}}
\newcommand{\ackm}{\ttt{ack\_m}}

\newcommand{\mccarthy}{\ttt{mccarthy}}
\newcommand{\bodymc}{\expr[\ttt{mccarthy91}]}
\newcommand{\sttransmc}{\ttt{mtrans\_mc}}
\newcommand{\mccarthym}{\ttt{mccarthy\_m}}

% syntax

\newcommand{\vars}{\ensuremath{X}}
\newcommand{\csts}{\ensuremath{C}}
\newcommand{\cstrs}{\ensuremath{\vec{C}}}

\newcommand{\freevars}{\tsf{FreeVars}}

\newcommand{\patts}{\ensuremath{P}}
\newcommand{\exprs}{\ensuremath{E}}

\newcommand{\patt}[1][]{\ensuremath{p_{#1}}}
\newcommand{\expr}[1][]{\ensuremath{e_{#1}}}
\newcommand{\cst}{\ensuremath{c}}
\newcommand{\vst}[1][]{\ensuremath{s_{#1}}}
\newcommand{\var}[1][]{\ensuremath{x_{#1}}}
\newcommand{\cstr}[1][]{\ensuremath{\vec{c}_{#1}}}

% semantics

\newcommand{\vals}{\ensuremath{V}}
\newcommand{\evjs}{\ensuremath{J}}

\newcommand{\val}[1][]{\ensuremath{v_{#1}}}
\newcommand{\evjvar}[1][]{\ensuremath{j_{#1}}}
\newcommand{\evt}[1][]{\ensuremath{\mathfrak{t}_{#1}}}
\newcommand{\evtP}[1][]{\ensuremath{\mathfrak{t}'_{#1}}}

\newcommand{\patternmatch}{\textsf{Bindings}}
\newcommand{\dom}{\ensuremath{\mathit{Dom}}}

\newcommand{\ectx}[1][]{\ensuremath{\mathcal{E}_{#1}}}
\newcommand{\emptyectx}{\ensuremath{\emptyset}}
\newcommand{\eext}{\ensuremath{+}}
\newcommand{\emapsto}{\ensuremath{\mapsto}}

\newcommand{\evj}[3]{\ensuremath{#1~\vdash~#2~\Rightarrow~#3}}

\newcommand{\scst}{(Cst)}
\newcommand{\svar}{(Var)}
\newcommand{\svarerr}{(Var Error)}
\newcommand{\stup}{(Tuple)}
\newcommand{\stuperr}{(Tuple Error)}
\newcommand{\stupexn}{(Tuple Exn)}
\newcommand{\scstr}{(Constr)}
\newcommand{\scstrerr}{(Constr Error)}
\newcommand{\sapp}{(App)}
\newcommand{\sapprec}{(App Rec)}
\newcommand{\sappcst}{(App Cst)}
\newcommand{\sapperrfun}{(App Error)}
\newcommand{\sapperrparam}{(App Param Error)}
\newcommand{\sfun}{(Fun)}
\newcommand{\sletrec}{(Let Rec)}
\newcommand{\slet}{(Let)}
\newcommand{\sleterrmatch}{(Let Error Match)}
\newcommand{\sleterrbound}{(Let Error Bound)}
\newcommand{\sasserttrue}{(Assert True)}
\newcommand{\sassertfalse}{(Assert False)}
\newcommand{\sasserterr}{(Assert Error)}
\newcommand{\siftrue}{(If True)}
\newcommand{\siffalse}{(If False)}
\newcommand{\siferr}{(If Cond Error)}
\newcommand{\smatch}{(Match)}
\newcommand{\smatcherr}{(Match Error)}

%%%\newcommand{\ercst}{
  \inference[{\tiny \scst}]{
  }{
    \evj{\ectx}{\cst}{\cst}
  }
}

\newcommand{\ervar}{
  \inference[{\tiny \svar}]{
    \var \in \dom\ \ectx
  }{
    \evj{\ectx}{\var}{\ectx\ \var}
  }
}

\newcommand{\ertup}{
  \inference[{\tiny \stup}]{
    \evj{\ectx}{\expr[n]}{\val[n]}
    &
    \ldots
    &
    \evj{\ectx}{\expr[1]}{\val[1]}
  }{
    \evj{\ectx}{\ntupleexpr}{\ntupleval}
  }
}

\newcommand{\ercstr}{
  \inference[{\tiny \scstr}]{
    \evj{\ectx}{\expr[n]}{\val[n]}
    &
    \ldots
    &
    \evj{\ectx}{\expr[1]}{\val[1]}
  }{
    \evj{\ectx}{\ttt{\cstr(\ntupleexpr)}}{\cstr(\ntupleval)}
  }
}

\newcommand{\erappcst}{
  \inference[{\tiny \sappcst}]{
    \evj{\ectx}{\expr[p]}{\val[p]}
    &
    \evj{\ectx}{\expr[f]}{\cst}
  }{
    \evj{\ectx}{\appexpr}{\cst\ \val[p]}
  }
}

\newcommand{\erapp}{
  \inference[{\tiny \sapp}]{
    \evj{\ectx}{\expr[p]}{\val[p]}
    &
    \evj{\ectx}{\expr[f]}{\closure[b]}
    \\
    \evj{\ectx[b] \eext \var \emapsto \val[p]}{\expr[b]}{\val}
  }{
    \evj{\ectx}{\appexpr}{\val}
  }
}

\newcommand{\erapprec}{
  \inference[{\tiny \sapprec}]{
    \evj{\ectx}{\expr[p]}{\val[p]}
    &
    \evj{\ectx}{\expr[f]}{\recclosure[b]}
    \\
    \evj{\ectx[b] \eext \var[f] \emapsto \recclosure[b] \eext \var
      \emapsto \val[p]}{\expr[b]}{\val}
  }{
    \evj{\ectx}{\appexpr}{\val}
  }
}

\newcommand{\erlet}{
  \inference[{\tiny \slet}]{
    \evj{\ectx}{\expr[1]}{\val[1]}
    &
    \evj
    { \ectx \eext \patternmatch~\patt~\val[1] }
    {\expr[2]}
    {\val}
  }{
    \evj
    {\ectx}
    {\ttt{let \patt\ = \expr[1]\ in \expr[2]}}
    {\val}
  }
}

\newcommand{\erfun}{
  \inference[{\tiny \sfun}]{
  }{
    \evj
    {\ectx}
    {\funexpr}
    {\closure}
  }
}

\newcommand{\erletrec}{
  \inference[{\tiny \sletrec}]{
    \evj
    { \ectx \eext \var[f] \emapsto \recclosure}
    {\expr[2]}
    {\val}
  }{
    \evj
    {\ectx}
    {\ensuremath{
        \ttt{let rec \var[f] = \funexpr\ in \expr[2]}
      }}
    {\val}
  }
}

\newcommand{\ermatch}{
  \inference[{\tiny \smatch}]{
    \evj{\ectx}{\expr[m]}{\val[m]}
    \\
    \trm{\ncstrpatt{k} is the first pattern to match \val[m]}
    \\
    \evj
    { \ectx \eext \patternmatch~(\ncstrpatt{k})~\val[m] }
    {\expr[k]}
    {\val}
  }{
    \evj
    {\ectx}
    {\ensuremath{
        \left(
        \begin{array}{l}
          \ttt{match \expr[m]\ with} \\
          \ttt{\tabT | \ncstrpatt{1}\ -> \expr[1]} \\
          \ttt{\tabT \ldots} \\
          \ttt{\tabT | \ncstrpatt{i}\ -> \expr[i]}
        \end{array}
        \right)
      }}
    {\val}
  }
}

%%% Local Variables:
%%% mode: latex
%%% TeX-master: "main"
%%% End: 

\newcommand{\ercst}{
  \inference[]{
  }{
    \evj{\ectx}{\cst}{\cst}
  }
}

\newcommand{\ervar}{
  \inference[]{
    \var \in \dom\ \ectx
  }{
    \evj{\ectx}{\var}{\ectx\ \var}
  }
}

\newcommand{\ertup}{
  \inference[]{
    \evj{\ectx}{\expr[n]}{\val[n]}
    &
    \ldots
    &
    \evj{\ectx}{\expr[1]}{\val[1]}
  }{
    \evj{\ectx}{\ntupleexpr}{\ntupleval}
  }
}

\newcommand{\ercstr}{
  \inference[]{
    \evj{\ectx}{\expr[n]}{\val[n]}
    &
    \ldots
    &
    \evj{\ectx}{\expr[1]}{\val[1]}
  }{
    \evj{\ectx}{\ttt{\cstr(\ntupleexpr)}}{\cstr(\ntupleval)}
  }
}

\newcommand{\erappcst}{
  \inference[]{
    \evj{\ectx}{\expr[p]}{\val[p]}
    &
    \evj{\ectx}{\expr[f]}{\cst}
  }{
    \evj{\ectx}{\appexpr}{\cst\ \val[p]}
  }
}

\newcommand{\erapp}{
  \inference[]{
    \evj{\ectx}{\expr[p]}{\val[p]}
    &
    \evj{\ectx}{\expr[f]}{\closure[b]}
    \\
    \evj{\ectx[b] \eext \var \emapsto \val[p]}{\expr[b]}{\val}
  }{
    \evj{\ectx}{\appexpr}{\val}
  }
}

\newcommand{\erapprec}{
  \inference[]{
    \evj{\ectx}{\expr[p]}{\val[p]}
    &
    \evj{\ectx}{\expr[f]}{\recclosure[b]}
    \\
    \evj{\ectx[b] \eext \var[f] \emapsto \recclosure[b] \eext \var
      \emapsto \val[p]}{\expr[b]}{\val}
  }{
    \evj{\ectx}{\appexpr}{\val}
  }
}

\newcommand{\erlet}{
  \inference[]{
    \evj{\ectx}{\expr[1]}{\val[1]}
    &
    \evj
    { \ectx \eext \patternmatch~\patt~\val[1] }
    {\expr[2]}
    {\val}
  }{
    \evj
    {\ectx}
    {\ttt{let \patt\ = \expr[1]\ in \expr[2]}}
    {\val}
  }
}

\newcommand{\erfun}{
  \inference[]{
  }{
    \evj
    {\ectx}
    {\funexpr}
    {\closure}
  }
}

\newcommand{\erletrec}{
  \inference[]{
    \evj
    { \ectx \eext \var[f] \emapsto \recclosure}
    {\expr[2]}
    {\val}
  }{
    \evj
    {\ectx}
    {\ensuremath{
        \ttt{let rec \var[f] = \funexpr\ in \expr[2]}
      }}
    {\val}
  }
}

\newcommand{\ermatch}{
  \inference[]{
    \evj{\ectx}{\expr[m]}{\val[m]}
    \\
    \trm{\ncstrpattDots{k} is the first pattern to match \val[m]}
    \\
    \evj
    { \ectx \eext \patternmatch~(\ncstrpattDots{k})~\val[m] }
    {\expr[k]}
    {\val}
  }{
    \evj
    {\ectx}
    {\ensuremath{
        \left(
        \begin{array}{l}
          \ttt{match \expr[m]\ with} %\\
%          \ttt{\tabT | \ncstrpattDots{1}\ -> \expr[1]} \\
          \ttt{\ | \ldots} %\\
%          \ttt{\tabT | \ncstrpattDots{i}\ -> \expr[i]}
        \end{array}
        \right)
      }}
    {\val}
  }
}

%%% Local Variables:
%%% mode: latex
%%% TeX-master: "main"
%%% End: 


% types

\newcommand{\basetypes}{\ensuremath{B}}
\newcommand{\typevars}{\ensuremath{A}}
\newcommand{\typecstrs}{\ensuremath{\vec{B}}}

\newcommand{\type}[1][]{\ensuremath{\tau_{#1}}}
\newcommand{\tybase}[1][]{\ensuremath{\iota_{#1}}}
\newcommand{\tyvar}{\ensuremath{\alpha}}
\newcommand{\tycstr}{\ensuremath{\vec{\iota}}}

\newcommand{\arity}{\tsf{Arity}}

% frequently used expressions and values

\newcommand{\ntupleexpr}{\ttt{\expr[1],\ \ldots,\ \expr[n]}}
\newcommand{\appexpr}{\expr[f]~\expr[p]}
\newcommand{\funexpr}{\ttt{fun~\var~->~\expr[b]}}

\newcommand{\ntupleval}{\ensuremath{\val[1],\ \ldots,\ \val[n]}}
\newcommand{\closure}[1][]{
  \ensuremath{(\funexpr,\ \ectx[#1])}
}
\newcommand{\recclosure}[1][]{
  \ensuremath{(\var[f], \funexpr,\ \ectx[#1])}
}

\newcommand{\vstup}[1][]{\ensuremath{s_{#1}^{\uparrow}}}
\newcommand{\vstdown}[1][]{\ensuremath{s_{#1}^{\downarrow}}}
\newcommand{\varapp}{\var[\mathit{app}]}

\newcommand{\ncstrpatt}[1]{\ttt{\cstr[#1](\var[\ensuremath{#1,1}],\ \ldots,\ \var[\ensuremath{#1,j_{#1}}])}}


\newcommand{\ncstrpattDots}[1]{\ttt{\cstr[#1](\ensuremath{\ldots_{#1}})}}

% monitors and monitored evaluation trees

\newcommand{\mon}{\ensuremath{\mathcal{M}}}
\newcommand{\states}{\ensuremath{\Sigma}}
\newcommand{\st}[1][]{\ensuremath{\sigma_{#1}}}
\newcommand{\trans}{\ensuremath{\rho}}

\newcommand{\monplus}{\ensuremath{\mon_{\ttt{+}}}}
\newcommand{\transplus}{\ensuremath{\trans_{\ttt{+}}}}

\newcommand{\monps}{\ensuremath{\mon_{\ttt{+}/\ttt{-}}}}
\newcommand{\transps}{\ensuremath{\trans_{\ttt{+}/\ttt{-}}}}

\newcommand{\rt}{\evjvar}

\newcommand{\mevjs}{\ensuremath{J^{m}}}
\newcommand{\mevjvar}[1][]{\ensuremath{j^{m}_{#1}}}
%%%\newcommand{\mevj}[3]{\ensuremath{#2,\ #1,\ #3}}
\newcommand{\mevj}[3]{\ensuremath{#1,\ #3}}
\newcommand{\mevt}[1][]{\ensuremath{\mathfrak{t}^{m}_{#1}}}

\newcommand{\stupa}[1][]{\ensuremath{\st[#1]^{\uparrow}}}
\newcommand{\stdown}[1][]{\ensuremath{\st[#1]^{\downarrow}}}

\newcommand{\monitor}{\tsf{Augment}\xspace}

\newcommand{\smstepcst}{
    \inference[\scst]{
    }{
      \mevj{\evj{\ectx}{\cst}{\cst}}{\st}{\st'}
    }
}

\newcommand{\smstepvar}{
    \inference[\svar]{
    }{
      \mevj{\evj{\ectx}{\var}{\ectx\ \var}}{\st}{\st'}
    }
}

\newcommand{\smstepappcst}{
  \inference[\sappcst]{
    \mevj{\evj{\ectx}{\expr[p]}{\val[p]}}{\st}{\st'}
    &
    \mevj{\evj{\ectx}{\expr[f]}{\cst}}{\st'}{\st''}
  }{
    \mevj{\evj{\ectx}{\appexpr}{\cst\ \val[p]}}{\st}{\st''}
  }
}

% \newcommand{\smstepapp}{
%   \inference[\sapp]{
%     \mevj{\evj{\ectx}{\expr[p]}{\val[p]}}{\st}{\st'}
%     &
%     \mevj{\evj{\ectx}{\expr[f]}{\closure[b]}}{\st'}{\st''}
%     &
%     \mevj{\evj{\ectx[b] \eext \var \emapsto \val[p]}{\expr[b]}{\val}}{\st''}{\st'''}
%   }{
%     \mevj{\evj{\ectx}{\appexpr}{\val}}{\st}{\st'''}
%   }
% }

% \newcommand{\smsteplet}{
%   \inference[\slet]{
%     \mevj{\evj{\ectx}{\expr[1]}{\val[1]}}{\st}{\st'}
%     &
%     \mevj
%     {
%       \evj
%       { \ectx \eext \patternmatch~\patt~\val[1] }
%       {\expr[2]}
%       {\val}
%     }{
%       \st'
%     }{
%       \st''
%     }
%   }{
%     \mevj
%     {
%       \evj
%       {\ectx}
%       {\ensuremath{
%           \ttt{let \patt\ = \expr[1]\ in \expr[2]}
%        }}
%       {\val}
%     }{
%       \st
%     }{
%       \st''
%     }
%   }
% }

%%% Local Variables:
%%% mode: latex
%%% TeX-master: "main"
%%% End: 

% \newcommand{\functx}{\ectx[\mathit{fun}]}
% \newcommand{\rcl}{\trm{(\var[f],
% \exprFun \var[p] \exprArrow \ensuremath{\expr[b],}
% \functx)}}

\newcommand{\mstepcst}{
  \inference[\scst]{
  }{
    \mevj{\evj{\ectx}{\cst}{\cst}}{\stupa}{\stdown}
  }
}

\newcommand{\mstepvar}{
  \inference[\svar]{
  }{
    \mevj{\evj{\ectx}{\var}{\ectx\ \var}}{\stupa}{\stdown}
  }
}

\newcommand{\msteptup}{
  \inference[\stup]{
    \mevj{\evj{\ectx}{\expr[n]}{\val[n]}}{\stupa[n]}{\stdown[n]}
    &
    \ldots
    &
    \mevj{\evj{\ectx}{\expr[1]}{\val[1]}}{\stupa[1]}{\stdown[1]}
  }{
    \mevj{\evj{\ectx}{\ntupleexpr}{\ntupleval}}{\stupa}{\stdown}
  }
}

\newcommand{\mstepcstr}{
  \inference[\scstr]{
    \mevj{\evj{\ectx}{\expr[n]}{\val[n]}}{\stupa[n]}{\stdown[n]}
    &
    \ldots
    &
    \mevj{\evj{\ectx}{\expr[1]}{\val[1]}}{\stupa[1]}{\stdown[1]}
  }{
    \mevj{\evj{\ectx}{\cstr(\ntupleexpr)}{\cstr(\ntupleval)}}{\stupa}{\stdown}
  }
}

\newcommand{\mstepappcst}{
  \inference[\sappcst]{
    \mevj{\evj{\ectx}{\expr[p]}{\val[p]}}{\stupa[p]}{\stdown[p]}
    &
    \mevj{\evj{\ectx}{\expr[f]}{\cst}}{\stupa[f]}{\stdown[f]}
  }{
    \mevj{\evj{\ectx}{\appexpr}{\cst\ \val[p]}}{\stupa}{\stdown}
  }
}

\newcommand{\mstepapp}{
  \inference[\sapp]{
    \mevj{\evj{\ectx}{\expr[p]}{\val[p]}}{\stupa[p]}{\stdown[p]}
    &
    \mevj{\evj{\ectx}{\expr[f]}{\closure[b]}}{\stupa[f]}{\stdown[f]}
    &
    \mevj{\evj{\ectx[b] \eext \var \emapsto \val[p]}{\expr[b]}{\val}}{\stupa[b]}{\stdown[b]}
  }{
    \mevj{\evj{\ectx}{\appexpr}{\val}}{\stupa}{\stdown}
  }
}

\newcommand{\msteplet}{
  \inference[\slet]{
    \mevj{\evj{\ectx}{\expr[1]}{\val[1]}}{\stupa[1]}{\stdown[1]}
    &
    \mevj
    {
      \evj
      { \ectx \eext \patternmatch~\patt~\val[1] }
      {\expr[2]}
      {\val}
    }{
      \stupa[2]
    }{
      \stdown[2]
    }
  }{
    \mevj
    {
      \evj
      {\ectx}
      {\ensuremath{
          \underbrace{
            \ttt{let \patt\ = \expr[1]\ in \expr[2]}
          }_{\expr}
        }}
      {\val}
    }{
      \stupa
    }{
      \stdown
    }
  }
}

\newcommand{\mstepfun}{
  \inference[\sfun]{
  }{
    \mevj
    {
      \evj
      {\ectx}
      {\funexpr}
      {\closure}
    }{
      \stupa
    }{
      \stdown
    }
  }
}

\newcommand{\mstepletrec}{
  \inference[\sletrec]{
    \mevj
    {
      \evj
      { \ectx \eext \var[f] \emapsto \recclosure}
      {\expr[2]}
      {\val}
    }{
      \stupa[2]
    }{
      \stdown[2]
    }
  }{
    \mevj
    {
      \evj
      {\ectx}
      {\ensuremath{
          \underbrace{
            \ttt{let rec \var[f] = \funexpr\ in \expr[2]}
          }_{\expr}
        }}
      {\val}
    }{
      \stupa
    }{
      \stdown
    }
  }
}

\newcommand{\mstepmatch}{
  \inference[\smatch]{
    \mevj{\evj{\ectx}{\expr[m]}{\val[m]}}{\stupa[m]}{\stdown[m]}
    &
    \mevj
    {
      \evj
      { \ectx \eext \patternmatch~(\ncstrpatt{k})~\val[m] }
      {\expr[k]}
      {\val}
    }{
      \stupa[k]
    }{
      \stdown[k]
    }
  }{
    \mevj
    {
      \evj
      {\ectx}
      {\ensuremath{
          \underbrace{
            \left(
              \begin{array}{l}
                \ttt{match \expr[m]\ with} \\
                \ttt{\tabT | \ncstrpatt{1}\ -> \expr[1]} \\
                \ttt{\tabT \ldots} \\
                \ttt{\tabT | \ncstrpatt{i}\ -> \expr[i]}
              \end{array}
            \right)
          }_{\expr}
        }}
      {\val}
    }{
      \stupa
    }{
      \stdown
    }
  }
}

%%% Local Variables:
%%% mode: latex
%%% TeX-master: "main"
%%% End: 


% monitor specifications

\newcommand{\stransformers}{\ensuremath{T}}

\newcommand{\ms}[1][]{\ensuremath{\tsf{Spec}_{#1}}}
\newcommand{\tsel}{\tsf{transsel}}
\newcommand{\tiCheck}{\tsf{trans}}
\newcommand{\tselEnter}{\tsf{enter}}
\newcommand{\tselExit}{\tsf{exit}}


\newcommand{\monfun}{\ttt{f}} % the monitored function
\newcommand{\sttrans}{\ensuremath{mtrans}} % state transformer

\newcommand{\sem}[1]{\ensuremath{[|#1|]}}

\newcommand{\msplus}{\ensuremath{\ms[\ttt{+}]}}
\newcommand{\tselplus}{\ensuremath{\tsel_{\ttt{+}}}}

\newcommand{\msid}{\ensuremath{\ms[\mathit{id}]}}
\newcommand{\tselid}{\ensuremath{\tsel_{\mathit{id}}}}

% product construction

\newcommand{\moncstr}{\ttt{m}}

\newcommand{\monadic}{\tsf{monadic}\xspace}
\newcommand{\premonadic}{\tsf{pre\_m}}

\newcommand{\freshvars}{\tsf{FreshVars}}
\newcommand{\freshvar}{\tsf{FreshVar ()}}

\newcommand{\quo}[1]{\tsf{``}#1\tsf{''}}
\newcommand{\aq}[1]{`#1`}
\newcommand{\depremon}{\tsf{de\_premonadize}}
\newcommand{\product}{\tsf{Transform}\xspace}

\newcommand{\scstprod}{
  \begin{minipage}{1.0\linewidth}
    \ttt{update \aq{\tsel\ \cst} >>= fun () -> \\
      unit (fun \var[1] -> \\
      \tabT \ldots \\
      \tabTT unit (fun \var[n] -> \\
      \tabTTT unit (\cst\ \var[1] \ldots\ \var[n])) \ldots\ )
    }
  \end{minipage}
}

\newcommand{\svarprod}{
  \begin{minipage}{1.0\linewidth}
    \ttt{update \aq{\tsel\ \var} >>= fun () -> \\
      unit \var
    }
  \end{minipage}
}

\newcommand{\sappprod}{
  \begin{minipage}{1.0\linewidth}
    \ttt{\aq{\product\ \expr[p]} >>= fun \var[p] -> \\
      \aq{\product\ \expr[f]} >>= fun \var[f] -> \\
      \var[f] \var[p]
    }
  \end{minipage}
}

% \newcommand{\sletprod}{
%   \begin{minipage}{1.0\linewidth}
%     \ttt{(fun \vst\ -> \\
%       \tabT let \patt, $\vst[1]'$\ = \aq{\product\ \expr[1]} \vst\ in \\
%       \tabT \aq{\product\ \expr[2]} $\vst[1]'$)
%     }
%   \end{minipage}
% }

%%% Local Variables:
%%% mode: latex
%%% TeX-master: "main"
%%% End: 

\newcommand{\cstprod}{
  \begin{minipage}{1.0\linewidth}
    \ttt{update \aq{\tsel\ \cst\ $\uparrow$} >>= fun () -> \\
      update \aq{\tsel\ \cst\ $\downarrow$} >>= fun () -> \\
      unit (fun \var[1] -> \\
      \tabT \ldots \\
      \tabTT unit (fun \var[n] -> \\
      \tabTTT unit (\cst\ \var[1] \ldots\ \var[n])) \ldots\ )
    }
  \end{minipage}
}

\newcommand{\varprod}{
  \begin{minipage}{1.0\linewidth}
    \ttt{update \aq{\tsel\ \var\ $\uparrow$} >>= fun () -> \\
      update \aq{\tsel\ \var\ $\downarrow$} >>= fun () -> \\
      unit \var
    }
  \end{minipage}
}

\newcommand{\tupleprod}{
  \begin{minipage}{1.0\linewidth}
    \ttt{update \aq{\tsel\ $($\ntupleexpr$)$ $\uparrow$} >>= fun () ->\\
      \aq{\product\ \expr[n]} >>= fun \var[n] -> \\
      \ldots \\
      \aq{\product\ \expr[1]} >>= fun \var[1] -> \\
      update \aq{\tsel\ $($\ntupleexpr$)$ $\downarrow$} >>= fun () -> \\
      unit (\var[1], \ldots, \var[n])
    }
  \end{minipage}
}

\newcommand{\cstrprod}{
  \begin{minipage}{1.0\linewidth}
    \ttt{update \aq{\tsel\ $($\cstr(\ntupleexpr)$)$ $\uparrow$} >>= fun () ->\\
      \aq{\product\ \expr[n]} >>= fun \var[n] -> \\
      \ldots \\
      \aq{\product\ \expr[1]} >>= fun \var[1] -> \\
      update \aq{\tsel\ $($\cstr(\ntupleexpr)$)$ $\downarrow$} >>= fun () -> \\
      unit (\cstr (\var[1], \ldots, \var[n]))
    }
  \end{minipage}
}

\newcommand{\appcstprod}{
  \begin{minipage}{1.0\linewidth}
    \ttt{update \aq{\tsel\ $($\appexpr$)$ $\uparrow$} >>= fun () ->\\
      \aq{\product\ \expr[p]} >>= fun \var[p] -> \\
      \aq{\product\ \expr[f]} >>= fun \var[f] -> \\
      \var[f] \var[p] >>= fun \var[{\cst\,\val[p]}] -> \\
      update \aq{\tsel\ $($\appexpr$)$ $\downarrow$} >>= fun () -> \\
      unit \var[{\cst\,\val[p]}]
    }
  \end{minipage}
}

\newcommand{\appprod}{
  \begin{minipage}{1.0\linewidth}
    \ttt{update \aq{\tsel\ $($\appexpr$)$ $\uparrow$} >>= fun () ->\\
      \aq{\product\ \expr[p]} >>= fun \var[p] -> \\
      \aq{\product\ \expr[f]} >>= fun \var[f] -> \\
      \var[f] \var[p] >>= fun \varapp -> \\
      update \aq{\tsel\ $($\appexpr$)$ $\downarrow$} >>= fun () -> \\
      unit \varapp
    }
  \end{minipage}
}

\newcommand{\funprod}{
  \begin{minipage}{1.0\linewidth}
    \ttt{update \aq{\tsel\ $($\funexpr$)$ $\uparrow$} >>= fun () ->\\
      update \aq{\tsel\ $($\funexpr$)$ $\downarrow$} >>= fun () -> \\
      unit (fun \var\ -> \aq{\product\ \expr[b]})
    }
  \end{minipage}
}

\newcommand{\letprod}{
  \begin{minipage}{1.0\linewidth}
    \ttt{update \aq{\tsel\ \expr\ $\uparrow$} >>= fun () -> \\
      (\aq{\product\ \expr[1]}\hspace{1.3ex} >>= fun \patt\ -> \\
      \tabT\hspace{-2ex}  \aq{\product\ \expr[2]}) >>= fun \var[v] -> \\
      update \aq{\tsel\ \expr\ $\downarrow$} >>= fun () -> \\
      unit \var[v]
      % update \aq{\tsel\ \expr\ $\uparrow$} >>= fun () ->\\
      % (fun \vstup -> \\
      % \tabT let \patt, \vstdown[1]\ = \aq{\product\ \expr[1]} \vstup\ in \\
      % \tabT \aq{\product\ \expr[2]} \vstdown[1]) >>= fun \var[v] -> \\
      % update \aq{\tsel\ \expr\ $\downarrow$} >>= fun () -> \\
      % unit \var[v]
    }
  \end{minipage}
}

\newcommand{\letrecprod}{
  \begin{minipage}{1.0\linewidth}
    \ttt{update \aq{\tsel\ \expr\ $\uparrow$} >>= fun () ->\\
      (let rec \var[f] = fun \var\ -> \aq{\product\ \expr[b]} \\
      \tabT\hspace{-2ex} in \aq{\product\ \expr[2]}) >>= fun \var[v]\ -> \\
      update \aq{\tsel\ \expr\ $\downarrow$} >>= fun () -> \\
      unit \var[v]
    }
  \end{minipage}
}

\newcommand{\matchprod}{
  \begin{minipage}{1.0\linewidth}
    \ttt{update \aq{\tsel\ \expr\ $\uparrow$} >>= fun () -> \\
      \aq{\product\ \expr[m]} >>= fun \var[m]\ -> \\
      (match \var[m] with \\
      \tabT | \ncstrpatt{1}\ -> \aq{\product\ \expr[1]} \\
      \tabT \ldots \\
      \tabT | \ncstrpatt{i}\ -> \aq{\product\ \expr[i]}) >>= fun \var[v]\ -> \\
      update \aq{\tsel\ \expr\ $\downarrow$} >>= fun () -> \\
      unit \var[v]
    }
  \end{minipage}
}

%%% Local Variables:
%%% mode: latex
%%% TeX-master: "main"
%%% End: 


% correctness

\newcommand{\seq}[1][]{\ensuremath{\tsf{Seq}_{#1}}}
\newcommand{\seqm}[1][]{\ensuremath{\tsf{Seqm}_{#1}}}

\newcommand{\peelval}{\tsf{peel\_val}}
\newcommand{\peelenv}{\tsf{peel\_env}}
\newcommand{\simpprod}{\mbox{\tsf{remove\_update}}}

\newcommand{\pectx}{\peelenv~\ectx}
\newcommand{\valprod}{\val[\mathit{prod}]}
\newcommand{\exprprod}{\expr[\mathit{prod}]}
\newcommand{\exprprodid}{\expr[\mathit{id}]}
\newcommand{\exprprodsimp}{\expr[\mathit{simp}]}

\newcommand{\correctness}{Theorem~\ref{thm-correctness}}
\newcommand{\lemmsimpprod}{Lemma~\ref{lem-simpprod}}
\newcommand{\lemmvals}{Lemma~\ref{lem-vals}}
\newcommand{\lemmvalsconcr}{Lemma~\ref{lem-valsconcr}}
\newcommand{\lemmstates}{Lemma~\ref{lem-states}}

% \newtheorem{corrproof}[thm]{Proof~of~Theorem~\ref{thm-correctness}} % incompatible w/ llncs

% comparing monitored evaluation steps and instrumentation cases

\newcommand{\mstepprodrel}{\ensuremath{\leftrightarrow}}

% experiments

\newcommand{\rpt}{\tsf{Prod}}
\newcommand{\funv}{\tsf{FunV}}

\newcommand{\sflinear}{\tsf{LinIneq}\xspace}
\newcommand{\sfchan}{\tsf{Set}\xspace} 
\newcommand{\liter}{\tsf{TI-Term}\xspace} 
\newcommand{\literrank}{\tsf{R-Term}\xspace} 

\newcommand{\hobenchmark}{\texttt{(H)}}
%\newcommand{\hobenchmark}{{\tiny\ensuremath{\heartsuit}}}
\newcommand{\benchmarkontrees}{{\tiny\ensuremath{\clubsuit}}}
\newcommand{\benchmarkonfiledescriptors}{{\tiny\ensuremath{\spadesuit}}}


%%% Local Variables: 
%%% mode: latex
%%% TeX-master: "main"
%%% End: 


%%% Local Variables: 
%%% mode: latex
%%% TeX-master: "main"
%%% End: 





\begin{document}

% \conferenceinfo{WXYZ '05}{date, City.} 
% \copyrightyear{2005} 
% \copyrightdata{[to be supplied]} 

% \titlebanner{}
% \preprintfooter{}

% \pagestyle{headings}

\title{Binary Reachability Analysis of Higher Order Functional Programs}
%\title{Temporal Monitoring and Verification of Higher Order Functional Programs}
\author{Rusl\'{a}n Ledesma-Garza \and Andrey Rybalchenko}
\institute{Technische Universit\"at M\"unchen}

\maketitle

\begin{abstract}
%

%
Automated verification of multi-threaded programs requires explicit
identification of the interplay between interacting threads, so-called
environment transitions, to enable scalable, compositional reasoning.  
Once the environment transitions are identified, we can prove program
properties by considering each program thread in isolation, as the
environment transitions keep track of the interleaving with other
threads.
Finding adequate environment transitions that are sufficiently precise
to yield conclusive results and yet do not overwhelm the verifier with
unnecessary details about the interleaving with other threads is a
major challenge.
In this paper we propose a method for safety verification of
multi-threaded programs that applies (transition) predicate
abstraction-based discovery of environment transitions, exposing a
minimal amount of information about the thread interleaving.
The crux of our method is an abstraction refinement procedure that
uses recursion-free Horn clauses to declaratively state abstraction
refinement queries.
Then, the queries are resolved by a corresponding constraint solving
algorithm.
We present preliminary experimental results for mutual exclusion
protocols and multi-threaded device drivers.



\iffalse % popl submission abstract
%
In order to scale, verification of multi-threaded programs requires
definition of auxiliary variables that reveal some details about the
local state of the concurrent threads and construction of
rely/guarantee assertions that describe the thread interaction in
terms of shared and auxiliary variables.
%
In this paper we present an automatic method for the discovery of
auxiliary variables and relies/guarantees.
Our method is based on a generalization of counterexample guided
abstraction refinement to the multi-threaded setting.
The crux of the generalization is a notion of tree-like
counterexamples that explicitly account for thread interleaving.
By solving the corresponding tree-like path constraints, our method
reveals additional local facts through auxiliary variables and
iteratively strengthens relies/guarantees.
Furthermore, the tree-like path constraints can avoid the introduction
of auxiliary variables when they are not necessary.
%Furthermore, the tree-like path constraints enable the discovery of
%modular relies/guarantees (whose auxiliary variables do not reveal any
%local facts) whenever a modular proof exists.
We devised a solver for the tree-like constraints that is based on
Craig interpolation and used it in a prototype implementation of our
method.
We present preliminary experimental results for mutual exclusion
protocols and concurrent device drivers.
\fi


\iffalse % PLDI abstract
Automated verification of multi-threaded programs requires explicit
identification of the interplay between interacting threads, so-called
environment assumptions, to enable scalable reasoning.  
Once identified, these assumptions can be used for reasoning with one
program thread at a time, which is possible by using the respective
environment assumption to model the interleaving with other threads.
Finding adequate assumptions that are sufficiently precise to yield
conclusive results and yet keep track only of necessary facts about
the execution environment in order to scale well is a major challenge.
In this paper we propose a constraint-based technique for the
inference of such assumptions.
Our technique automatically steers towards an optimal
precision/efficiency trade-off between the extremes of efficient, but
incomplete thread-modular reasoning and complete, but prohibitively
expensive consideration of all interleavings.
For this task, we pinpoint a declarative formulation of modular
verification that allows one to express requisite environment
assumptions using constraints and admits algorithmic solutions based
on abstract fixpoint checking.
We describe an application of our environment assumption inference for
the verification of reachability and termination properties of
multi-threaded programs, and present our experience with its
implementation as well as evaluation in practice. 
\fi

%%% Local Variables: 
%%% mode: latex
%%% TeX-master: "main"
%%% End: 


\iffalse
The automata-theoretic approach to program verification is an
effective and uniform way of dealing with complex temporal properties.
The crux of the approach is to represent the property by a monitor
that inspects program computations.
The monitor keeps record of the (intermediate) inspection results, and
if the property violation is detected then an appropriate note is
recorded.
Then, the monitor is composed with the program, which produces a
product program that monitors its own computation and can be analyzed
using tools for proving reachability and (fair) termination (i.e.,
sophisticated temporal reasoning is no longer necessary).
While already ubiquitous for dealing with imperative programs, monitor
based approaches were recently applied for termporal verification of
higher order functional programs by using higher order recursion
schemes (HORS), e.g.~\cite{Ong06,Kobayashi09}.

In this paper, we present an alternative approach to monitoring and
verifying higher order functional programs that applies the monitor
composition directly on the program source code (instead of composing
with a HORS derived from the program by applying finite state
abstraction on data types).
We formulate our approach for a \miniocaml language as a program
transformation algorithm and present its implementation.
Our experimental evaluation in combination with the DSolve verifier
for OCaml~\cite{DSolveCAV} shows that the scalability and
effectiveness of existing abstraction-based assertion checking tools
can be effectively leveraged for dealing with temporal safety and
liveness properties of higher order functional programs.



The automata-theoretic approach is a standard method for dealing with
complex temporal properties of imperative programs and only requires
program verifiers that deal reachability and (fair) termination
properties.
The automata-theoretic semantics of imperative programs directly
supports monitoring and product construction.
Until now, the temporal reasoning about functional programs does not
rely on monitoring and reduction to reachability and (fair)
termination.
Instead, the reasoning is executed using type systems and type
inference algorithms that are developed/customized for the specific
property at hand.
\fi



\iffalse

In this paper, we present an application of the automata-theoretic
approach to the temporal verification, i.e., safety and liveness, of
\miniocaml\ programs with higher-order procedures.  
First, we present the notion of monitoring for \miniocaml\ computations
given as evaluation trees, and define monitor specifications that can be
composed with \miniocaml\ programs.
Second, we present a product construction algorithm that composes an
\miniocaml\ program with a monitor specification in the presence of
higher order procedures.
Third, we present an implementation of the proposed approach that uses
the \dsolve\ reachability checker, together with its experimental
evaluation on examples from the literature.
The examples include resource analysis and termination checking tasks.
\fi
\end{abstract}
 
% \mbox{ }
% \\ \\
%\nocite{Hofmann00,Hofmann02,Hofmann03,Kobayashi09}

\section{Introduction}

Tools and techniques for proving program termination are important for
increasing software quality~\cite{CACM}.
System routines written in imperative programming languages received a
significant amount of attention recently,
e.g.,~\cite{copytrick,Julia,KroeningCAV10,GieslBytecode,YangESOP08,SepLogTerm}. 
A number of the proposed approaches rely on transition invariants -- a
termination argument that can be constructed incrementally using
abstract interpretation~\cite{transitioninvariants}.
Transition invariants are binary relations over program states.
Checking if an incrementally constructed candidate is in fact a
transition invariant of the program is called binary reachability
analysis.
For imperative programs, its efficient implementation can be obtained
by a reduction to the reachability analysis, for which practical tools
are available, e.g.,~\cite{Slam,Blast,fsoft,Impact}.
The reduction is based on a program transformation that stores one
component of the pair of states under consideration in auxiliary program
variables, and then checks if the pair is in the transition
invariant~\cite{copytrick}. 
The transformed program is verified using an existing safety checker. 
If the safety checker succeeds then the original program terminates on all inputs.  
% The set of reachable program states of the transformed program can be
% directly used in the binary reachability analysis.

For functional programs, recent approaches for proving termination
apply the size change termination (SCT) argument~\cite{SCTPOPL2001}.
This argument requires checking the presence of an infinite descent within
data values passed to application sites of the program on any infinite
traversal of the call graph.
This check can be realized in two steps.
First, every program function is translated into a set of so-called
size-change graphs that keep track of decrease in values between the
actual arguments and values at the application sites in the
function.
Second, the presence of a descent is checked by computing a transitive
closure of the size-change graphs. 
Originally, the SCT analysis was formulated for first order functional
programs manipulating well-founded data, yet using an appropriate
control-flow analysis, it can be extended to higher order programs,
see
e.g.,~\cite{SereniICFP07,Sereni05terminationanalysis,SereniAPLAS05}.
Alternatively, an encoding into term rewriting can be used to make
sophisticated decidable well-founded orderings on terms applicable to proving
termination of higher order programs~\cite{GieslHaskell}.

The SCT analysis is a decision problem (that checks if there is an infinite descent in the abstract program defined by size change graphs), however it is an incomplete method for proving termination. 
SCT can return ``don't know'' for terminating programs that
manipulate non-well-founded data, e.g., integers, or when an interplay
of several variables witnesses program termination. 
In such cases, a termination prover needs to apply a more general termination argument.
Usually, such termination arguments require proving that certain expressions over program variables decrease as the computation progresses and yet the decrease cannot happen beyond a certain bound.

In this paper, we present a general approach for proving termination of higher order functional programs that goes beyond the SCT analysis.
Our approach explores the applicability of transition invariants to
this task by proposing an extension (wrt.\ imperative case) that deals
with partial applications, a programming construct that is particular for functional programs.
Partial applications of curried functions, i.e., functions that return other functions, represent a major obstacle for the binary reachability analysis. 
For a curried function, the set of variables whose values need to be stored in auxiliary variables keeps increasing as the function is subsequently applied to its arguments. 
However these arguments are not necessarily supplied simultaneously,
which requires intermediate storage of the argument values given so
far.
In this paper, we address such complications.

We develop the binary reachability analysis for higher order programs in two steps.
First, we show how intermediate nodes of program evaluation trees, \mbox{so-called} judgements, can be augmented with auxiliary values needed for tracking binary reachability. 
The auxiliary values store arguments provided at application sites.
Then, we show how this augmentation can be performed on the program source code such that the evaluation trees of the augmented program correspond to the result of augmenting the evaluation trees of the original program.
The source code transformation introduces additional parameters to functions occurring in the program. For curried functions, these additional parameters are interleaved with the original parameters, which allows us to deal with partial applications.

Our binary reachability analysis for higher order programs opens up an approach for termination proving in the presence of higher order functions that exploits a highly optimized safety checker, e.g.,~\cite{TerauchiPOPL10,RupakRanjit,KobayashiPLDI11,Dsolve,HMC},
for checking the validity of a candidate termination argument.
Hence, we can directly benefit from sophisticated abstraction
techniques and algorithmic improvements offered by these tools, as inspired by~\cite{copytrick}.

In summary, this paper makes the following contributions.
%
\begin{itemize}
\item A notion of binary reachability analysis for 
  higher order functional programs.
\item A program transformation that reduces the binary
  reachability analysis to the reachability analysis.
  % code   transformation. 
\item An implementation of our approach and its evaluation on micro
  benchmarks from the literature.
\end{itemize}


%%% Local Variables: 
%%% mode: latex
%%% TeX-master: "main"
%%% End: 

\section{Illustration}

In this section we illustrate what our transformation adds to the
program in order to keep track of pairs of argument valuations for
checking a transition invariant candidate.

We consider the following curried function \texttt{f} that has a type
\texttt{x:int -> y:int -> ret:int}.
Here, we annotated the parameter and return value types with
identifiers to improve readability.
%
\begin{center}
\begin{minipage}[h]{.8\linewidth}
\begin{small}
\begin{verbatim}
let rec f x = if x > 0 then f (x-1) else fun y -> f x y
\end{verbatim}
\end{small}
\end{minipage}
\end{center}
%
This function shows that -- in contrast to proving termination of
recursive procedures in imperative programs -- it is important to
differentiate between partial and complete applications when dealing
with curried functions.
First, we observe that any partial application of \texttt{f}
terminates.
For example, \texttt{f 10} stops after ten recursive calls and returns
a function \texttt{fun y -> f 10 y} where \texttt{f} is bound to a
closure.
That is, there is no infinite sequence of \texttt{f} applications that
are passed only one argument.
In contrast, any complete application of \texttt{f} does not terminate.
For example, \texttt{f 1 1} will lead to an infinite sequence of
\texttt{f} applications such that each of them is given two arguments.

Our binary reachability analysis takes as input a specification that
determines which kind of applications we want to keep track of. 
The specification consists of a function identifier, e.g., \texttt{f},
a number of parameters, e.g., one, and a transition invariant candidate. 
Then, such a specification requires that applications of \texttt{f} to
one argument satisfy the transition invariant candidate.
Alternatively, we may focus on applications of \texttt{f} to two
arguments.

Once the specification is given, we transform \texttt{f} into a
function \texttt{f\_m} that keeps track of arguments on which \texttt{f} was
applied using additional parameters \texttt{old\_x} and~\texttt{old\_y}.
As a result, \texttt{f\_m} fulfills two requirements. 
First, it computes a result value \texttt{res} such that $\mathtt{res}
= \mathtt{f\ x\ y}$. 
Second, it computes new values of additional parameters.
If \texttt{f\_m} was an imperative program, we would obtain the type
%
\begin{small}
\begin{verbatim}
        x:int * y:int * copied:bool * old_x:int * old_y:int ->
          ret:int * new_copied:bool * new_x:int * new_y:int
\end{verbatim}
\end{small}
% 
where \texttt{new\_x} and \texttt{new\_y} are computed as follows. 
If \texttt{old\_x} already stores a value that was given to \texttt{x} in the past, i.e.,
if $\mathtt{copied} = \mathtt{true}$, then $\mathtt{new\_x} = \mathtt{old\_x}$. 
Otherwise, \texttt{f\_m} can nondeterministically either store
\texttt{x} in \texttt{new\_x} and set $\mathtt{new\_copied} =
\mathtt{true}$, or leave $\mathtt{new\_x} = \mathtt{old\_x}$
and~$\mathtt{new\_copied} = \mathtt{false}$.
The computation of \texttt{new\_y} is similar.
Given a transformed program, checking binary reachability amounts to
checking that at each application site the pair of tuples
$(\mathtt{old\_x}, \mathtt{old\_y})$ and $(\mathtt{x}, \mathtt{y})$
satisfies the transition invariant whenever \texttt{copied}
is~\texttt{true}.

Due to partial applications we cannot expect that values of \texttt{x}
and \texttt{y} are provided simultaneously, which complicates both
computation of \texttt{new\_x} and \texttt{new\_y} and checking if
$(\mathtt{old\_x}, \mathtt{old\_y})$ together with $(\mathtt{x},
\mathtt{y})$ satisfy the transition invariant. 
Hence, we need to keep track of arguments as they are provided, which
requires ``waiting'' for missing arguments.
We implement this waiting process by introducing additional parameters
\texttt{old\_state\_x} and \texttt{old\_state\_y} for each partial
application, together with their updated versions
\texttt{new\_state\_x} and~\texttt{new\_state\_y}.
Each additional parameter accumulates arguments in its first
component, and it keeps a tuple of previously provided arguments in
its second component. 
We obtain the following type for \texttt{f\_m}.
%
\begin{small}
\begin{verbatim}
x:int 
-> old_state_x:((int * int) *                (* accumulate x and y     *)
                (bool * int * int))          (* store copied, x, and y *)
-> (y:int 
    -> old_state_y:((int * int) *            (* accumulate x and y     *)    
                    (bool * int * int))      (* store copied, x, and y *)
    -> ret:int * new_state_y:((int * int) * 
                              (bool * int * int))) *
   new_state_x:((int * int) * 
                (bool * int * int))
\end{verbatim}
\end{small}
% 
We refer to \texttt{(int * int) * (bool * int * int)} as~\texttt{state}. 
Then \texttt{f\_m} has the type:
%
\begin{small}
\begin{verbatim}
      x:int -> old_state_x:state -> 
        (y:int -> old_state_y:state -> ret:int * new_state_y:state) *
        new_state_x:state
\end{verbatim}
\end{small}
% 
We formalize the above transformation in
Section~\ref{sec-transformation}. 
Figure~\ref{fig-ex-prod} presents a detailed execution protocol of
applying our transformation on the above program.

\iffalse
When dealing with unary applications, the binary reachability
analysis can show that every pair of unary applications is included
in a transition invariant using the following check.
\fi

Note that if complete applications of a function terminate, then every
partial application of the function terminates.
For example, consider the following function~\texttt{g}.
%
\begin{center}
\begin{verbatim}
let rec g x = if x > 0 then g (x-1) else fun y -> x+y
\end{verbatim}
\end{center}
%
This function does not have any infinite application sequences neither
for complete nor for partial applications.


%%% Local Variables: 
%%% mode: latex
%%% TeX-master: "main"
%%% End: 

\section{Preliminaries}

A \emph{Petri net} is a tuple $(P, T, F, M_0)$, where $P$ is the set of
\emph{places}, $T$ is the set of \emph{transitions},
$F \subseteq P\times T \cup T\times P$ is the \emph{flow relation}
and $M_0$ is the initial marking.
We identify $F$ with its characteristic function
$P\times T \cup T\times P \to \{0, 1\}$.

For $x\in P\cup T$, the \emph{pre-set} is
$\pre{x}=\{y\in P\cup T\mid (y,x) \in F\}$
and the \emph{post-set} is $\post{x}=\{y\in P\cup T\mid (x,y) \in F\}$.
The pre- and post-set of a subset of $P \cup T$ are the union of
the pre and post-sets of its elements.

A \emph{marking} of a Petri net is a function $M\colon P \to \mathbb{N}$,
which describes the number of tokens $M(p)$ that the marking puts in
each place $p$.

Petri nets are represented graphically as follows: places and transitions
are represented as circles and boxes, respectively. An element $(x,y)$
of the flow relation is represented by an arc leading from $x$ to $y$.
A token on a place $p$ is represented by a black dot in the circle
corresponding to $p$.

A transition $t \in T$ is \emph{enabled at $M$} iff
$\forall p \in \pre{t} \colon M(p) \ge F(p, t)$.
If $t$ is enabled at $M$, then $t$ may \emph{fire} or \emph{occur},
yielding a new marking $M'$ (denoted as $M \xrightarrow{t} M'$),
where $M'(p) = M(p) + F(t,p) - F(p,t)$.

A sequence of transitions, $\sigma = t_1 t_2 \ldots t_r$ is an
$\emph{occurence sequence}$ of $N$ iff there exist markings
$M_1, \ldots, M_r$ such that $M_0 \xrightarrow{t_1} M_1
\xrightarrow{t_2} M_2 \ldots \xrightarrow{t_r} M_r$. The marking
$M_r$ is said to be \emph{reachable} from $M_0$ by the occurence
of $\sigma$ (denoted $M \xrightarrow{\sigma} M_r$).

A property $P$ is a safety property expressed over linear arithmetic
formulas. The property $P$ holds on a marking $M$ iff $M \models P$.
Examples of properties are $s_1 \le 2$, $s_1 + s_2 \ge 1$ and
$((s_1 \le 1) \land (s_2 \ge 1)) \lor (s_3 \le 1)$.

A Petri net $N$ satisfies a property $P$ (denoted by $N \models P$)
iff for all reachable markings $M_0 \xrightarrow{\sigma} M$
$M \models P$ holds.

An invariant $I$ of a Petri net $N$ is a property such that $N$ satisfies $I$.
The invariant $I$ is inductive iff for all markings
$M$, if $M \models I$ and $M \xrightarrow{t} M'$ for some $t$, then
$M' \models I$.

A trap is a set of places $S \subseteq P$ such that $\post{S} \subseteq \pre{S}$.
If a trap $S$ is marked in $M_0$, i.e. $\sum_{p \in S} M_0(p) > 0$, then it is also marked in all reachable markings.


\section{Binary reachability on evaluation trees}
\label{sec-semantics}

In this section we make the first step towards our program
transformation.
We present an augmentation of evaluation trees that allows us to
reduce the binary reachability analysis to the validity analysis of
annotated judgement.
Each judgement is augmented with a Boolean and an $\theArity$-ary
tuple of values, which we will refer to as a \emph{state}.

\newcommand{\stf}{\ensuremath{(\mathit{false}, 0)}}
\newcommand{\stzero}{\ensuremath{(\mathit{true}, 0)}}
\newcommand{\stone}{\ensuremath{(\mathit{true}, 1)}}
\newcommand{\vstf}{\ensuremath{\st[f]}}
\newcommand{\vstzero}{\ensuremath{\st[0]}}
\newcommand{\vstone}{\ensuremath{\st[1]}}


\newcommand{\vf}{\ensuremath{(\ttt{f},
    \ttt{fun~x~->~if~x~>~0~then~f~(x~-~1)~else~fun~y~->~y~+~1}, \emptyectx)}}
\newcommand{\vvf}{\val[f]}
\newcommand{\vfx}{\ensuremath{(\ttt{fun~y~->~y~+~1}, [\ttt{f} \emapsto \vvf; \ttt{x} \emapsto 0])}}
\newcommand{\vvfx}{\val[\ttt{f\ x}]}
% \newcommand{\ectxf}{\ensuremath{\ttt{f} \emapsto \vf}}
\newcommand{\vectxf}{\ectx[\ttt{f}]}
\newcommand{\vectxfxone}{\ectx[\ttt{f\ 1}]}
\newcommand{\vectxfxzero}{\ectx[\ttt{f\ 0}]}

\begin{figure}[t]
\begin{minipage}{\linewidth}
  \small

    \[
    \scalebox{.95}{
      \inference{
        \begin{sideways}
        \inference{}{
          \mevj{\evj{\vectxf}{\ttt{1}}{\ttt{1}}}{}{\vstf}
        }
        \end{sideways}
        &
        \begin{sideways}
        \inference{}{
          \mevj{\evj{\vectxf}{\ttt{f}}{\vvf}}{}{\vstf}
        }
        \end{sideways}
        &
        \inference{
          \begin{sideways}
          \inference{
            \vdots
          }{
            \mevj{\evj{\vectxfxone}{\ttt{x~>~0}}{\ttt{true}}}{}{\vstone}
          }
          \end{sideways}
          &
          \inference{
            \begin{sideways}
            \inference{
              \vdots
            }{
              \mevj{\evj{\vectxfxone}{\ttt{x~-~1}}{\ttt{0}}}{}{\vstone}
            }
            \end{sideways}
            &
            \begin{sideways}
            \inference{
            }{
              \mevj{\evj{\vectxfxone}{\ttt{f}}{\vvf}}{}{\vstone}
            }
            \end{sideways}
            &
            \inference{
              \begin{sideways}
              \inference{
                \vdots
              }{
                \mevj{\evj{\vectxfxzero}{\ttt{x~>~0}}{\ttt{true}}}{}{\vstone}
              }
              \end{sideways}
              &
              \begin{sideways}
              \inference{
              }{
                \mevj{\evj{\vectxfxzero}{\ttt{fun~y~->~y~+~1}}{\vvfx}}{}{\vstone}
              }
              \end{sideways}
            }{
              \mevj{\evj{ \underbrace{\vectxf \eext \ttt{x} \emapsto \ttt{0}}_{\vectxfxzero} }{\ttt{if~\ldots}}{\vvfx}}{}{\vstone}
            }
          }{
            \mevj{\evj{\vectxfxone}{\ttt{f (x - 1)}}{\vvfx}}{}{\vstone}
          }
        }{
          \mevj{\evj{ \underbrace{\vectxf \eext \ttt{x} \emapsto \ttt{1}}_{\vectxfxone} }{\ttt{if~x~>~0~then~\ldots}}{\vvfx}}{}{\vstone}
        }
      }{
        \mevj{\evj{
            \vectxf 
            % \underbrace{
            %   \ttt{f} \emapsto \underbrace{\vf}_{\vvf}
            % }_{\vectxf} 
          }{\ttt{f\ 1}}{\underbrace{\vfx}_{\vvfx}}}{}{\vstf} 
      }
    }
   \]

   \iffalse
    % 
    \begin{center}
      (a) 
    \end{center}
    \vspace{2ex}
    % 
    \[
    \scalebox{.45}{
      \inference{
        \inference{}{
          \mevj{\evj{\vectxf}{\ttt{1}}{\ttt{1}}}{}{\vstf}
        }
        &
        \inference{}{
          \mevj{\evj{\vectxf}{\ttt{f}}{\vvf}}{}{\vstf}
        }
        &
        \inference{
          \inference{
            \vdots
          }{
            \mevj{\evj{\vectxfxone}{\ttt{x~>~0}}{\ttt{true}}}{}{\vstf}
          }
          &
          \inference{
            \inference{
              \vdots
            }{
              \mevj{\evj{\vectxfxone}{\ttt{x~-~1}}{\ttt{0}}}{}{\vstf}
            }
            &
            \inference{
            }{
              \mevj{\evj{\vectxfxone}{\ttt{f}}{\vvf}}{}{\vstf}
            }
            &
            \inference{
              \inference{
                \vdots
              }{
                \mevj{\evj{\vectxfxzero}{\ttt{x~>~0}}{\ttt{true}}}{}{\vstzero}
              }
              &
              \inference{
              }{
                \mevj{\evj{\vectxfxzero}{\ttt{fun~y~->~y~+~1}}{\vvfx}}{}{\vstzero}
              }
            }{
              \mevj{\evj{ \underbrace{\vectxf \eext \ttt{x} \emapsto \ttt{0}}_{\vectxfxzero} }{\ttt{if~x~>~0~then~\ldots}}{\vvfx}}{}{\vstzero}
            }
          }{
            \mevj{\evj{\vectxfxone}{\ttt{f (x - 1)}}{\vvfx}}{}{\vstf}
          }
        }{
          \mevj{\evj{ \underbrace{\vectxf \eext \ttt{x} \emapsto \ttt{1}}_{\vectxfxone} }{\ttt{if~x~>~0~then~\ldots}}{\vvfx}}{}{\vstf}
        }
      }{
        \mevj{\evj{\underbrace{\ttt{f} \emapsto \underbrace{\vf}_{\vvf}}_{\vectxf}}{\ttt{f\ 1}}{\underbrace{\vfx}_{\vvfx}}}{}{\vstf}
      }
    }
    \]
    \begin{center}
      (b)
    \end{center}
    \fi
  \end{minipage}

  \caption{Annotation of the evalution tree for \ttt{f\ 1} in the 
     environment \vectxf\ = \vf.
     The initial judgement is annotated with $\vstf = \stf$. 
     The annotation changes from $\vstf$ to $\vstone = \stone$ 
     after the first call to \ttt{f}.
     Due to the lack of width, we had to turn several judgements.
   }

%    $\vectxf  = \vf$
            % \underbrace{
            %   \ttt{f} \emapsto \underbrace{\vf}_{\vvf}
            % }_{\vectxf} 



  \label{fig-monitoring}
\end{figure}

%%% Local Variables:
%%% mode: latex
%%% TeX-master: "main"
%%% End: 
 

Before presenting the augmentation procedure, we consider examples of
tree augmentation shown in Figure~\ref{fig-monitoring}.
The root of the tree is augmented with a state $\vstf = (\lfalse, 0)$,
where $\lfalse$ indicates that no argument has been used for the
augmentation yet.
We use $\vst$ to augment judgements in the subtree for the branches
that do not correspond to the evaluation of the body of~\texttt{f}.
When augmenting the subtree that deals with the body, we can
nondeterministically decide to start augmenting with a state that
records the argument of the current application.
That is, in the body subtree we augment with the state $\vstone =
\stone$.
Here, $\ltrue$ indicates that we took a snapshot of the current
application argument, and $1$ is the argument value.
The remaining judgements are augmented with $\vstone$, since we will
not change the snapshot if it was taken, i.e., if the first component
of the augmenting state is~$\ltrue$.

We proceed with an algorithm \monitor that takes as input an initial
state and an evaluation tree and produces an augmented tree.
Each augmented judgement is of the form
\mevj{\evj{\ectx}{\expr}{\val}}{\stupa}{\st}, where \st\ is a state.
As an \emph{initial} state we take a pair $(\lfalse, (v_1 \dots,
v_\theArity))$ where $v_1\ \dots, v_\theArity$ are some values.

See Figure~\ref{fig-monitor-alg}.
\monitor traverses the input tree recursively, by starting from the
root.
Whenever the current judgement is a
$\theFunction/\theArity$-application judgement, then we choose whether
to create a snapshot of the arguments and store them in the state that
is used to augment the subtree that evaluates the body.
We only create a snapshot if the Boolean component of the current
state is~$\lfalse$.
In case we currently do not deal with a
$\theFunction/\theArity$-application judgement, no state change
happens and we proceed with the subtrees.
Once we obtain the augmented versions of the subtrees, we put them
together by creating a node that connects the roots of the subtrees.


  % \begin{tabular}{r@{\ }l l}
  %   $\mevjs \ni$ & \mevjvar \;\; ::=
  %   \mevj{\evj{\ectx}{\expr}{\val}}{\stupa}{\stdown}
  %   & monitored judgement \\[\jot]
  %   $2^{{\mevjs}^+} \ni$ & \mevt  & monitored evaluation tree
  % \end{tabular}

\begin{figure}[t]
  \centering
  \begin{minipage}[t]{\linewidth}
    \linespread{1.2}
    \begin{minipage}[t]{.03\linewidth}
      \small
      \gutternumbering{1}{13}
    \end{minipage}
    \begin{minipage}[t]{.95\linewidth}
      \small
      \algLet \monitor\ ((c, \_\!\_) \textbf{as} \st)\ \evt\ = \\
      \tabT \algLet \rt\ = \tsf{root}\ \evt\ \algIn \\
      \tabT \algMatch \rt\ \algWith\\
      \tabT \algCase{\evj{\ectx}{\theFunction\ \expr[1]\ \dots\
          \expr[\theArity]}{\val}} \\
      \tabTT \algLet \val[1], \dots, \val[\theArity]  = 
      \tsf{eval}\ \ectx\ \expr[1], \dots,
      \tsf{eval}\ \ectx\ \expr[\theArity]\ \algIn \\
      \tabTT \algLet \st'\ = 
      \algIf $\neg c \land \textbf{nondet()}$ \algThen
      ($\ltrue$, \val[1], \dots, \val[\theArity])
      \algElse \st\ \algIn \\
      \tabTT \algLet \evt[p], \evt[f], \evt[b] = 
      immediate subtrees of \evt\ \algIn \\
      \tabTT \algLet \evtP[p], \evtP[f], \evtP[b]\ = 
      \monitor\ \st\ \evt[p], \monitor\ \st\ \evt[p], \monitor\ \st'\ \evt[b]\ \algIn\\
      \tabTT $(\setOf{\tsf{root}\ \evtP[p], 
        \tsf{root}\ \evtP[f], \tsf{root}\ \evtP[b], (\rt, \st))} 
      \cup \evtP[p] \cup \evtP[f] \cup \evtP[b]
      $ \\
      \tabT \algCase{\_\!\_} \\
      \tabTT \algLet \evt[1], \dots, \evt[n] = 
      immediate subtrees of \evt\ \algIn \\
      \tabTT \algLet \evtP[1], \dots, \evtP[n]\ = 
      \monitor\ \st\ \evt[1], \dots, \monitor\ \st\ \evt[n]\ \algIn\\
      \tabTT $(\setOf{\tsf{root}\ \evtP[1], \dots,
        \tsf{root}\ \evtP[n], (\rt, \st))} 
      \cup \evtP[1] \cup \dots \cup \evtP[n]
      $ 
   \end{minipage}
 \end{minipage}
 \caption{Evaluation tree monitoring.  The input consists of a
    monitor state $\st$ and an evaluation tree~\evt.
    $\textbf{nondet()}$ non-deterministically returns either $\ltrue$
    or $\lfalse$.
  }
  \label{fig-monitor-alg}
\end{figure}
%%% Local Variables: 
%%% mode: latex
%%% TeX-master: "main"
%%% End: 


We establish a formal relationship between the
$\theFunction/\theArity$-recursion relation with the augmented
judgements obtained by applying \monitor using the following theorem.
%
\begin{theorem}[\monitor keeps track of
$\theFunction/\theArity$-recursion relation]
\label{thm-monitor}
%
A pair $(v_1, \dots, v_\theArity)$ and~$(u_1, \dots, u_\theArity)$ is
in the $\theFunction/\theArity$-recursion relation if and only if the
result of applying \monitor wrt.\ some sequence of nondeterministic
choices on \evjvar and an initial state contains an augmented
judgement of the form
%
\begin{center}
  \mevj{\evj{\ectx}{$\theFunction$\ \expr[1] \dots
      \expr[\theArity]}{\val}}{\stupa}{
    (\ltrue, (v_1, \dots, v_\theArity))}
\end{center}
%
such that for each $i \in 1..\theArity$ we have $\tsf{eval}\ \ectx\
\expr[i] = u_i$
%
\end{theorem}
%

%%% Local Variables:
%%% mode: latex
%%% TeX-master: "main"
%%% End: 

%\paragraph{\bf Logger monad}
A \emph{monad} consists of a type constructor \ttt{m} of arity~1 and two
operations
%
\begin{small}
\begin{align*}
  \ttt{unit~}&\ttt{:~'a~->~'a~m} \\
  \ttt{(~>>=~)~}&\ttt{:~'a~m~->~('a~->~'b~m)~->~'b~m}
\end{align*}
\end{small} 
%
These operations need to satisfy three conditions called \emph{left
unit}, \emph{right unit}, and \emph{associative}~\cite{Wadler95}.  
We assume that \ttt{state} is a given type.  
Figure~\ref{fig-state-monad} presents a variant of the \emph{logger
monad}~\cite{freetheorems}. The state
update operator \ttt{update} is of type
\ttt{(state~->~state)~->~unit~m}. A monadic expression (resp. value)
is an expression (resp. value) of the logger monad type. 

For example, the monadic expression \ttt{unit 1} evaluates to a
function that takes a state \st\ and returns a pair (\ttt{1}, \st). As
another example, the following monadic expression evaluates to a
function that takes a state \st\ and returns a pair (\ttt{1}, \st\ $+$
\ttt{1}).
\begin{center}
  \ttt{update (fun s -> s + 1) >>= fun () -> unit \ttt{1}}
\end{center}

\begin{figure}[t]
  \linespread{1}
  \begin{minipage}[t]{\columnwidth}
    \small
   \begin{tabular}{l@{\ }l}
     1 & \ttt{(* logger monad type *)} \\[\jot]
     2 & \ttt{type 'a m = state -> 'a * state } \\[\jot]
     3 & \ttt{(* unit operator *)} \\[\jot]
     4 & \ttt{let unit a = fun s -> (a, s) } \\[\jot]
     5 & \ttt{(* bind operator *)} \\[\jot]
     6 & \ttt{let ( >>= ) m k = fun s0 -> } \\[\jot]
     7 & \ttt{\tabTTTTTTTT let v1, s1 = m s0 in } \\[\jot]
     8 & \ttt{\tabTTTTTTTT k v1 s1 } \\[\jot]
     9 & \ttt{(* state transform operator *)} \\[\jot]
     10 & \ttt{let update f = fun s -> ( (), f s ) }
    \end{tabular}
  \end{minipage}
  \caption{Logger monad with state transform operator
    \ttt{update}. The unit operator takes a value and constructs a
    monadic value. The bind operator takes a monadic value and a
    function returning a monadic value, and constructs a new monadic
    value. The state transform operator creates a new monadic value by
    applying the state transformer \ttt{f}.}
  \label{fig-state-monad}
\end{figure}

\iffalse
We clarify the product expressions above by expressing them in
\haskell's \emph{do notation}~\cite{DoNotation}. For
the product of expression \ttt{1} and \msplus, the corresponding
\haskell\ program is the following.
\begin{center}
  \begin{minipage}{.4\columnwidth}
   \small
    \ttt{do \\
      \tabT update (fun s -> s) \\
      \tabT update (fun s -> s) \\
      \tabT unit \ttt{1}}
 \end{minipage}
\end{center}
For the product of expression \ttt{+} and \msplus, the corresponding
\haskell\ program is the following.
\begin{center}
  \begin{minipage}{.9\columnwidth}
    \small
    \ttt{do \\
      \tabT update (fun s -> s) \\
      \tabT update (fun s -> s + 1) \\
      \tabT unit (fun x -> unit (fun y -> unit ((+)\ x\ y)))} 
 \end{minipage}
\end{center}
\fi
%%% Local Variables:
%%% mode: latex
%%% TeX-master: "main"
%%% End: 

\section{Program transformation}
\label{sec-transformation}

In this section we present a program transformation that realizes the
function \monitor presented in Section~\ref{sec-semantics}.
To implement the state passing between judgemets we apply a so-called
logger monad.

\paragraph{\bf Logger monad}
A \emph{monad} consists of a type constructor \ttt{m} of arity~1 and two
operations
%
\begin{small}
\begin{align*}
  \ttt{unit~}&\ttt{:~'a~->~'a~m} \\
  \ttt{(~>>=~)~}&\ttt{:~'a~m~->~('a~->~'b~m)~->~'b~m}
\end{align*}
\end{small} 
%
These operations need to satisfy three conditions called \emph{left
unit}, \emph{right unit}, and \emph{associative}~\cite{Wadler95}.  
We assume that \ttt{state} is a given type.  
Figure~\ref{fig-state-monad} presents a variant of the \emph{logger
monad}~\cite{freetheorems}. The state
update operator \ttt{update} is of type
\ttt{(state~->~state)~->~unit~m}. A monadic expression (resp. value)
is an expression (resp. value) of the logger monad type. 

For example, the monadic expression \ttt{unit 1} evaluates to a
function that takes a state \st\ and returns a pair (\ttt{1}, \st). As
another example, the following monadic expression evaluates to a
function that takes a state \st\ and returns a pair (\ttt{1}, \st\ $+$
\ttt{1}).
\begin{center}
  \ttt{update (fun s -> s + 1) >>= fun () -> unit \ttt{1}}
\end{center}

\begin{figure}[t]
  \linespread{1}
  \begin{minipage}[t]{\columnwidth}
    \small
   \begin{tabular}{l@{\ }l}
     1 & \ttt{(* logger monad type *)} \\[\jot]
     2 & \ttt{type 'a m = state -> 'a * state } \\[\jot]
     3 & \ttt{(* unit operator *)} \\[\jot]
     4 & \ttt{let unit a = fun s -> (a, s) } \\[\jot]
     5 & \ttt{(* bind operator *)} \\[\jot]
     6 & \ttt{let ( >>= ) m k = fun s0 -> } \\[\jot]
     7 & \ttt{\tabTTTTTTTT let v1, s1 = m s0 in } \\[\jot]
     8 & \ttt{\tabTTTTTTTT k v1 s1 } \\[\jot]
     9 & \ttt{(* state transform operator *)} \\[\jot]
     10 & \ttt{let update f = fun s -> ( (), f s ) }
    \end{tabular}
  \end{minipage}
  \caption{Logger monad with state transform operator
    \ttt{update}. The unit operator takes a value and constructs a
    monadic value. The bind operator takes a monadic value and a
    function returning a monadic value, and constructs a new monadic
    value. The state transform operator creates a new monadic value by
    applying the state transformer \ttt{f}.}
  \label{fig-state-monad}
\end{figure}

\iffalse
We clarify the product expressions above by expressing them in
\haskell's \emph{do notation}~\cite{DoNotation}. For
the product of expression \ttt{1} and \msplus, the corresponding
\haskell\ program is the following.
\begin{center}
  \begin{minipage}{.4\columnwidth}
   \small
    \ttt{do \\
      \tabT update (fun s -> s) \\
      \tabT update (fun s -> s) \\
      \tabT unit \ttt{1}}
 \end{minipage}
\end{center}
For the product of expression \ttt{+} and \msplus, the corresponding
\haskell\ program is the following.
\begin{center}
  \begin{minipage}{.9\columnwidth}
    \small
    \ttt{do \\
      \tabT update (fun s -> s) \\
      \tabT update (fun s -> s + 1) \\
      \tabT unit (fun x -> unit (fun y -> unit ((+)\ x\ y)))} 
 \end{minipage}
\end{center}
\fi
%%% Local Variables:
%%% mode: latex
%%% TeX-master: "main"
%%% End: 


\paragraph{\bf Transformation of types}

We transform each program expression into a monadic expression that
keeps track of the state that results in the judgement augmentation.
Figure~\ref{fig-types-prods} presents the function \monadic\ that maps
types of expressions in the original program to types of the
transformed program.
Function \monadic\ indicates that a tranformed program is a \miniocaml\
function that takes an initial state and returns a \mbox{pre-monadic}
program value together with a final state.

\begin{figure}[t]
  \linespread{1}
  \begin{minipage}[t]{\columnwidth}
    \small
    \begin{tabular}{l@{\ }l}
      1 & \algLet \algRec \monadic\ \type\ = (\premonadic\ \type) \moncstr \\[\jot]
      & \\[\jot]
      2 & \algAnd \premonadic\ = \algFunction \\[\jot]
      3 & \tabT $|$ \ttt{\type[1] -> \type[2]}\
      \algArrow \ttt{$(\premonadic\ \type[1])$ -> $(\premonadic\
        \type[2])$ m} \\[\jot]
      % 6 & \tabT $|$ \ttt{\type[1]\ *\ \ldots\ *\ \type[n]}\
      % \algArrow \ttt{$(\premonadic\ \type[1])$\ *\ \ldots\ *\
      %   $(\premonadic\ \type[n])$} \\[\jot] 
      4 & \tabT $|$ \ttt{(\type[1],\ \ldots,\ \type[n])\ \tycstr}\ 
      \algArrow \ttt{($(\premonadic\ \type[1])$,\ \ldots,\ $(\premonadic\
      \type[n])$)\ \tycstr} \\[\jot]
      5 & \tabT $|$ \type\ \algArrow \type
    \end{tabular}
  \end{minipage}
  \caption{Type transformation function \monadic.}
  \label{fig-types-prods}
\end{figure}

For example, consider the following applications of \monadic.
%
\begin{align*}
  \monadic\ \texttt{(int -> (int -> int))} = \;
  \begin{array}[t]{@{}l@{}}
    \texttt{(int -> ((\premonadic\ int -> int) m)) m}\\[\jot]
    \texttt{(int -> ((int -> int m) m)) m}
    % (int -> ((prem int -> int) m)) m
    % (int -> ((int -> int m) m)) m
  \end{array}
\end{align*}
%
\begin{figure}[!t]
%  \hline
  \centering
  \begin{minipage}[t]{1.05\linewidth}
    \linespread{1.2}
    \begin{minipage}[t]{.03\linewidth}
      \small
      1 \\ 2 \\ 3 \\[\jot] \vspace{.3ex} \\ 4 \\ 5 \\ 6 \\7 \\8 \\ 9 \\
      10 \\ 11 \\ 12 \\ \\ 13 \\ 14 \\ 15 \\ 16
%      \gutternumbering{1}{29}
    \end{minipage}
    \begin{minipage}[t]{1.02\linewidth}
      \small
     % \algLet \tiCheck\ a\ = \\
     %  \tabT \quo{ \hspace{.5ex} \ttt{let m\_c, m\_1, ..., m\_\theArity\ = m in \\
     %    \tabTT if m\_c then assert(\theTI); \\
     %    \tabTT }a\ttt{, if not m\_c \&\& nondet () then true, }a\ttt{ else m }  } \\
     %  \\
      \algLet \tselEnter\ = \algFunction \\
      \tabT \algCase{\theFunction\ \expr[1]} \\ \tabTT
      \quo{ \ttt{fun v -> 
          fun ((\_\!\_, a$_2$, ..., a$_\theArity$), m) -> 
          (v, a$_2$, ..., a$_\theArity$), m} }
      \\
      \tabT \hspace{.5ex} \vdots\\
      \tabT \algCase{( \ldots\ (\theFunction\ \expr[1]) \dots\ \expr[\theArity-1])} \\ \tabTT
      \quo{ \ttt{fun v -> 
          fun ((a$_1$, ..., a$_{\theArity-2}$, \_\!\_, a$_\theArity$), m) -> 
%          \\ \hspace*{25ex}
          (a$_1$, ..., a$_{\theArity-2}$, v, a$_\theArity$), m} }
      \\
      \tabT \algCase{( \ldots\ (\theFunction\ \expr[1]) \dots\
        \expr[\theArity])} \\ 
%     \tabTT
%       \quo{ \ttt{fun v -> 
%           fun ((a$_1$, ..., a$_{\theArity-1}$, \_\!\_), m) -> 
% %          \\ \hspace*{25ex}
%           \aq{
%             \tiCheck\ 
%             \quo{(a$_1$, ..., a$_{\theArity-1}$, v)}} }}
%       \\
      \tabTT
      \quo{ \ttt{fun v -> 
          fun ((a$_1$, ..., a$_{\theArity-1}$, \_\!\_), m) -> \\ \hspace*{25ex}
          % let a$_{\theArity}$ =  v in           \\ \hspace*{25ex}
          let a = a$_1$, ..., a$_{\theArity-1}$, v in           \\ \hspace*{25ex}
          let m\_c, m\_1, ..., m\_\theArity\ = m in \\ \hspace*{25ex}
           if m\_c then assert\ \theTI ; \\ \hspace*{25ex}
          a, if not m\_c \&\& nondet () then true, a else m} } \\
      \tabT \algCase{\_\!\_}\ \quo{ \ttt{fun \_\!\_ -> id} }
      \\
      \\
      \algLet \tselExit\ = \algFunction \\
      \tabT \hspace{1ex}$|$ \theFunction\ \expr[1] $| \, \ldots \, |$ ( \ldots\ (\theFunction\ \expr[1]) \dots\ \expr[\theArity - 1]) $\rightarrow$
      \quo{ \ttt{fun \_\!\_ s -> (fun \_\!\_ -> s) } }
      \\
      \tabT \algCase{( \ldots\ (\theFunction\ \expr[1]) \dots\ \expr[\theArity])}
      \quo{ \ttt{fun s \_\!\_ -> (fun \_\!\_ -> s) } }
      \\
      \tabT \algCase{\_\!\_}\ \quo{ \ttt{fun \_\!\_ \_\!\_ -> id} }
    \end{minipage}
  \end{minipage}
  \caption{The transformer selector functions \tselEnter\ and \tselExit.  
    The operator \quo{$\;\cdot\;$} emits a \miniocaml\ expression after evaluating
    expressions that are embedded using~\aq{$\;\cdot\;$}.  }
  \label{fig-tsel}
\end{figure}
%%% Local Variables: 
%%% mode: latex
%%% TeX-master: "main"
%%% End: 


\begin{figure}[p]
%  \hline
  \centering
  \begin{minipage}[t]{\linewidth}
    \linespread{1.2}
    \begin{minipage}[t]{.02\linewidth}
      \small
%      \mbox{} \\
      \gutternumbering{1}{36}
    \end{minipage}
    \begin{minipage}[t]{.962\linewidth}
      \small
%      \mbox{} \\
      \algLet \product\ \expr\ = \\
      \tabT \algMatch \expr\ \algWith
      \\
      \tabT \algCase{\cst\ } \\
      \tabTT \algLet \var[1], \ldots, \var[\arity\ \cst]\ = \freshvar, \ldots, \freshvar\ \algIn \\
      \tabTTT\hspace{-2.3ex}
      \quo{
        \ttt{unit (fun \var[1]-> \\ \tabTTTT 
          \ldots\ \\ \tabTTTTT 
          unit (fun \var[\arity\ \cst]-> \\  \tabTTTTTT 
          unit (\cst\ \var[1] \ldots\ \var[\arity\ \cst])) \ldots\ )
        }}\\
      \iffalse
      \tabT \algCase{\cst\ } \\
      \tabTT \algLet \var[1], \ldots, \var[\arity\ \cst]\ = \freshvar, \ldots, \freshvar\ \algIn \\
      \tabTTT\hspace{-2.3ex}
      \quo{
        \ttt{unit (fun \var[1]-> \\ \tabTTTT 
          \ldots\ \\ \tabTTTTT 
          unit (fun \var[\arity\ \cst]-> \\  \tabTTTTTT 
          unit (\cst\ \var[1] \ldots\ \var[\arity\ \cst])) \ldots\ )
        }
      }\\
      \fi
      \tabT \algCase{\var\ } % \\      \tabTTT\hspace{-2.3ex}
      \quo{
        \ttt{unit \var}
      }
      \\
%      % \tabT \algCase{\ntupleexpr\ \algWhen $n > 1$} \\
      % \tabTT \algLet \var[1], \ldots, \var[n]\ = \freshvar, \ldots, \freshvar\ \algIn \\
      % \tabTTT\hspace{-2.3ex}
      % \quo{
      %   \ttt{\aq{\product\ \expr[n]} >>= fun \var[n]-> %\\ \tabTTT 
      %     \ldots %\\ \tabTTT 
      %     \aq{\product\ \expr[1]} >>= fun \var[1]-> %\\ \tabTTT 
      %     unit (\var[1], \ldots, \var[n])}~}
%      % \\
      \tabT \algCase{\ttt{\cstr(\ntupleexpr)}} \\
      \tabTT \algLet \var[1], \ldots, \var[n]\ = \freshvar, \ldots, \freshvar\ \algIn \\
      \tabTTT\hspace{-2.3ex}
      \quo{
        \ttt{
          % update \aq{\tsel\ \expr\ $\uparrow$} >>= fun () ->\\
          % \tabTTT
          \aq{\product\ \expr[n]} >>= fun \var[n] -> \\
          \tabTTT \ldots \\
          \tabTTT \aq{\product\ \expr[1]} >>= fun \var[1] -> \\
          % \tabTTT update \aq{\tsel\ \expr\ $\downarrow$} >>= fun () -> \\
          \tabTTT unit (\cstr (\var[1], \ldots, \var[n]))
        }
      }
      \\
      \tabT \algCase{\expr[f]\ \expr[p]\ } \\
      \tabTT \algLet \var[app], \var[p], \vst[\mathit{full}]\ \vst[\mathit{partial}] = \freshvar, \freshvar, \freshvar, \freshvar\ \algIn \\
      \tabTTT\hspace{-2.3ex}
      \quo{
        \ttt{
          fun \vst[\mathit{full}]\ -> \\
          \tabTTTT (\aq{\product\ \expr[p]} >>= fun \var[p] -> \\
          \tabTTTT \aq{\product\ \expr[f]} >>= fun \var[f] -> \\
          \tabTTTT update (\aq{\tselEnter\ \expr } \var[p]) >>= fun () \vst[\mathit{partial}]\  -> \\
          \tabTTTT (\var[f]\ \var[p] >>= fun \varapp\ -> \\
          \tabTTTT update (\aq{\tselExit\ \expr} \vst[\mathit{full}] \vst[\mathit{partial}]) >>= fun () -> \\
          \tabTTTT unit \varapp) \vst[\mathit{partial}]) \vst[\mathit{full}]
        }
      }
      \\
      \tabT
      \algCase{ \funexpr\ } %\\      \tabTT\hspace{-3ex}
      \quo{ \ttt{unit (fun \var\ -> \aq{\product\ \expr[b]})} }
      \\
      \tabT \algCase{ \ttt{let \patt\ = \expr[1]\ in \expr[2]}\ } 
%      \\ \tabTTTT\hspace{-2.3ex}
      \quo{
        \ttt{( \aq{\product\ \expr[1]} %\hspace{2.5ex}
          >>= fun \patt\ -> %\\  \tabTTTT 
          \aq{\product\ \expr[2]} )}
      }
      \\
      \tabT
      \algCase{ \ttt{let rec \var[f] = \funexpr\ in \expr[2]}\ } 
      \\ \tabTTTT\hspace{-2.3ex}
      \quo{
        \ttt{( let rec \var[f] = fun \var\ -> \aq{\product\ \expr[b]} 
          in \aq{\product\ \expr[2]})}
      }
      \\
      \tabT
      \algCase{ 
        \ttt{match \expr[m]\ with} %\\
        \ttt{ | \ensuremath{e^p_1}\ -> \expr[1]} % \\
        \ttt{\tabT \ldots} %\\
        \ttt{ |  \ensuremath{e^p_i}\ -> \expr[i]}
      } \\
      \tabTTT \algLet \var[m]
      \ = \freshvar
      \ \algIn \\
      \tabTTTT\hspace{-2.3ex}
      \quo{
        \ttt{
          % update \aq{\tsel\ \expr\ $\uparrow$} >>= fun () -> \\
          % \tabTTT
          \aq{\product\ \expr[m]} >>= fun \var[m]\ -> \\
          \tabTTTT  ( match \var[m] with \\ \tabTTTTT 
          | \ensuremath{e^p_1}\ -> \aq{\product\ \expr[1]}  \\
          \tabTTTTT \ldots  \\
          \tabTTTTT| \ensuremath{e^p_i}\ -> \aq{\product\ \expr[i]} \\ \tabTTTT 
          )
        }
      }
      \iffalse
      \\
      \tabT
      \algCase{ 
        \ttt{match \expr[m]\ with} \\
        \ttt{\tabTT | \ensuremath{e^p_1}\ -> \expr[1]} \\
        \ttt{\tabTT \ldots} \\
        \ttt{\tabTT |  \ensuremath{e^p_i}\ -> \expr[i]}
      } \\
      \tabTTT \algLet \var[m]
%      , \var[v]
      \ = \freshvar
%      , \freshvar
      \ \algIn \\
      \tabTTTT\hspace{-2.3ex}
      \quo{
        \ttt{
          % update \aq{\tsel\ \expr\ $\uparrow$} >>= fun () -> \\
          % \tabTTT
          \aq{\product\ \expr[m]} >>= fun \var[m]\ -> \\
          \tabTTTT  ( match \var[m] with \\
          \tabTTTTT | \ensuremath{e^p_1}\ -> \aq{\product\ \expr[1]} \\
          \tabTTTTT \ldots \\
          \tabTTTTT | \ensuremath{e^p_i}\ -> \aq{\product\ \expr[i]} \\
          \tabTTTT ) %>>= fun \var[v]\ -> \\
          % \tabTTT  update \aq{\tsel\ \expr\ $\downarrow$} >>= fun () -> \\
%          \tabTTT unit \var[v]
        }
      }
      \fi
    \end{minipage}
  \end{minipage}
  \caption{The transformation function \product.  
    The operator \quo{$\;\cdot\;$} emits a \miniocaml\ expression after evaluating
    expressions that are embedded using~\aq{$\;\cdot\;$}.  }
  \label{fig-prod-alg}
\end{figure}
\iffalse  

\begin{figure}[t]
  %
  \begin{minipage}[t]{\linewidth}
    \linespread{1.2}
    \begin{minipage}[t]{.04\linewidth}
      \small
      \mbox{} \\
      \gutternumbering{31}{48}
    \end{minipage}
    \begin{minipage}[t]{.94\linewidth}
      \small
      \mbox{} \\
      % \algCase{\expr[f]\ \expr[p]\ } \\
      % \tabT \algLet \var[f], \var[p]\ = \freshvar, \freshvar\ \algIn \\
      % \tabTTT\hspace{-2.3ex}
      % \quo{
      %   \ttt{
      %     \aq{\product\ \expr[p]} >>= fun \var[p] -> \\
      %     \tabTTT \aq{\product\ \expr[f]} >>= fun \var[f] -> \\
      %     \tabTTT \var[f]\ \var[p]
      %   }
      % }
      % \\
      \algCase{\funexpr\ } %\\      \tabTT\hspace{-3ex}
      \quo{
        \ttt{
          % update \aq{\tsel\ \expr\ $\uparrow$} >>= fun () ->\\
          % \tabTT update \aq{\tsel\ \expr\ $\downarrow$} >>= fun () -> \\
          % \tabTT
          unit (fun \var\ -> \aq{\product\ \expr[b]})
        }
      }
      \\
      \algCase{ \ttt{let \patt\ = \expr[1]\ in \expr[2]}\ } \\
%      \tabTT \algLet \var[v]\ = \freshvar\ \algIn \\
      \tabTTT\hspace{-2.3ex}
      \quo{
        \ttt{
          % update \aq{\tsel\ \expr\ $\uparrow$} >>= fun () -> \\
          % \tabTTT\hspace{-0.5ex} 
          ( \aq{\product\ \expr[1]} %\hspace{2.5ex}
          >>= fun \patt\ -> %\\  \tabTTTT 
          \aq{\product\ \expr[2]} ) % >>= fun \var[v] -> \\
          % \tabTTT update \aq{\tsel\ \expr\ $\downarrow$} >>= fun () -> \\
 %         \tabTTT unit \var[v]
        }
      }
      \\
      \algCase{ \ttt{let rec \var[f] = \funexpr\ in \expr[2]}\ } \\
%      \tabTT \algLet \var[v]\ = \freshvar\ \algIn \\
      \tabTTT\hspace{-2.3ex}
      \quo{
        \ttt{
          % update \aq{\tsel\ \expr\ $\uparrow$} >>= fun () ->\\
          % \tabTTT\hspace{-0.5ex}
          ( let rec \var[f] = fun \var\ -> \aq{\product\ \expr[b]} 
          \\ \tabTTTT 
          in \aq{\product\ \expr[2]} ) % >>= fun \var[v]\ -> \\
          % \tabTTT update \aq{\tsel\ \expr\ $\downarrow$} >>= fun () -> \\
%          \tabTTT unit \var[v]
        }
      }
      \\
      \algCase{ 
        \ttt{match \expr[m]\ with} \\
        \ttt{\tabT | \ncstrpatt{1}\ -> \expr[1]} \\
        \ttt{\tabT \ldots} \\
        \ttt{\tabT | \ncstrpatt{i}\ -> \expr[i]}
      } \\
      \tabTT \algLet \var[m]
%      , \var[v]
      \ = \freshvar
%      , \freshvar
      \ \algIn \\
      \tabTTT\hspace{-2.3ex}
      \quo{
        \ttt{
          % update \aq{\tsel\ \expr\ $\uparrow$} >>= fun () -> \\
          % \tabTTT
          \aq{\product\ \expr[m]} >>= fun \var[m]\ -> \\
          \tabTTT  ( match \var[m] with \\
          \tabTTTT | \ncstrpatt{1}\ -> \aq{\product\ \expr[1]} \\
          \tabTTTT \ldots \\
          \tabTTTT | \ncstrpatt{i}\ -> \aq{\product\ \expr[i]} \\
          \tabTTT ) %>>= fun \var[v]\ -> \\
          % \tabTTT  update \aq{\tsel\ \expr\ $\downarrow$} >>= fun () -> \\
%          \tabTTT unit \var[v]
        }
      }
      \\
    \end{minipage}
  \end{minipage}
  \caption{The product construction function \product.  It takes as
    input an expression \expr.
    The output is the product of \expr.  The operator
    \quo{$\;\cdot\;$} emits a \miniocaml\ expression after evaluating
    the embedded programs~\aq{$\;\cdot\;$}.  }
  \label{fig-prod-alg}
\end{figure}
\fi
%%% Local Variables: 
%%% mode: latex
%%% TeX-master: "main"
%%% End: 


\paragraph{\bf Transformation of expressions}


We present the transformation function \product\ in
Figure~\ref{fig-prod-alg}. 
\product uses two auxiliary functions $\tselEnter$ and $\tselExit$
shown in Figure~\ref{fig-tsel}.

For a expression \expr, \product\ traverses the abstract syntax tree
of \expr\ and gives a core monadic expression that evaluates the user
program together with two state transform operations.
\product generates \miniocaml expressions using the $\quo{\cdot}$
function.
For example, $\quo{\ttt{let x = 1 in 1}}$ emits the expression
\ttt{let x = 1 in 1}.
Within $\quo{\cdot}$ we can perform an evaluation by
applying~$\aq{\cdot}$.
For example, $\quo{\ttt{let x = \aq{\quo{1+2}} in 1}}$ emits $\ttt{let
x = 1+2 in 1}$.

The important case is the tranformation of
$\theFunction/\theArity$-applications.
Such applications are recognized in $\tselEnter$ and $\tselExit$.
The emitted code either saves the argument values into the state, or
propagates further the current state.
Furthermore, $\tselEnter$ performs a check if the snapshot stored in a
state together with the arguments of a
$\theFunction/\theArity$-application satisfy the transition invariant
candidate~$\theTI$.
This check is guarded by the condition that the snapshot must have
been stored previously.

We show an example application of \product in Figure~\ref{fig-ex-prod}
for analysing $\texttt{f}/1$-applications.
First, we present subexpressions of the program and then show the
result of the application of \product on them (we have partially
simplified the transformed expressions to improve readability).
%
\begin{figure}[!t]
  \centering
{
\small
\begin{verbatim}
e = let rec f x = if x > 0 then f (x - 1) else fun y -> f x y in f 1
e1 = if x > 0 then f (x - 1) else fun y -> f x y
e2 = x > 0
e3 = 0
e4 = (>) x
e5 = x
e6 = (>)
e7 = f (x - 1)
...

Transform e = let rec f = fun x -> 'Transform e1' in 'Transform e20'
Transform e1 = 'Transform e2' >>= fun x2 -> (if x2 then 'Transform e7'
                                             else 'Transform e14')
Transform e2 = fun s_full -> ('Transform e3' >>= fun x3 ->
                              'Transform e4' >>= fun x4 ->
                              update (fun s -> s) >>= fun () s_partial ->
                             (x4 x3 >>= fun xapp2 ->
                              update (fun s -> s) >>= fun () ->
                              unit xapp2) s_partial) s_full
Transform e3 = unit 0
Transform e4 = fun s_full -> ('Transform e5' >>= fun x5 ->
                              'Transform e6' >>= fun x6 ->
                              update (fun s -> s) >>= fun () s_partial ->
                             (x6 x5 >>= fun xapp4 ->
                              update (fun s -> s) >>= fun () ->
                              unit xapp4) s_partial) s_full
Transform e5 = unit x
Transform e6 = unit (fun z1 -> unit (fun z2 -> unit (z1 > z2)))
Transform e7 = fun s_full -> ('Transform e8' >>= fun x8 ->
                              'Transform e13' >>= fun x13 ->
                              update (fun ( _, m ) ->
                                let m_c, m_1 = m in
                                if m_c then assert ( x8 > 0 && m_1 > x8 );
                                x8, if not m_c && nondet () then true, x8 else m
                              ) >>= fun () s_partial ->
                             (x13 x8 >>= fun xapp7 ->
                              update (fun _ -> s_full) >>= fun () ->
                              unit xapp7) s_partial) s_full
...
\end{verbatim}
}
  \caption{Example application of \product\ with $N = 1$ and \mbox{$\theTI = (\
      a_1 > 0 \wedge m_1 > a_1 \ )$}. Given a stored snapshot, the
    transformed application \ttt{f (x - 1)} checks that the current
    actual satisfies $\theTI$.}
  \label{fig-ex-prod}
\end{figure}
%

We establish a relationship between augmented evaluation trees and
evaluation trees of the transformed program in the following theorem.
%
\begin{theorem}[\product implements \monitor]
\label{thm-product}
%
A pair $(v_1, \dots, v_\theArity)$ and~$(u_1, \dots, u_\theArity)$ is
obtained from the augmented evaluation tree as described in
Theorem~\ref{thm-monitor}
if and only if a judgement of the following form appears in the
evaluation tree of the
program obtained by applying \product:
%
\begin{center}
  \mevj{\evj{\ectx}{$\theFunction$\ \expr[1]\ \_\!\_\ \dots\
      \expr[\theArity]\ s}{\val}}{\stupa}{}
\end{center}
%
such that $\tsf{eval}\ \ectx\ s = (true, (v_1, \dots, v_\theArity))$
and for each $i \in 1..\theArity$ we have \mbox{$\tsf{eval}\ \ectx\ \expr[i]
= u_i$}.
%
\end{theorem}
%

The following corollary of Theorem~\ref{thm-product} allows one to
rely on the assertion validity in the transformed program to implement
the binary reachability analysis of the original program.
%
\begin{theorem}[Binary reachability analysis as assertion checking]
\label{thm-assert}
%
Each pair $(v_1, \dots, v_\theArity)$ and~$(u_1, \dots, u_\theArity)$
in the $\theFunction/\theArity$-recursion relation of the program 
satisfies $\theTI$ if and only if the assertion inserted by
$\tselEnter$ is valid in the transformed program.
%
\end{theorem}
%


%%% Local Variables:
%%% mode: latex
%%% TeX-master: "main"
%%% End: 

%\section{Transformation + assertion}
\label{sec-trans-assert}

\begin{theorem}[T3]
  \todo{TODO}
\begin{verbatim}
Binary reachability analysis iff assertion validity on reachability relation.
\end{verbatim}
\end{theorem}



%%% Local Variables:
%%% mode: latex
%%% TeX-master: "main"
%%% End: 

\section{Experiments}

To evaluate the described methods, we implemented them in a tool called
\emph{Petrinizer}. Petrinizer is essentially a set of Bash and Prolog scripts
that mediate between the input, specified in the input format of
MIST2\footnote{\url{https://github.com/pierreganty/mist}}, and the SMT
solver Z3\footnote{\url{http://z3.codeplex.com/}} \cite{DeMouraTACAS08}. With
the choice of Bash and Prolog over languages with bindings for Z3 API, we
restricted Z3's full power. Indeed, by using Z3 as an external tool, we
were not able to fully exploit the incremental nature of its linear arithmetic
solver (TODO: cite appropriate papers). However, even the non-optimal
implementation turned out to be robust and scalable, with the main advantage of
being able to tackle problem instances that are out of reach for
state-of-the-art coverability checkers.

The evaluation of Petrinizer had three main goals. First, we wanted to compare
its performance against state-of-the-art tools like MIST2, BFC
and IIC (cite!). Second, as the method it
implements is incomplete, we wanted to measure its rate of success on safe problem instances. As a
subgoal, we wanted to investigate the usefulness and necessity of traps, as
well as to what extent real arithmetic suffices over integer arithmetic in
proving safety. And last, since the language of linear arithmetic, used by
Petrinizer, allows for a more succint representation of formulas than the
language used by IIC, we wanted to compare sizes of invariants the two tools produce.

Inputs that were used in tests come from various sources (TODO: write a
sentence to justify the variety).
One source is the collection of Petri nets from the literature that is part of the MIST2 toolkit.
The second source are Petri nets originating from the analysis of concurrent C
programs that were used in the evaluation of BFC \cite{KaiserCONCUR12}. Then
there are inputs originating from the provenance analysis of a medical and a
bug-tracking system \cite{MajumdarSAS13}. Finally, we generated a set of Petri
nets from Erlang programs, using an Erlang verification tool Soter
\cite{DOsualdoSAS13}. (TODO: Acknowledge Emanuele D'Osualdo for help.)

All experiments were performed on the identical machines, equipped with
Intel Xeon 2.66 GHz CPUs and 48 GB of memory, running Linux 3.2.48.1 in 64-bit
mode. Execution time was limited to 100,000 seconds (27 hours, 46 minutes and
40 seconds), and memory to 2 GB.

\subsection{Performance}

Explain what we observe:
\begin{itemize}
  \item On small examples IIC mostly outperforms Petrinizer. Probably due
    to a slower parser, setting up and invoking Z3.
  \item On examples of a similar size, coming from the same source,
    unlike with IIC, resource consumption is fairly uniform. (Medical examples.)
  \item Petrinizer is able to handle some huge examples that are out of
    scope for IIC.
  \item There were no cases when IIC was able to finish, but Petrinizer ran
    out of resources. As a matter of fact, Petrinizer finishes in all cases,
    whereas other tools somethimes fail.
\end{itemize}

\emph{Invariant sizes.} Again, explain how we measure it. For IIC, we measure
the number of non-null literals, the number of clauses, total length of
the invariant in characters. For Petrinizer, we measure the number of
non-null coefficients in the linear expressions and the total length of the
invariant in characters. Argue why it makes sense to compare any of these
quantities. Point out the difference in size with and without the invariant
minimization.
  
\emph{Rate of success.} Point out that Petrinizer mostly successfully proves
safety. Specifically, it proves safety in all of the safe Soter examples.
Point out the cases where traps were actually useful, and potentially progress
into some pseudo-philosophical discussion on what constitutes a good set of examples.


\section{Related work}
\label{sec:relwork}

% \todo{TODO: WHAT THEY DO, WHAT WE CAN DO THAT THEY CAN, WHAT WE CAN DO
%   THAT IS AN IMPROVEMENT OVER THEM}


\newcommand{\oldRelWork}{false}
\ifthenelse{\equal{\oldRelWork}{true}}{

In this section we describe existing techniques for the
analysis and verification of functional programs.

\newcommand{\lfts}{\ensuremath{\tsf{LF}^\diamond}\xspace}
\paragraph{Resource bounds analysis.}

Hofmann et. al.~\cite{Hofmann11,Hofmann03} present type systems and
corresponding type inference algorithms for the analysis of
quantitative resource bounds.  The computed bounds are represented by
linear or polynomial expressions obtained by solving a linear
program based encoding of the type constraints.  The type system
keeps track of the resource consumption using an augmentation of type
judgements by resource counters.  These counters can be viewed as
monitor states that are embedded into type judgements.  In
contrast, our monitor states are embedded into the program.

The type inference algorithm of \cite{Hofmann11} can be used in combination
with our product construction to infer quantitative properties of
monitored programs. 
On the other hand, extending \cite{Hofmann11} to keep track of other
properties requires a redesign of the type system and the invention of
the corresponding inference algorithm.



\paragraph{Resource usage verification.}
The resource usage verification problem consists in checking that the
resource access traces generated by a program satisfy a
specification~\cite{Kobayashi09}.  Several works address the problem
of resource usage verification under different names,
e.g.~\cite{Terauchi02,Kobayashi05,Kobayashi09,Kobayashi11}.

Kobayashi~\cite{Kobayashi09} presents an algorithm for checking
properties given as finite state tree automata over a finite alphabet.
In constrast, we do not give a checking algorithm since we deal with
infinite state systems. Instead, we rely on the
reachability/termination tool \dsolve~\cite{Dsolve}. Our monitors can
check non-regular properties with the help of a
reachability/termination backend.

Foster~et.~al.~\cite{Terauchi02} presents a type inference system
for checking properties given as assume/assert program
annotations. Their annotations are over atomic predicates on program
values called \emph{type qualifiers}. Their type inference system is
similar to the liquid type inference system of \dsolve. In contrast to
our work, they can handle functional programs with references.

% addresses the resource usage verification
% problem for higher-order functional programs.
% % The resource usage
% % verification problem consists in verifying that the resource access
% % traces given by a program satisfy a specification. A specification is
% % a language of infinite words on a finite alphabet.
% Kobayashi gives a novel model checking technique for finite-state
% programs. Their technique applies a type checking approach to the
% model checking of higher-order recursion schemes corresponding to user
% programs. The technique is applied on $\omega$-regular safety
% properties. In contrast, our work enables reasoning about temporal
% properties of infinite-state programs using existing
% techniques. Combining their technique with approach is an interesting
% research goal.


% Our tool can verify the examples given in
% Section~3.1 of~\cite{Kobayashi09} (see experiments 15-18,43,44 in Figure~\ref{fig-experiments}) but we failed to verify the 2-place
% buffer in Example~6.1 of~\cite{Kobayashi09} because constructing the
% corresponding monitor specification requires ingenuity. Our work and
% Kobayashi's are different in that our we provide a framework for
% expressing temporal properties that can be verified with existing
% tools and Kobayashi developed a full verification framework.
 
% Kobayashi \cite{Kobayashi09} applied HORS as an abstract model of
% higher order functional programs. Kobayashi presented a type based
% algorithm for checking that HORS satisfy finite state machine-based
% specifications. There are two main differences to our work. First, we
% target \ocaml\ programs, while Kobayashi's considers a prototype
% functional language. Second, we deal with properties represented by
% infinite state monitors.


\paragraph{Contract checking.}

Contracts are pre- and post-condition specifications for
functions. Xu created a verification tool \cite{DanaPhd} for Haskell that is
based on contracts and partial evaluation. 
Pre- and post-conditions can only refer to the program execution
before and after procedure calls. In contrast, our monitor
specifications have the ability of observing program executions at all times.

\paragraph{Liquid typing.}

\dsolve\ \cite{Dsolve} is a reachability checker based on refinement
type inference.  The type inference algorithm consists of two
parts. First, a set of constraints over refinement predicates is
generated from program code. Second, an iterative algorithm tests
candidate solutions constructed from a set of predicate schemes in the
theory of linear arithmetic and uninterpreted functions. While \dsolve\
is effective for the verification of assertion validity, our product
construction allows \dsolve\ to verify arbitrary safety properties.



}{

\paragraph{Termination and control-flow analysis}

Traditionally, termination analysis of higher-order programs is
developed on top of a control-flow
analysis~\cite{Sereni05terminationanalysis,SereniAPLAS05,SereniICFP07}.
Our approach relies on the applied safety checker to keep track of
control flow. 
In principle, our transformation could benefit from the results of
control-flow analysis, as discussed in Section~\ref{sec:conclusion}. 
Practically, such additional information was not necessary when
proving termination of all examples presented
in~\cite{Sereni05terminationanalysis,SereniAPLAS05,SereniICFP07}.
Adding a control-flow analysis
pass~\cite{Shivers88,MightJFP08,MightPOPL11,MightSCHEME2010} before the transformation
would be akin to the application of (function) pointer analysis for
imperative programs. 
We leave a study of such an integration as future work.

  \paragraph{Abstraction for termination.} The size change termination
  argument~\cite{SCTPOPL2001} can be extended to higher-order functional
  programs, see e.g.~\cite{JonesBohr04,SereniAPLAS05}. This argument requires checking the presence of an
  infinite descent in values passed to application sites of the program
  on any infinite traversal of the call graph. In contrast, our
  approach can keep track of a rank descent in arbitrary expressions
  over program values.
  In principle, SCT can be seen a specific abstract domain that yields
  termination arguments related to disjunctive
  well-foundedness~\cite{PodelskiSAS10}.
  We leave the question if general abstraction techniques for
termination~\cite{CousotHigherOrder94,CousotPOPL12} can be reduced by
an appropriate source to source transformation as furture work.


\paragraph{Contract checking.}

Contracts are pre- and post-condition specifications for functions. Xu
created a verification tool \cite{DanaPhd} for Haskell that is based
on contracts and partial evaluation.  Their approach works well for
checking safety properties, and can even detect divergence of
programs. In contrast, our approach is specialized to the verification
of termination, and is a step towards the automated verification of
termination through counterexample guided abstraction refinement.

\paragraph{Liquid typing.}

\dsolve\ \cite{Dsolve} is a reachability checker based on refinement
type inference.  The type inference algorithm consists of two
parts. First, a set of constraints over refinement predicates is
generated from program code. Second, an iterative algorithm tests
candidate solutions constructed from a set of predicate schemes in the
theory of linear arithmetic and uninterpreted functions. As our
experiments show, the composition of our transformation and \dsolve\ 
yields a binary reachability analysis tool.

}

%%% Local Variables: 
%%% mode: latex
%%% TeX-master: "main"
%%% End: 

\section{Conclusions}

TODO


\bibliographystyle{abbrv}
\bibliography{references}

\end{document}

%%% Local Variables: 
%%% mode: latex
%%% TeX-master: t
%%% End: 
