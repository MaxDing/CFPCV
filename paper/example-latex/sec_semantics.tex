\section{Binary reachability on evaluation trees}
\label{sec-semantics}

In this section we make the first step towards our program
transformation.
We present an augmentation of evaluation trees that allows us to
reduce the binary reachability analysis to the validity analysis of
annotated judgement.
Each judgement is augmented with a Boolean and an $\theArity$-ary
tuple of values, which we will refer to as a \emph{state}.

\newcommand{\stf}{\ensuremath{(\mathit{false}, 0)}}
\newcommand{\stzero}{\ensuremath{(\mathit{true}, 0)}}
\newcommand{\stone}{\ensuremath{(\mathit{true}, 1)}}
\newcommand{\vstf}{\ensuremath{\st[f]}}
\newcommand{\vstzero}{\ensuremath{\st[0]}}
\newcommand{\vstone}{\ensuremath{\st[1]}}


\newcommand{\vf}{\ensuremath{(\ttt{f},
    \ttt{fun~x~->~if~x~>~0~then~f~(x~-~1)~else~fun~y~->~y~+~1}, \emptyectx)}}
\newcommand{\vvf}{\val[f]}
\newcommand{\vfx}{\ensuremath{(\ttt{fun~y~->~y~+~1}, [\ttt{f} \emapsto \vvf; \ttt{x} \emapsto 0])}}
\newcommand{\vvfx}{\val[\ttt{f\ x}]}
% \newcommand{\ectxf}{\ensuremath{\ttt{f} \emapsto \vf}}
\newcommand{\vectxf}{\ectx[\ttt{f}]}
\newcommand{\vectxfxone}{\ectx[\ttt{f\ 1}]}
\newcommand{\vectxfxzero}{\ectx[\ttt{f\ 0}]}

\begin{figure}[t]
\begin{minipage}{\linewidth}
  \small

    \[
    \scalebox{.95}{
      \inference{
        \begin{sideways}
        \inference{}{
          \mevj{\evj{\vectxf}{\ttt{1}}{\ttt{1}}}{}{\vstf}
        }
        \end{sideways}
        &
        \begin{sideways}
        \inference{}{
          \mevj{\evj{\vectxf}{\ttt{f}}{\vvf}}{}{\vstf}
        }
        \end{sideways}
        &
        \inference{
          \begin{sideways}
          \inference{
            \vdots
          }{
            \mevj{\evj{\vectxfxone}{\ttt{x~>~0}}{\ttt{true}}}{}{\vstone}
          }
          \end{sideways}
          &
          \inference{
            \begin{sideways}
            \inference{
              \vdots
            }{
              \mevj{\evj{\vectxfxone}{\ttt{x~-~1}}{\ttt{0}}}{}{\vstone}
            }
            \end{sideways}
            &
            \begin{sideways}
            \inference{
            }{
              \mevj{\evj{\vectxfxone}{\ttt{f}}{\vvf}}{}{\vstone}
            }
            \end{sideways}
            &
            \inference{
              \begin{sideways}
              \inference{
                \vdots
              }{
                \mevj{\evj{\vectxfxzero}{\ttt{x~>~0}}{\ttt{true}}}{}{\vstone}
              }
              \end{sideways}
              &
              \begin{sideways}
              \inference{
              }{
                \mevj{\evj{\vectxfxzero}{\ttt{fun~y~->~y~+~1}}{\vvfx}}{}{\vstone}
              }
              \end{sideways}
            }{
              \mevj{\evj{ \underbrace{\vectxf \eext \ttt{x} \emapsto \ttt{0}}_{\vectxfxzero} }{\ttt{if~\ldots}}{\vvfx}}{}{\vstone}
            }
          }{
            \mevj{\evj{\vectxfxone}{\ttt{f (x - 1)}}{\vvfx}}{}{\vstone}
          }
        }{
          \mevj{\evj{ \underbrace{\vectxf \eext \ttt{x} \emapsto \ttt{1}}_{\vectxfxone} }{\ttt{if~x~>~0~then~\ldots}}{\vvfx}}{}{\vstone}
        }
      }{
        \mevj{\evj{
            \vectxf 
            % \underbrace{
            %   \ttt{f} \emapsto \underbrace{\vf}_{\vvf}
            % }_{\vectxf} 
          }{\ttt{f\ 1}}{\underbrace{\vfx}_{\vvfx}}}{}{\vstf} 
      }
    }
   \]

   \iffalse
    % 
    \begin{center}
      (a) 
    \end{center}
    \vspace{2ex}
    % 
    \[
    \scalebox{.45}{
      \inference{
        \inference{}{
          \mevj{\evj{\vectxf}{\ttt{1}}{\ttt{1}}}{}{\vstf}
        }
        &
        \inference{}{
          \mevj{\evj{\vectxf}{\ttt{f}}{\vvf}}{}{\vstf}
        }
        &
        \inference{
          \inference{
            \vdots
          }{
            \mevj{\evj{\vectxfxone}{\ttt{x~>~0}}{\ttt{true}}}{}{\vstf}
          }
          &
          \inference{
            \inference{
              \vdots
            }{
              \mevj{\evj{\vectxfxone}{\ttt{x~-~1}}{\ttt{0}}}{}{\vstf}
            }
            &
            \inference{
            }{
              \mevj{\evj{\vectxfxone}{\ttt{f}}{\vvf}}{}{\vstf}
            }
            &
            \inference{
              \inference{
                \vdots
              }{
                \mevj{\evj{\vectxfxzero}{\ttt{x~>~0}}{\ttt{true}}}{}{\vstzero}
              }
              &
              \inference{
              }{
                \mevj{\evj{\vectxfxzero}{\ttt{fun~y~->~y~+~1}}{\vvfx}}{}{\vstzero}
              }
            }{
              \mevj{\evj{ \underbrace{\vectxf \eext \ttt{x} \emapsto \ttt{0}}_{\vectxfxzero} }{\ttt{if~x~>~0~then~\ldots}}{\vvfx}}{}{\vstzero}
            }
          }{
            \mevj{\evj{\vectxfxone}{\ttt{f (x - 1)}}{\vvfx}}{}{\vstf}
          }
        }{
          \mevj{\evj{ \underbrace{\vectxf \eext \ttt{x} \emapsto \ttt{1}}_{\vectxfxone} }{\ttt{if~x~>~0~then~\ldots}}{\vvfx}}{}{\vstf}
        }
      }{
        \mevj{\evj{\underbrace{\ttt{f} \emapsto \underbrace{\vf}_{\vvf}}_{\vectxf}}{\ttt{f\ 1}}{\underbrace{\vfx}_{\vvfx}}}{}{\vstf}
      }
    }
    \]
    \begin{center}
      (b)
    \end{center}
    \fi
  \end{minipage}

  \caption{Annotation of the evalution tree for \ttt{f\ 1} in the 
     environment \vectxf\ = \vf.
     The initial judgement is annotated with $\vstf = \stf$. 
     The annotation changes from $\vstf$ to $\vstone = \stone$ 
     after the first call to \ttt{f}.
     Due to the lack of width, we had to turn several judgements.
   }

%    $\vectxf  = \vf$
            % \underbrace{
            %   \ttt{f} \emapsto \underbrace{\vf}_{\vvf}
            % }_{\vectxf} 



  \label{fig-monitoring}
\end{figure}

%%% Local Variables:
%%% mode: latex
%%% TeX-master: "main"
%%% End: 
 

Before presenting the augmentation procedure, we consider examples of
tree augmentation shown in Figure~\ref{fig-monitoring}.
The root of the tree is augmented with a state $\vstf = (\lfalse, 0)$,
where $\lfalse$ indicates that no argument has been used for the
augmentation yet.
We use $\vst$ to augment judgements in the subtree for the branches
that do not correspond to the evaluation of the body of~\texttt{f}.
When augmenting the subtree that deals with the body, we can
nondeterministically decide to start augmenting with a state that
records the argument of the current application.
That is, in the body subtree we augment with the state $\vstone =
\stone$.
Here, $\ltrue$ indicates that we took a snapshot of the current
application argument, and $1$ is the argument value.
The remaining judgements are augmented with $\vstone$, since we will
not change the snapshot if it was taken, i.e., if the first component
of the augmenting state is~$\ltrue$.

We proceed with an algorithm \monitor that takes as input an initial
state and an evaluation tree and produces an augmented tree.
Each augmented judgement is of the form
\mevj{\evj{\ectx}{\expr}{\val}}{\stupa}{\st}, where \st\ is a state.
As an \emph{initial} state we take a pair $(\lfalse, (v_1 \dots,
v_\theArity))$ where $v_1\ \dots, v_\theArity$ are some values.

See Figure~\ref{fig-monitor-alg}.
\monitor traverses the input tree recursively, by starting from the
root.
Whenever the current judgement is a
$\theFunction/\theArity$-application judgement, then we choose whether
to create a snapshot of the arguments and store them in the state that
is used to augment the subtree that evaluates the body.
We only create a snapshot if the Boolean component of the current
state is~$\lfalse$.
In case we currently do not deal with a
$\theFunction/\theArity$-application judgement, no state change
happens and we proceed with the subtrees.
Once we obtain the augmented versions of the subtrees, we put them
together by creating a node that connects the roots of the subtrees.


  % \begin{tabular}{r@{\ }l l}
  %   $\mevjs \ni$ & \mevjvar \;\; ::=
  %   \mevj{\evj{\ectx}{\expr}{\val}}{\stupa}{\stdown}
  %   & monitored judgement \\[\jot]
  %   $2^{{\mevjs}^+} \ni$ & \mevt  & monitored evaluation tree
  % \end{tabular}

\begin{figure}[t]
  \centering
  \begin{minipage}[t]{\linewidth}
    \linespread{1.2}
    \begin{minipage}[t]{.03\linewidth}
      \small
      \gutternumbering{1}{13}
    \end{minipage}
    \begin{minipage}[t]{.95\linewidth}
      \small
      \algLet \monitor\ ((c, \_\!\_) \textbf{as} \st)\ \evt\ = \\
      \tabT \algLet \rt\ = \tsf{root}\ \evt\ \algIn \\
      \tabT \algMatch \rt\ \algWith\\
      \tabT \algCase{\evj{\ectx}{\theFunction\ \expr[1]\ \dots\
          \expr[\theArity]}{\val}} \\
      \tabTT \algLet \val[1], \dots, \val[\theArity]  = 
      \tsf{eval}\ \ectx\ \expr[1], \dots,
      \tsf{eval}\ \ectx\ \expr[\theArity]\ \algIn \\
      \tabTT \algLet \st'\ = 
      \algIf $\neg c \land \textbf{nondet()}$ \algThen
      ($\ltrue$, \val[1], \dots, \val[\theArity])
      \algElse \st\ \algIn \\
      \tabTT \algLet \evt[p], \evt[f], \evt[b] = 
      immediate subtrees of \evt\ \algIn \\
      \tabTT \algLet \evtP[p], \evtP[f], \evtP[b]\ = 
      \monitor\ \st\ \evt[p], \monitor\ \st\ \evt[p], \monitor\ \st'\ \evt[b]\ \algIn\\
      \tabTT $(\setOf{\tsf{root}\ \evtP[p], 
        \tsf{root}\ \evtP[f], \tsf{root}\ \evtP[b], (\rt, \st))} 
      \cup \evtP[p] \cup \evtP[f] \cup \evtP[b]
      $ \\
      \tabT \algCase{\_\!\_} \\
      \tabTT \algLet \evt[1], \dots, \evt[n] = 
      immediate subtrees of \evt\ \algIn \\
      \tabTT \algLet \evtP[1], \dots, \evtP[n]\ = 
      \monitor\ \st\ \evt[1], \dots, \monitor\ \st\ \evt[n]\ \algIn\\
      \tabTT $(\setOf{\tsf{root}\ \evtP[1], \dots,
        \tsf{root}\ \evtP[n], (\rt, \st))} 
      \cup \evtP[1] \cup \dots \cup \evtP[n]
      $ 
   \end{minipage}
 \end{minipage}
 \caption{Evaluation tree monitoring.  The input consists of a
    monitor state $\st$ and an evaluation tree~\evt.
    $\textbf{nondet()}$ non-deterministically returns either $\ltrue$
    or $\lfalse$.
  }
  \label{fig-monitor-alg}
\end{figure}
%%% Local Variables: 
%%% mode: latex
%%% TeX-master: "main"
%%% End: 


We establish a formal relationship between the
$\theFunction/\theArity$-recursion relation with the augmented
judgements obtained by applying \monitor using the following theorem.
%
\begin{theorem}[\monitor keeps track of
$\theFunction/\theArity$-recursion relation]
\label{thm-monitor}
%
A pair $(v_1, \dots, v_\theArity)$ and~$(u_1, \dots, u_\theArity)$ is
in the $\theFunction/\theArity$-recursion relation if and only if the
result of applying \monitor wrt.\ some sequence of nondeterministic
choices on \evjvar and an initial state contains an augmented
judgement of the form
%
\begin{center}
  \mevj{\evj{\ectx}{$\theFunction$\ \expr[1] \dots
      \expr[\theArity]}{\val}}{\stupa}{
    (\ltrue, (v_1, \dots, v_\theArity))}
\end{center}
%
such that for each $i \in 1..\theArity$ we have $\tsf{eval}\ \ectx\
\expr[i] = u_i$
%
\end{theorem}
%

%%% Local Variables:
%%% mode: latex
%%% TeX-master: "main"
%%% End: 
